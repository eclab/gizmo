\synctex=1

%%%% Copyright 2016 by Sean Luke
%%%% Distributed Under the Apache 2.0 License


\documentclass{article}
\usepackage{fullpage}
\usepackage{mathpazo}
\usepackage{microtype}
\usepackage{graphicx}
\usepackage{wrapfig}
\usepackage{amsmath}
\usepackage{amssymb}
\usepackage{bm}
\usepackage{array}
\usepackage{color}
\usepackage{multicol}
\usepackage[svgnames]{xcolor}
\usepackage[hyperfootnotes=false,linktocpage=true,linkbordercolor={0.5 0 0}]{hyperref}
\usepackage{accsupp}
\hypersetup{
    colorlinks,
    linkcolor={red!50!black},
    citecolor={blue!50!black},
    urlcolor={blue!80!black}
}%%% Note that to avoid a link being created from \pageref, just use \pageref*
%%% End hyperref stuff

\newcommand\ignore[1]{}
\newcommand\bump{\vspace{11in}}


\sloppy

\newcommand\myfrac[2]{#1/#2}
\begin{document}

\noindent {\Huge\bf Gizmo}\\[0.5em]
{\large \bf An Arduino MIDI Tool, Version 8\\
By Sean Luke\quad sean@cs.gmu.edu} 
\hspace{\fill}{\it \textbf{For Aric}}\\

\vspace{-1em}
\setcounter{tocdepth}{2}
\tableofcontents
\clearpage

\section{About Gizmo}

\noindent Gizmo is a small MIDI device intended to be inserted between the MIDI OUT of one device and the MIDI IN of one or more other devices.  There are a number of tools like this, such as the MIDIPal\footnote{Though they share nothing in common, MIDIPal is similar to Gizmo in functionality and goals.  But I wasn't even aware of the existence of MIDIPal when I started building Gizmo.  And besides, re-inventing the wheel is fun!  I've since ripped off a few of MIDIPal's features.  MIDIPal's page is http:/\!/mutable-instruments.net/midipal\ \ This device has been discontinued, but its open source software lives on in other hardware clones, such as the MIDIGal (https:/\!/midisizer.com/midigal/) or MIDIBro (http:/\!/www.audiothingies.com/product/midibro/)} or a variety of tools from MIDISolutions.\footnote{http:/\!/www.midisolutions.com/}

Gizmo is the unholy union of an Arduino Uno or (much better) an Arduino Mega 2560,\footnote{http:/\!/arduino.cc} an Adafruit 16x8 I2C LED matrix,\footnote{https:/\!/www.adafruit.com/categories/326\quad I use the 1.2-inch 16x8 matrix with red round LEDs, which works well.    I find that the green LED options tend to make noise.  This particular model can be found at https:/\!/www.adafruit.com/products/2037} and a SparkFun MIDI Shield\footnote{https:/\!/www.sparkfun.com/products/12898} to perform a variety of helpful MIDI tasks.  Gizmo has several built-in applications, and assuming there's space you can develop additional applications for it.

This is a software project: the hardware is just three off-the-shelf boards and four wires.  There's no enclosure, no custom board.  If you'd like to make an enclosure, I'd love to see it!  In fact the Gizmo web page\footnote{http:/\!/cs.gmu.edu/\(\sim\)sean/projects/gizmo/} has a number of examples.

Gizmo comes built-in with a number of applications and other capabilities.  All of these capabilities are modular, because the Arduino Uno cannot handle many of them at one time.  In previous versions of Gizmo, I attempted to cram multiple simplified versions of these applications on the Uno: but nowadays Gizmo permits only a single application at a time on the Uno (see the file \texttt{All.h}).  And the Drum Sequencer is too large to fit on the Uno at all, and the Step Sequencer can only fit in simplified form.  All of the applications combined fit just fine on the Mega (which you definitely ought to get instead of an Uno).


\paragraph{Applications}
\begin{itemize}
\item {\bf Arpeggiator.}\quad Gizmo's arpeggiator has built-in up, down, up/down, repeated chord, note-assign, and random arpeggios spanning one to three octaves.  Additionally, you can define up to ten additional arpeggiator patterns, each up to 32 notes long, involving up to 14 different chord notes, plus rests and ties.  Arpeggios can be {\it latched} (or not), meaning that they may continue to play even after you have released the keys.  You can specify  the note value relative to the tempo (ranging from eighth triplets to double whole notes), the note length as a percentage of the note value (how legato or staccato a note is), whether or not the velocity is fixed, the output MIDI channel, and the degree of swing (syncopation).  Gizmo will show the arpeggio on-screen.

\item {\bf Step Sequencer.}\quad Gizmo's step sequencer can be organized as 12 16-note tracks, 8 24-note tracks, 6 32-note tracks, 4 48-note tracks, 3 64-note tracks, or 2 96-note tracks.  Each note in a track can have its own unique pitch and velocity, or be a rest, or continue (lengthen, tie) the previous note.  You can also fix the velocity for an entire track, mute tracks, fade their volume, specify their independent output MIDI channels, specify the note value relative to the tempo (again ranging from eighth triplets to double whole notes), the note length as a percentage of the note value (how legato or staccato a note is), and the degree of swing (syncopation).   The step sequencer lets you edit in two modes: either by triggering independent steps like a classic drum sequencer, or by playing a sequence of notes.  Gizmo will show and edit sequences on-screen.   You can specify a mute-pattern for each track along four bars of the sequencer.  There's also a ``performance mode'' which allows sequence chaining, playing along, and transposition, among other things.  You can stipulate whether starting/stopping the sequencer will also start/stop the MIDI clock.  You can also have the Sequencer avoid playing notes out as they are entered.  The Arduino Uno can store up to two sequences in its slots (shared with the Recorder, discussed next).  The Mega can store up to nine sequences.  

In the {\bf advanced version of the sequencer},\footnote{This is an \texttt{\#include} option found in \texttt{All.h}} which only fits on the Mega, you can also sequence CC, RPN/NRPN, Pitch Bend, Aftertouch, or PC data in addition to note data.  The advanced version also allows tracks to be played one at a time, with repeats, in a fashion more useful when controlling a monophonic or duophonic synthesizer.


\item {\bf Drum Sequencer.}\quad The drum sequencer is similar to the step sequencer but has many features more oriented towards controlling drum synthesizers in the vein of a drum machine, rather than playing notes.  Whereas the step sequencer repeats a single 16-to-64-step sequence, the drum sequencer is has up to 15 such sequences (called {\it groups}) and can chain them together in a series of up to 20 transitions from sequence to sequence in order to define drum patterns for an entire song.  Each group can have from 1 to 64 steps, and up to 20 tracks, depending on the organization (there are 16 organization choices). Each track plays a single drum note on a single MIDI channel at a given velocity; the steps in the track simply state when the note should be played.  Like the step sequencer, you can specify a mute pattern for each track (and for each group). You have many speed options, including independent speed of groups, plus swing and other features.  Also like the step sequencer there's an edit mode which allows you to toggle steps x0x-style, and another mode which lets you enter notes in real time as the sequencer is playing.  There's also a `performance mode'' which allows chaining, playing along, etc., plus an additional ``group mode'' for changing and editing groups.  You can stipulate whether starting/stopping the sequencer will also start/stop the MIDI clock.    The drum sequencer will only fit on the Mega, and can store up to nine sequences.

\item {\bf Recorder.}\quad The Arduino doesn't have a lot of memory, but we provide a note recorder which records and plays back up to 64 notes (pitch and velocity) played over 21 measures. The recorder has 16-voice polyphony.   It's enough to record a very short ditty.  You can also set the recorder to loop the recording while playing, and to provide a click track.  The Arduino Uno can store up to two recordings in its slots (shared with the Step Sequencer).  The recorder gives you the option of auto-repeating a recording or stopping when it is finished.  The Uno can store up to two recordings; the Mega can store up to nine.

\item {\bf MIDI Gauge.}\quad The gauge will display all incoming MIDI information on one or all channels.  Note on, note off, and polyphonic aftertouch are shown with pitch and velocity (or pressure).  Channel aftertouch is shown with the appropriate pitch.  Program changes indicate the number.  Pitch bends indicate the value in full 14-bit.  Control Change, Channel Mode, NRPN, and RPN messages display the number, value, and whether the value is being set, incremented, or decremented.  Sysex, song position, song select, tune request, start, continue, stop, and system reset are simply noted.  Rapid-fire MIDI signals such as MIDI clock, active sensing, and time code just turn on individual LEDs. You can also read raw Control Change messages (that is, not parsed into NRPN etc.), and provides more useful information on Channel Mode messages.

\item {\bf MIDI Control Surface.}\quad Gizmo has two potentiometers and two buttons.  Each can be configured to send a unique Program Change, Control Change, NRPN, or RPN commands.  The potentiometers can send up to 10 bits of data (MSB + partial LSB).   You can send pitch bend, and aftertouch.  The control surface also contains an {\bf 8-stage loopable envelope} reminiscent of the wave envelope in the Waldorf Microwave and Casio CZ synthesizers, and a {\bf random LFO}, both of which can manipulate any of these commands as well.  Finally, you can {\bf bulk-send program and bank changes} to several channels up to 12 different times.

\item {\bf  Keyboard Splitter.}\quad You can split incoming MIDI notes by pitch and route them to different MIDI out channels.  You can also fade notes between two channels depending on note velocity.

\item {\bf  MIDI Thru}\quad Gizmo can do magic with incoming notes.  It can {\it merge} and {\it block} channels%, and can map CC to NRPN
.  It can also {\it distribute} notes to different MIDI channels as you play: this lets you play multiple mono synths as if they were polyphonic, for example.  Gizmo can also {\it layer} notes: when you play a note, Gizmo will send some \(N>1\) note-on messages instead of one.  This lets you use a polyphonic synthesizer to play a note with more than one voice, making it fatter, but not going all the way to unison.  Gizmo can also perform {\it chord memory}, playing a transposed chord of your choosing starting at a note you are currently playing.  Last, Gizmo can {\it debounce} notes (typically from drum pads), refusing to play a second identical note if it happens to soon after the first one.  You can do all of these simultaneously.

\item {\bf  Measure Counter.}\quad You can count elapsed beats, measures (bars), and phrases; and alternatively count eighths of a second, seconds, and minutes.  The counter also responds to external MIDI clock directives.

\item {\bf  Synth MIDI Helpers.}\quad Utilities designed to improve the MIDI of specific synthesizers.  At present Gizmo comes with helpers for the Waldorf Blofeld, Korg Microsampler, Oberheim Matrix 1000 (firmware 1.2) and Yamaha TX81Z.

\end{itemize}

\paragraph{Other Cool Stuff}

As if this weren't enough, Gizmo also comes with the following capabilities.  All of these settings are automatically saved permanently in memory.

\begin{itemize}
\item {\bf Tempo and the MIDI Clock.}\quad Gizmo can sync to an external MIDI clock, can ignore it and use its own internal clock, and can emit the same as a MIDI clock.  Gizmo can also pass the external MIDI clock through or block it.  When using its own internal clock, Gizmo supports tempos ranging from 1 to 999 BPM, settable numerically or via tapping.  You can start, pause/un-pause, and stop this clock via NRPN. Gizmo can also divide the MIDI clock, thereby outputting a slower clock than it is inputting.  And you can also start, stop, and continue clock via buttons on Gizmo or via CC.

Gizmo applications can also be set with a {\it note value} (or {\it note speed}) to play their notes relative to this clock, ranging from eighth triplets (1/24 beat) to double whole notes (8 beats).  At any time, Gizmo pulses on-screen LEDs indicating the clock and note speed pulses.  Finally, Gizmo can be configured to vary the degree of {\it swing} or {\it syncopation}.

\item {\bf In and Out MIDI Channels}\quad Gizmo can be configured with input and output MIDI channels (some applications will use these: others will have multiple channels and will treat these as defaults).  Both channels can be turned off, and the input channel can be OMNI.  Gizmo pulses on-board LEDs indicating incoming and outgoing MIDI data.  

\item {\bf Remote Control via MIDI CC or NRPN.}\quad Gizmo can be controlled remotely via CC or NRPN rather than using its on-board controls.  If you start using the controls, Gizmo's onboard potentiometers are locked out so their noise will not interfere with your remote control.  You can unlock the potentiometers by pressing a button on-board Gizmo, or by sending Gizmo's {\it Release} NRPN message.  

\item {\bf Bypass.}\quad You can quickly put Gizmo in Bypass mode, to pass through all MIDI signals and generate no new ones (with the exception of the Controller, see Section~\ref{controller}, and the ``Merge'' MIDI clock option, see Section~\ref{options}).  Putting Gizmo in Bypass mode will also send an All Notes Off message, so toggling Bypass is also a useful way to hush your synthesizer.

\item {\bf Screen Brightness.}\quad You can change the screen brightness.  

%\item {\bf  \textit{Mega} Control Voltage / Gate.}\quad You can configure Gizmo to output control voltage (CV) values from 0--5V, as well as trigger a 5V gate, and output 0--5V in response to another signal, either velocity or aftertouch, all in response to notes being played. 

\item {\bf Transposition and Volume Control.}\quad You can transpose all of Gizmo's MIDI output pitch by anywhere from \(-60...60\) steps.  Additionally you can multiply the MIDI output velocity (volume) by multiples of two ranging from \(1/8\) to \(8\).

\item {\bf Menu Delay.}\quad Many of Gizmo's menus display the first few letters of a menu item, then pause for some interval of time before they start scrolling the full item.  You can change this interval.

%\item {\bf \quad Sysex Dumps.}\quad Arpeggios, Step Sequences, and Recordings can be uploaded and downloaded to your computer.

\end{itemize} 

\clearpage

\section{Building Gizmo}

\begin{enumerate}

\item {\bf Assemble the Hardware}\quad Gizmo consists of an Arduino Uno or Arduino Mega, a SparkFun MIDI Shield\footnote{https:/\!/www.sparkfun.com/products/12898}, and an Adafruit 16x8 LED Matrix\footnote{This matrix comes in various colors LED shapes (round, square), and matrix sizes (0.8 inch, 1.2 inch).  I personally use a 1.2 inch round red matrix.  The URL for my model is https:/\!/www.adafruit.com/products/2037\quad I also have a round green matrix: it looks cool but it's significantly louder and has more electrical noise.  I prefer the red.} attached via four wires (I2C).  The SparkFun MIDI Shield does not come with headers to plug into the Arduino, and you'll need to get those too: it's assumed you'll either use plain headers, or use stackable headers.  I use stackable headers as it makes it easy for me to plug my four I2C wires into it.  See Section \ref{debouncing} for a warning about the low-quality buttons supplied with the MIDI Shield and what to do about it.

Attach the four wires for SDA, SCL, 5V (VCC), and GND between the MIDI Shield and the LED Matrix.  On the Mega, these four connections are particularly easy to make: SDA has a dedicated port (20), as does SCL (21), and there are exposed 5V and GND ports in the corners: none of these covered by the SparkFun MIDI Shield and are easily accessed.

You may need to change the I2C address that Gizmo uses to send messages to your backpack.  See a little ways below for information on that.  But first some discussion on backpack options:

\paragraph{Using the FeatherWing 8x16 Backpack} Instead of the AdaFruit 16x8 LED Matrix, some people have lately been using Adafruit's ``8x16 FeatherWing Backpack''\footnote{https:/\!/www.adafruit.com/product/3152}for their LED display.   This device rotates the LED matrices, so you'll need to account for that so Gizmo can rotate them back.  In the Gizmo file \texttt{LEDDisplay.h}, change the lines:

\begin{verbatim}
#define SCREEN_TYPE_ADAFRUIT_16x8_BACKPACK
// #define SCREEN_TYPE_ADAFRUIT_16x8_FEATHERWING_BACKPACK
// #define SCREEN_TYPE_TWO_ADAFRUIT_8x8_BACKPACKS
\end{verbatim}

to

\begin{verbatim}
// #define SCREEN_TYPE_ADAFRUIT_16x8_BACKPACK
#define SCREEN_TYPE_ADAFRUIT_16x8_FEATHERWING_BACKPACK
// #define SCREEN_TYPE_TWO_ADAFRUIT_8x8_BACKPACKS
\end{verbatim}

\paragraph{Using a Pair of 8x8 Backpacks} Perhaps for some reason you need to use two 8x8 backpacks.\footnote{For example, https:/\!/www.adafruit.com/product/1049} This is going to be slower than a 16x8 Matrix Backpack because Gizmo must send two I2C commands each screen refresh, which will worsen MIDI timing. If you want to do this, in the Gizmo file \texttt{LEDDisplay.h}, change the lines:


\begin{verbatim}
#define SCREEN_TYPE_ADAFRUIT_16x8_BACKPACK
// #define SCREEN_TYPE_ADAFRUIT_16x8_FEATHERWING_BACKPACK
// #define SCREEN_TYPE_TWO_ADAFRUIT_8x8_BACKPACKS
\end{verbatim}

to

\begin{verbatim}
// #define SCREEN_TYPE_ADAFRUIT_16x8_BACKPACK
// #define SCREEN_TYPE_ADAFRUIT_16x8_FEATHERWING_BACKPACK
#define SCREEN_TYPE_TWO_ADAFRUIT_8x8_BACKPACKS
\end{verbatim}

\paragraph{Stacking Screens Vertically} It's possible to add an additional 16x8 screen vertically, forming a 16x16 square.  At present the only applications which take advantage of this are the Drum Sequencer and the Step Sequencer, which will lay out their tracks using the 16x16 space.  Otherwise (for now) all other operations will use the bottom LED screen, leaving the other one blank.  

This is going to be slower than a single 16x8 Matrix Backpack because Gizmo must send two I2C commands each screen refresh, which will worsen MIDI timing. If you would like to try this out, in the Gizmo file \texttt{LEDDisplay.h}, change the lines:


\begin{verbatim}
// #define TWO_SCREENS_VERTICAL
\end{verbatim}

to

\begin{verbatim}
#define TWO_SCREENS_VERTICAL
\end{verbatim}

You can in fact use for 8x8 backpacks as a square, but I would not recommend it: too many calls to I2C.

\paragraph{Customizing the Screen I2C Address}  The four I2C addresses potentially used by the screen are found in the Gizmo file \texttt{LEDDisplay.h}, and are by default:

\begin{verbatim}
#define I2C_ADDRESS_1    			((uint8_t) 0x70)
#define I2C_ADDRESS_2     			((uint8_t) 0x72)
#define I2C_ADDRESS_3     			((uint8_t) 0x71)	
#define I2C_ADDRESS_4     			((uint8_t) 0x73)
\end{verbatim}

\begin{itemize}
\item A screen consisting of a single 16x8 backpack will use \texttt{I2C\char`_ADDRESS\char`_1}.
\item A screen consisting of two 8x8 backpacks will use \texttt{I2C\char`_ADDRESS\char`_1} for the right backpack and  \texttt{I2C\char`_ADDRESS\char`_2} for the left backpack.
\item A screen consisting of two 16x8 backpacks stacked vertically will use \texttt{I2C\char`_ADDRESS\char`_1} for the bottom backpack and  \texttt{I2C\char`_ADDRESS\char`_3} for the top backpack.
\item A screen consisting of four 8x8 backpacks arranged in a square will use \texttt{I2C\char`_ADDRESS\char`_1} for the bottom right,   \texttt{I2C\char`_ADDRESS\char`_2} for the bottom left, \texttt{I2C\char`_ADDRESS\char`_3} for the top right, and \texttt{I2C\char`_ADDRESS\char`_4} for the top left backpack, 
\end{itemize}

You may need to change these I2C addresses as necessary.

\paragraph{Rotating the Screen} If your screen is upside down (depending on how you installed it), then in \texttt{LEDDisplay.h}, change the line:

\begin{verbatim}
// #define ROTATE WHOLE SCREEN
\end{verbatim}

to

\begin{verbatim}
#define ROTATE WHOLE SCREEN
\end{verbatim}



\item {\bf Modify your Arduino Software}\quad It used to be that you had to perform surgery on your Arduino software to modify the Wire library.  This is no longer required: Gizmo now has its own copies of modified versions of the Wire library.

\ignore{
\item {\bf Modify your Arduino Software}\quad You will need a pretty recent version of the Arduino software, as newer versions have better compilers which compile more compact code.  I developed Gizmo on Arduino 1.6.12 for MacOS X.  You will want to modify the source code files for the Arduino libraries in two spots:

\paragraph{Add Nonblocking I2C Writes}  This will dramatically speed up I2C for our purposes.
\begin{enumerate}
\item Locate your \texttt{Wire.cpp} and \texttt{Wire.h} files.  On the Mac they're located in\\\texttt{Arduino.app/Contents/Java/hardware/arduino/avr/libraries/Wire/src/}\\(and probably one other location).

\item Add the following line inside the ``public'' method region in \texttt{Wire.h:}

\begin{verbatim}
    uint8_t endTransmissionNonblocking();
\end{verbatim}

\item Add the following method to \texttt{Wire.cpp}:

\begin{verbatim}
    uint8_t TwoWire::endTransmissionNonblocking()
        {
        uint8_t ret = twi_writeTo(txAddress, txBuffer, txBufferLength, 0, 1);
        return ret;
        }
\end{verbatim}
\end{enumerate}

Be warned that certain recent versions of the Arduino software may have these files located in {\it multiple} places.  On the Mac, for example, there are often two copies of each of these files.  You'll need to change all of them.

\paragraph{Reduce the I2C Buffer Size from 32 to 20} This gives us some extra static RAM space, and is probably only useful on the Arduino Uno.  You can probably skip this if you're on a Mega.

\begin{enumerate}
\item Locate your \texttt{Wire.h} file.  Again, on the Mac it's located in\\
\texttt{Arduino.app/Contents/Java/hardware/arduino/avr/libraries/Wire/src/}\\(and probably one other location).

\item Change the \texttt{BUFFER\_LENGTH} constant in the \texttt{Wire.h} file as follows:

\begin{verbatim}
        #define BUFFER_LENGTH 20   // Was 32
\end{verbatim}

\item Identify your \texttt{twi.h} file.  On the Mac it's located in\\
\texttt{Arduino.app/Contents/Java/hardware/arduino/avr/libraries/Wire/src/utility/}\\(and probably one other location).

\item Change the \texttt{TWIBUFFER\_LENGTH} constant in the \texttt{twi.h} file as follows:

\begin{verbatim}
        #define TWI_BUFFER_LENGTH 20    // Was 32
\end{verbatim}
\end{enumerate}

Again, be warned that certain recent versions of the Arduino software may have these files located in {\it multiple} places.  On the Mac, for example, there are often two copies of each of these files. You'll need to change all of them.
}


\item {\bf Download the Gizmo Software}\quad You can grab it with git or svn at\\
\texttt{https:/\!/github.com/eclab/gizmo/}\quad or download a \texttt{.zip} file at\\
\texttt{https:/\!/github.com/eclab/gizmo/archive/master.zip}

\item {\bf Install the FortySevenEffects MIDI Library}\quad But not straight off of its website.\footnote{https:/\!/github.com/FortySevenEffects/arduino\_midi\_library}  The MIDI library is in a state of flux and is having major additions being made.  The Gizmo code is extremely sensitive to code size.  Thus for now, use the version I include in Gizmo's Github repository.

\item {\bf Install the SoftReset Library}\quad A copy of it is in Gizmo's Github repository.

\item {\bf Switch the MIDI Shield to ``Prog''}

\item {\bf Build and Upload Gizmo's Code to the Arduino}\quad  It should compile cleanly for either an Arduino Uno or an Arduino Mega 2560.  On the Uno it'll complain of Low Memory.

\item {\bf Switch the MIDI Shield to ``Run''}\quad  You can plug in MIDI cables now.

\item {\bf Do a Full Reset of Gizmo}\quad Power up the Arduino.  Then hold down all three pushbuttons on the MIDI Shield.  While doing so, press and release the reset button on the MIDI Shield.  Continue to hold down the three push buttons until the two LEDs on the MIDI Shield light up.\footnote{Sometimes you have to do this a couple of times on the Mega to get the lights to light up.}  Let go of the pushbuttons.  This procedure initializes the memory, options, and files in the device.  See Section~\ref{initializinggizmo} for more details about the full reset and other reset options.  

\item {\bf Now you're ready to go!}

\end{enumerate}

\subsection{Customizing Gizmo}

Gizmo is modular: you can choose which application or applications you want to have on your device, as well as various features.  See the file \texttt{All.h} for customization options.  It's als0 where you can pick which application you want on your Uno.

\subsection{Buttons, Shield Quality, and Debouncing}
\label{debouncing}

The SparkFun MIDI Shield uses poor quality buttons.  You can push down on the button and wiggle it and it'll think it's been pressed multiple times.  As a result occasionally pressing a button may send two or more button-presses to Gizmo.

Gizmo has a debounce mechanism built-in: it ignores secondary apparent button-presses if they occurred within \(N\) milliseconds of an earlier button release.  At present \(N\) is set to 100ms.  If you need to press buttons faster than ten times a second (and it's possible) set the \texttt{BUTTON\_PRESSED\_COUNTDOWN\_DEBOUNCE} constant (in \texttt{TopLevel.h}) to something smaller.

After about four months of extremely heavy use, my select and back buttons began to fail: if I pressed either one the right way it wouldn't indicate that it's been pressed.  If you reach this state you may have to replace the buttons, though that's kind of hard to do: I just built a new board with different, better-quality low-profile buttons (just look on Adafruit for 12mm pushbutton switches).  Fingers crossed.

\subsection{A Bootloader Bug on the Arduino Mega}

Many versions of the Arduino Mega have an error in their bootloader which may cause the Arduino to hang if you've got MIDI device cables plugged into MIDI IN when you power Gizmo up.  Similarly, if you press the reset button, it'll look like it's not booting up, but just frozen.  The issue is simple: if on bootup the Mega hears data coming in on MIDI IN (which is its serial port), the bootloader assumes that you're downloading a new program and immediately goes into program-load mode.    But in fact it's probably just MIDI Clock or Active Sensing commands from a chatty synthesizer.  You have four options:

\begin{itemize}
\item Turn on or Reboot Gizmo without anything plugged into MIDI IN, then plug the cable in.
\item Turn off the synthesizer, turn on or Reboot Gizmo, then turn the synthesizer back on.
\item Turn on or Reboot Gizmo with its switch set to ``Prog'', then when it's running, switch it to ``Run''.  I suggest this one.
\item Permanently fix the issue by burning a new bootloader.  This is an involved process, and I don't suggest it.  See https:/\!/github.com/arduino/Arduino/issues/2101
\end{itemize}

\subsection{Earlier Versions of Gizmo}

Gizmo versions 1 through 6 attempted to run the full suite of applications, more or less, on both the Mega and the Uno: the Uno would run ``reduced'' versions of many of them.  This resulted in an awful mess of {\tt \#ifdef} statements and bizarre tweaks to squeeze out the last bit of memory from the Uno.

Gizmo 7 and on no longer do this.  Instead, all the applications run fine on the Mega: but only one application at a time will run on the Uno.  Furthermore, the Step Sequencer has a ``reduced'' version which runs on the Uno (the ``full'' version is too large), and the Drum Sequencer is simply too large to run on the Uno at all.

If you're on an Uno and want to have a large complement of applications, you might try Gizmo 6, which can be found in the {\tt old/} subdirectory.  Note that this version will no longer be maintained.  Otherwise, you are strongly encouraged to get a Mega instead.

\clearpage
\section{Initializing Gizmo}
\label{initializinggizmo}

Gizmo must be initialized before it can be used.  You can do a factory reset on Gizmo with this procedure:

\begin{enumerate}
\item Hold down all three buttons
\item Press the reset button
\item Continue to hold down all three buttons until both on-board LEDs have lit.
\item Let go of the buttons.
\end{enumerate}

Gizmo will reset all of its flash RAM and reboot.  This means that all options will be set to their default state, and all file storage (including arpeggios) will be emptied.

\paragraph{Doing a ``Partial Reset''} Whenever you load new Gizmo software, you should reinitialize Gizmo or strange things will happen because the order and makeup of options in RAM often changes.  However a full factory reset also erases your saved files (sequences, arpeggios, recordings).  If you want to preserve these but just reset the options, hold down the left and right buttons instead of all three (don't hold down the middle button).

\subsection{Starting Gizmo}

\begin{wrapfigure}{r}{2in}
\vspace{-1.5em}\includegraphics[width=2in]{Gizmo1.pdf}
\vspace{-2em}\caption{\small Boot Screen}\vspace{-2em}
\label{BootScreen}
\end{wrapfigure}

If you power up Gizmo, or press its reset button, it will display its {\bf boot screen}, which includes the version number (at right, the version number is ``1'').

Gizmo will then show the {\bf root menu}.  Use the left potentiometer to scroll to any of the following applications, and use the select button to select an application.  

\subsection{Starting Gizmo Headless}

Gizmo normally targets the SparkFun MIDI Shield, but you might be able to use its code with some other MIDI shield, as long as the shield just intercepts Serial to do its MIDI and doesn't interfere with I2C.  Such MIDI shields are unlikely to have the button, pots, or LEDs that the SparkFun shield has.  But that's okay, you can add those things yourself.  

Or you could just control the buttons and pots virtually via NRPN, and just do without the two LEDs.  To do this you'll need some way to clear Gizmo's options and files without needing to press any buttons (as you won't have any).  You can do it like this:

\begin{enumerate}
\item In the \texttt{All.h} file, change the line 
\begin{verbatim}
        //#define HEADLESS_RESET
\end{verbatim}
...to...
\begin{verbatim}
        #define HEADLESS_RESET
\end{verbatim}
\item Compile and download the code, and run it once.  Gizmo will show its opening screen, then display \texttt{RSET} and not do anything else.
\item In the \texttt{All.h} file, change the line back, that is, to 
\begin{verbatim}
        //#define HEADLESS_RESET
\end{verbatim}

\end{enumerate}

Gizmo's options have now been reset in a slightly different way than usual: specifically, the {\bf Control MIDI} option (see ``Control MIDI'' in Section \ref{options}) has been set to MIDI channel 16, rather than to OFF.

Now you need to modify Gizmo's code to run headless, so it doesn't bother lighting up the on-board LEDs or reading the buttons and pots.   You can do this as follows:

\begin{enumerate}
\item In the \texttt{All.h} file, change the line 
\begin{verbatim}
        //#define HEADLESS
\end{verbatim}
...to...
\begin{verbatim}
        #define HEADLESS
\end{verbatim}
\item Compile and download the code, and run it.
\item You should now be able to control Gizmo via NRPN (see ``Control MIDI'' in Section \ref{options}) on channel 16.
\item In the \texttt{All.h} file, you might wish to change the line back, that is, to 
\begin{verbatim}
        //#define HEADLESS
\end{verbatim}

\end{enumerate}

I make no guarantees that this will work: but it might.

\clearpage

\section{Using Gizmo}

\begin{wrapfigure}{r}{3in}
\vspace{-1.5em}\includegraphics[width=3in]{OverallLayout.pdf}
\vspace{-2.5em}\caption{\small High-Level Gizmo Layout}
\label{HighLevelGizmoLayout}
\vspace{-1em}
\end{wrapfigure}

Gizmo lets you select and use one or more applications such as arpeggiators, step sequencers, MIDI gauges, and so on.  To run these applications, Gizmo has a modest interface: a 16x8 LED display, two on-board LEDs (red and green), three buttons, and two potentiometers (dials): but it strives hard to make good use of these.  

Gizmo's primary interface layout is a menu hierarchy.  You select menu items, which take you deeper into the hierarchy, or you head go back up the hierarchy.  

When Gizmo starts, you're in the {\bf Top Level Menu}.  Here you may choose among different {\bf applications}.  Once you have entered an application, its {\bf application number} is displayed in the {\bf current application region} (see Figure~\ref{HighLevelGizmoLayout}). The current application LEDs light up right-to-left in a certain pattern with increasing application numbers, as shown in Figure~\ref{applicationvalues}.   

\begin{wrapfigure}{r}{1.6in}
\vspace{-1em}\includegraphics[width=1.6in]{Application.pdf}
\vspace{-2em}\caption{\small Application Values}
\vspace{-1em}
\label{applicationvalues}
\end{wrapfigure}

Also shown in Figure~\ref{HighLevelGizmoLayout} are the {\bf Beat/Bypass} light and the {\bf Note Pulse} light.  The Beat/Bypass turns on and off every {\it beat} (quarter note).  This shows the current {\bf tempo}.\footnote{A beat occurs every 24 {\it pulses} of a MIDI Clock or Gizmo's internal clock.}  The Note Pulse light turns on and off every {\it note pulse}: this is the speed you have chosen for the arpeggiator, step sequencer, etc.  Note pulses are defined in terms of note value: for example, if the note pulse is presently in sixteenth notes, then this light will turn on and off four times faster than the beat light, which is in quarter notes. The note pulse light is also affected by the degree of swing (syncopation) you have defined.

If Gizmo's clock is presently {\bf stopped}, either because you have stopped it internally or because Gizmo is being driven by an external MIDI clock which is currently stopped, then {\bf both the Beat/Bypass and the Note Pulse lights will be off} (though if Bypass mode is on, then Beat/Bypass will still blink rapidly: read on for more information about that).  See Section \ref{startingclock} for more information about starting and stopping the clock.

The rest of the screen is given over to the application.  In many cases an application will display text and other data in the {\bf Application Display} region, and light certain status LEDs in the {\bf Application Footer}.

You wend your way through menus, choose applications, and manipulate them using Gizmo's three buttons and two potentiometers as follows:

\begin{wrapfigure}{r}{1in}
\vspace{-1em}
\includegraphics[width=1in]{exit.pdf}
\caption{Exit?}
\label{exit}
\vspace{-1em}
\end{wrapfigure}

\paragraph{Buttons}  The left button is called the {\bf back button}.  The right button is called the {\bf select button}.  The center button is called (unsurprisingly) the {\bf middle button}.   The buttons respond to being {\bf pressed} and immediately released, and also respond to {\bf long presses}: holding a button down for a long time (presently one-half second) and then releasing it. 

Buttons do different things.  Here are certain common items:

\begin{itemize}
\item {\bf Pressing the Back Button}\quad This is the ``escape'' or ``cancel'' button: it {\bf always goes back up the menu hierarchy.}  Thus you can't ever get lost\,---\,just keep pressing the Back button and you'll eventually find yourself at the root menu.  Sometimes when you press the Back button, you'll be presented with an ``Exit?'' sign as shown in Figure~\ref{exit}, to double-check if you really want to back out this far (like leaving an application). If you do, just press the Back button again.  If not, press the Select button to undo.

\item {\bf Long-Pressing the Back Button}\quad This toggles {\bf bypass mode}, where Gizmo tries hard to act as if it's just a MIDI THRU coupler between its input and output MIDI wires: it passes through all the MIDI it receives,\footnote{Except Sysex, which is too much for the Arduino's tiny memory.} and with two exceptions (the Controller, see Section \ref{controller}, and the ``Merge'' MIDI clock option, see Section~\ref{options}) it doesn't output any new data.  In bypass mode, the {\bf Beat/Bypass} light (Figure~\ref{HighLevelGizmoLayout}) will stop beating and instead will flash rapidly.  Bypass mode also sends an ``all sounds off'' command to all outgoing MIDI channels, so if you need to shut up a misbehaving synth, you can enter bypass mode and then leave it again.

\item {\bf Pressing the Select Button}\quad If you're being presented with a menu of options, or a number to choose, or a glyph to choose, the Select button will select the currently-displayed option.  In some applications, the Select Button is assigned other tasks.

\item {\bf Pressing the Middle Button}\quad Likewise if you're being presented with a menu of options, or a number to choose, or a glyph to choose, the Middle button will increment the current display (but not select it).   The Middle button does lots of other things as needed.
\end{itemize}

\paragraph{Potentiometers}

Gizmo's potentiometers (or {\bf pots}) do a number of tasks, including scrolling through menu options, moving a cursor horizontally or vertically, or choosing a number.  The left pot does primary tasks, and the right pot assists when appropriate.  Note that these are potentiometers, not encoders: if you start turning one, you may experience a big jump initially.  The potentiometers have a limited range (0...1023), and a significant degree of noise, so realistically they can only choose numbers between 0...127 or so.  If asked to select a number from a {\it big} range, the {\bf left pot} will act as a coarse-level selector, and the {\bf right pot} will fine-tune the number.


\paragraph{On-Board LEDs}

There are two LEDs on-board the SparkFun MIDI Shield.  The {\bf red LED} is toggled on and off to reflect incoming MIDI data.  The {\bf green LED} is toggled on and off to reflect outgoing MIDI data.  

\subsection{How Gizmo Handles MIDI}

Gizmo listens for certain kinds of incoming MIDI data, then hands it to an {\bf application}, which may in turn send MIDI data out.  Applications can send data out any way they wish, but in many cases they restrict themselves to sending data out only to Gizmo's {\bf default MIDI Out Channel.}  Note data (Note on, Note Off) sent out may also be subject to further user-specified transformation in the form of transposition or volume control.  

\vspace{-0.5em}\paragraph{Channel Data} If the incoming data is {\it channel data} (Note on/off, Aftertouch, Pitch Bend, Program Change, Control Change / NRPN / RPN), most applications will listen to it only if it is from the {\bf default MIDI In Channel}, which can be 1--16, or OMNI (all channels), or OFF (no channels).  Some applications may listen to channel data from certain additional channels: for example the MIDI Thru Facility also listens to data on an additional ``Merge channel''.  {\bf Data from all remaining channels is simply passed through to MIDI Out. }   And when you are in the Top Level Menu, there is one other option which lets you route from MIDI In to a special ``MIDI Routing Channel''.  See Section~\ref{secretsofthetoplevelmenu} for more information on that.

\vspace{-0.5em}\paragraph{Clock Data} Gizmo can follow clock data or ignore it and generate its own clock.  Furthermore Gizmo can pass incoming clock data through to MIDI Out, or block it.  Gizmo can also emit its own clock data.

\vspace{-0.5em}\paragraph{Sysex Data} Sysex is not passed through: it requires too large a buffer in the underlying MIDI library.

\vspace{-0.5em}\paragraph{Other Data} Most other data is simply passed through from MIDI In to MIDI Out.

\vspace{-0.5em}\paragraph{Bypass Mode} When bypass mode is engaged, all data received is simply passed to MIDI Out.  Applications may still listen to the data.  One application (the Controller) may send data during bypass mode.

\vspace{-0.5em}\paragraph{Controlling Gizmo} One incoming channel can be defined as the {\bf default MIDI Control Channel}.  Data on this channel is used only to manipulate Gizmo (via NRPN or CC), for example, changing a Gizmo parameter.


\subsection{Secrets of the Top Level Menu}
\label{secretsofthetoplevelmenu}

When you start up Gizmo, you begin at the {\bf Top Level Menu}, where you can choose from various applications. You scroll through the applications\footnote{If you're using an Uno, your only applications will be the application you compiled, and Options.  If you're using the Mega, you'll have a number of applications.} by turning the {\bf left knob} and you select an application by pressing the {\bf select button} (the right button).  If you leave an application by repeatedly pressing the {\bf back button} (the left button), you will eventually find yourself back up at the Top Level Menu.

The Top Level Menu has three secret tricks up its sleeve that you should know about.  

\begin{itemize}
\item By default Gizmo starts up at the Top Level Menu with {\bf Bypass turned on} in a special way: when you select your very first application, Gizmo will automatically turn Bypass off (without sending an ``all sounds off''), enter that application, and thereafter treat Bypass normally.  This only happens on selecting your first application, and that's it.  Also if you toggle Bypass manually, Gizmo will automatically start treating Bypass normally.  The point of this is to allow Gizmo to be in Bypass mode unintrusively as soon as it's powered up so you can have it attached to your keyboard all the time even if you don't use it that often.\footnote{If you would prefer that Gizmo start up with Bypass turned {\it off}, in \texttt{All.h} comment out the line\quad\texttt{\#define TOPLEVEL{\textunderscore}BYPASS}\quad by changing it to\quad\texttt{// \#define TOPLEVEL{\textunderscore}BYPASS}\quad and rebuild Gizmo.}

\item If you turn the {\bf right knob} a large distance from its current position,\footnote{It's a large distance to make it less likely to happen by accident.}  this will trigger Gizmo to let you select the {\bf  MIDI Routing Channel}, either 0 (meaning no routing) or a channel from 1--16.  If you have chosen a channel, and you have turned off Bypass, then while in the Top Level Menu (and {\it only} while in the Top Level Menu), if you play MIDI into the Default MIDI In Channel, it will get sent out the MIDI Routing Channel.  This feature is because I usually have Gizmo attached to a MIDI controller keyboard (a StudioLogic SL 990) which can only output on Channel 1.  This gives me an easy way to quickly reroute it on other channels.    This feature is only available on the Mega, not the Uno.

\item If you {\bf long-press the Middle Button}, you can toggle starting or stopping the MIDI clock; and if you {\bf long-press the Select Button} (the right button), you can toggle {\it continuing} or stopping the MIDI clock.  This also occurs in the Options menu.  How or whether this all works depends on Gizmo's clock settings: see Section~\ref{startingclock} for more details.
\end{itemize}


\clearpage
\section {The Arpeggiator}

\begin{wrapfigure}{r}{3in}
\vspace{-1.5em}\includegraphics[width=3in]{arpeggio.pdf}
\vspace{-2em}\caption{\small The Arpeggiator Display}\vspace{-1em}
\label{arpeggiator}
\end{wrapfigure}

The arpeggiator allows you to play different kinds of arpeggios, and also to create and save up to ten of them.  The arpeggiator's main menu lets you choose to play five different kinds of pre-set arpeggios, or to select an arpeggio from user-created arpeggio slots 1--10, or finally to create an arpeggio.

An arpeggio is a pattern for repeatedly playing the notes in a chord.  When you hold down a chord on the keyboard, the arpeggiator uses this pattern to play the notes, typically one at a time, according to the pattern selected.  Patterns do not specify exact notes, but rather note orderings.  For example, an arpeggio might be defined as 1, 2, 2, 3, 4, 2, 3, where if 1 is the root (lowest) note of a chord, 2 is the second lowest note, and so on.  If, say, you played a C7 chord (C E G B$\flat$) with this arpeggiator pattern, then the arpeggiator would assign 1 = C, 2 = E, 3=G, 4=B$\flat$, and so repeat the sequence \hbox{C E E G B$\flat$ E G.}

\paragraph{Playing an Arpeggio}

When you start the arpeggiator, you are presented with a menu of six preset arpeggios, plus ten user-created arpeggios, and finally an option to create new arpeggios.  You enter the arpeggiator's play mode by choosing any but the last option (if you're trying things out, the first option would be a good choice).

When in the play mode, the arpeggiator will play an arpeggio for a chord as long as you hold down the keys in the chord.  Alternatively if you toggle the {\bf latch}, then the playing will continue after you have released the keys, and will only shift to a new chord when you release all notes and then hold down an entirely new chord.  Latch mode is toggled by pressing the Middle button.  When Latch mode is engaged, an LED will light as shown in Figure~\ref{arpeggiator}.  Latch mode is stored permanently in memory, so next time you start up the Arpeggiator, your option will be remembered.  Note that you can stop the current notes from playing by toggling Latch mode again.

The arpeggiator also has a {\bf performance mode} which you toggle by Long-Pressing the Middle and Select buttons simultaneously: the latch is also set.  In this mode, if you press notes on the keyboard, they are not sent to the arpeggiator, but instead are used for one of two things:
	\begin{itemize}
	\item {\bf Transposing the arpeggio.}  Here, as soon as you enter performance mode, you will be asked to enter the key of the arpeggio chord (as note).  Thereafter, if you press other keys, the arpeggio will be transposed from that key to the other keys (including shifting it by multiple octaves if your keys are far away).  This really only makes sense if you've latched an arpeggio first.  The Performance Mode LED will blink.
	\item {\bf Playing along.}  As you play notes, they will be routed out a MIDI channel of your choosing while the arpeggio continues as before.  If you didn't latch an arpeggio first, this will act as a lightweight bypass mode of sorts. The Performance Mode LED will be solid.
	\end{itemize}
You can chose which of these items occurs when you enter performance mode choosing the menu option {\bf Performance}.

If you Long-Press the Middle button, you will toggle {\bf octave accompaniment} when you release.  As a result, every time Gizmo plays a note it will also play the same note one octave above.  That's all.

In some situations, you may need to manually {\bf start or stop} the clock during arpeggiation.  You can do this with an option in the menu (long-press the Select button).

\begin{wrapfigure}{r}{2.1in}
\begin{center}
\vspace{-1em}
\includegraphics[width=2.0in]{updown}
\vspace{-1em}
\caption{\small Up, Down, Up-Down, and +Up-Down Symbols}
\vspace{-3em}
\end{center}
\end{wrapfigure}

\paragraph{Preset Arpeggios}

There are seven preset arpeggios: {\it up}, {\it down}, {\it up-down}, {\it  +up-down}, {\it random}, {\it assign}, {\it chord repeat}, and {\it chord hold}.  They work as follows:

\begin{itemize}
\item {\bf Up}.  Each note in your chord is played in turn, lowest note first.  When all notes are exhausted, the pattern is repeated.
\end{itemize}

\begin{itemize}
\item {\bf Down}.  Each note in your chord is played in turn, highest note first.  When all notes are exhausted, the pattern is repeated.
\item {\bf Up-Down}.  Each note in your chord is played in turn, lowest note first, up to but not including the highest note.  When all notes are exhausted, all notes in the chord are played in turn, highest note first, down to but not including the lowest note.  When all notes are again exhausted, this pattern repeats.
\item {\bf  +Up-Down}.  This is the same as Up-Down, except that the bottom note is added again to the very top.  This is particularly useful for arpeggiating over multiple octaves.  For example, if you played C G, two octaves of ordinary Up-Down would be C1 G1 C2 G2 C2 G1, but two octaves of +Up-Down would be C1 G1 C2 G2 \textit{\textbf{C3 G2}} C2 G1, adding two more notes.  
\item {\bf Random}.  Chord notes are played at random.  Gizmo tries to not play the same note twice in a row, unless there's only one or two notes in the chord.
\item {\bf Assign}. Each note in your chord is played in turn, in the order in which you had pressed the keys when you formed the chord.  When all notes are exhausted, the pattern is repeated.
\item {\bf Chord Repeat}.  The entire chord is repeatedly played in its entirety.
\item {\bf Chord Hold}.  This just plays the chord once and holds it.  The idea is that with Latch mode on, Chord Hold is an automatic sustain: it plays the chord until you release all keys, then begin a new chord.
  
\end{itemize}

\begin{wrapfigure}{r}{1.45in}
\begin{center}
\vspace{-2em}
\includegraphics[width=1.45in]{arpvelocity}\\
\caption{Displayed arpeggio note velcocity.}
\end{center}
\vspace{-2em}
\label{arpvelocity}
\end{wrapfigure}

As you play a preset arpeggio, the notes played by the arpeggio will appear at right.  The symbol representing the preset will appear at left.

While playing a preset arpeggio, you can select the number of octaves by calling up the play menu (Long-Pressing the Select button), then choosing {\bf Octaves}.  If you have an octaves value \(\geq 0\), then when the arpeggiator exhausts its notes, it will repeat them again one octave higher, then another octave higher, and so on, up to the number of octaves chosen. You an select any value from 0--6 octaves.\footnote{Note that this means that if your chord is larger than one octave, you may not get what you expected.  For example, if you are playing UP with C5 E5 G5 D6, and with octaves\(=\)1, then the arpeggiator will play C5 E5 G5 D6 C6 E6 G6 D7.  Note that C6 is played after D6 is played.  That's the current behavior right now anyway.}

By default the note velocity\footnote{By velocity we mean in the MIDI sense: how fast the key is struck; this basically translates to loudness.} that {\it all} notes in the track will use (as a value 0...127) is determined by the {\bf first note} you press when you hold down a chord.  This strategy usually works well, but if you can instead fix the velocity to a specific value by (once again) calling up the menu by Long-Pressing the Select button, then choosing {\bf Velocity}.  At the far highest end of the velocity options is \texttt{FREE}, which means that the velocity is determined by your fingers again (this is the default). The current note velocity used by the arpeggiator is roughly displayed below the note pitch, using the range as shown in Figure~\ref{arpvelocity} at right.
%
\begin{wrapfigure}{r}{1.6in}
\begin{center}
\vspace{-1em}
\includegraphics[width=1in]{arpeggioexample.pdf}\\
Example Arpeggio\\[2em]
%\vspace{-1.5em}\includegraphics[width=1in]{arpeggioup.pdf}\\
%Ascending Notes\\in Sparse Mode\\[2em]
%\vspace{-1.5em}\includegraphics[width=1in]{arpeggiodoubleup.pdf}\\
%Ascending Notes\\in Dense Mode\\
\end{center}
%\vspace{-1em}\caption{\small Example Arpeggio Editing Displays}
\vspace{-2em}
\label{arpeggioediting}
\end{wrapfigure}
%
\paragraph{Creating an Arpeggio}

The last menu option in the top-level Arpeggiator menu is {\bf Create}.  If you choose this, you can create an Arpeggio.  To do this you enter notes one by one.  These won't be the {\it actual} notes that are played when you hold down a chord.  Rather, after you save the result, the Arpeggiator will compact these notes into a sequence of consecutive numbers.  For example, if you play C3 E3 G3 E3 G2, then the Arpeggiator will compact these into 2 3 4 3 1.  Then when you play this as an arpeggio, the Arpeggiator will assign each of these numbers, in order, to the notes in your chord. 

The first question you will be asked is \texttt{NOTE}:\ the Arpeggiator is asking you to specify the note in your arpeggio pattern which will correspond to the root (lowest) note in a played chord.  This allows you to make a pattern which plays notes both above and below the played chord.  Consider the above pattern (C3 E3 G3 E3 G2, thus 2 3 4 3 1).  If you assigned C3 (thus 2) to be the root, then the Arpeggiator will play its final note (1) {\it below} the lowest not in your played chord.

After you have entered the root, the Arpeggiator will wait for you to start playing notes.  You can enter a sequence of up to 32 notes, including 14 unique notes.  If you enter too many unique notes, or more than 32 notes in total, the Arpeggiator will simply refuse to accept the next note.  You can enter rests by pressing the Select button.  You can also enter ties (stretching the last note) by pressing the Middle button.  The first note may not be a tie: what would it be tying?

As you enter notes, they are displayed in the left half of the LED, and the latest note pitch is displayed in the right half. Figure~\ref{arpeggioediting} shows different arpeggio patterns.  If you have eight or fewer unique notes in your pattern, then they will be displayed in {\it sparse mode} as shown at right.  If you have over eight notes, they will be displayed in {\it dense mode} in order to show all of the notes.  The LEDs don't show the notes per se, but rather the consecutive numbers that the notes represent.  For example, the ascending notes shown at right could be C, D, E, F, G, A, B, C, or they could be C, C\(\sharp\), D, D\(\sharp\), E, F, F\(\sharp\), B.

\begin{wrapfigure}{r}{3.5in}
\begin{center}
\vspace{-2.5em}
\includegraphics[width=3.5in]{arpeggionotes.pdf}\\
\end{center}
\vspace{-1em}\caption{Displayed arpeggio notes.  If you have entered more than 7 unique notes, the arpeggio will switch to ``dense mode'' to display them.}
\vspace{-1em}
\label{arpeggionotes}
\end{wrapfigure}

If you don't like the latest portion of your arpeggio, you can scroll the cursor back to an earlier section by turning the right knob.  Then if you start playing, the later notes will be discarded and the arpeggio pattern will continue from where the cursor was.  You can also scroll to view your past notes, then scroll all the way back to continue.

An {\bf important note regarding scrolling.}  The cursor is always located at the position where new notes will be entered.  But as you scroll back and forth, you'll hear the notes of your arpeggio.  When you scroll and hear a note, that note is the note {\it immediately before} the cursor position.  It's the last note in the arpeggio before you start adding new notes. It is {\it not} the note that you'll replace when you start entering notes. 

Anyway, when you are satisfied with your arpeggio, Long-Press the Select button to save.\footnote{How do you load an arpeggio to be re-edited, you ask?  The answer is, unfortunately, that you can't.  Once saved, arpeggios are compressed into consecutive numbers, stripping out the original notes you used to edit them.}  You'll be asked for a slot to save in, and after that, you'll be back in the main Arpeggiator menu.

\paragraph{Playing a User-Created Arpeggio}

To play a user-created arpeggio, from the main Arpeggiator menu, choose one of the numbers 0--9, each of which represents the arpeggiator slot you had saved your arpeggio in (these options appear after the preset options).  Then hold down a chord!  As you play a user-created arpeggio, the notes played by the arpeggio will appear at right.  The arpeggio pattern will scroll by at left.


The preset arpeggio patterns (such as Up-Down) vary with the number of notes in your chord: if you hold down five notes for example, and octaves=0, then five notes will be played in the arpeggio pattern.  This isn't the case for user-created arpeggios, which have a fixed user-specified number of notes.

What happens if your arpeggio is designed for three notes but you hold down  five notes in your chord?  Only three of your five notes will be played.

\bump

On the other hand, what happens if your arpeggio is designed for five notes but you hold down only three notes in your chord?  Then notes from the bottom (or top, as appropriate) of your chord will be transposed up/down one octave to make up the remaining notes.

Also, because they are fixed in size, user-created arpeggios do not respond to the octaves option.

If you're playing a user-created arpeggio, and you press the Select button, Gizmo will schedule an {\bf advance to the next non-empty user-created arpeggio} after the current arpeggio has finished.\footnote{Which brings up the question: how do you delete a user-created arpeggio?  Maybe you have some junk you don't want to advance to.  The answer is: go to CREATE an arpeggio, then while it's still empty, save it to the slot you want to delete.}

\paragraph{More Arpeggiator Options}

In addition to Octaves, the arpeggiator menu (Long-Press the select button) also includes the Options Menu as a submenu.\footnote{Recall that the Options Menu is for options shared among several applications: that's why it's a separate menu.  That and code space issues.}  See Section \ref{options} for detailed information on each of the options.  The Arpeggiator responds to the following options:

\vspace{0.75em}
\begin{tabular}{lllll}
Tempo& Note Speed& Swing & Note Length&Transpose\\
In MIDI& Out MIDI&Control MIDI&MIDI Clock&Volume\\
\end{tabular}

\vspace{0.75em}
Of particular interest to you may be Length and Tempo.  And Swing!

One note regarding the {\bf Note Length} option.  If the Note Length is set to 100 (and {\it only} 100), the Arpeggiator behaves specially: each note will extend until the next note (or rest) is played, regardless of how long that is.  Thus if you have Swing turned on, notes will extend until the next syncopated (delayed) note even if their time is up. This provides a fully legato feel.


\paragraph{The Left and Right Knobs}  When playing an arpeggio, turning the left knob will transport you directly to the Tempo menu option, so you can easily change the note tempo.  Similarly, turning the right knob will send you to the Note Length menu option.  You'll have to turn a lot to get this to happen (as a safety precaution).\footnote{{\it What about Arpeggio Key Velocity?  Swing?  Note Speed? Velocity or Transpose?}\qquad Arpeggio Key Velocity's easy: it's the first menu item.  For the others you'll have to go through the options menu, or remote-control the arpeggiator via CC.}  Once you're done changing the tempo or note length, press the Select button to save it, or the Back button to cancel and return to your previous value.

\paragraph{Controls Summary}  Here's what the arpeggiator's controls do in different contexts.


\begin{center}
\vspace{1em}
\noindent {\small
\begin{tabular}{@{}rll@{}}
\it Control					& \it When Playing an Arpeggio & \it When Creating a User Arpeggio\\
\hline\\[-0.5em]
Back Button				& Exit				& Exit\\
						& {\it or} Leave Performance Mode\\
Back Button (Long)			& Toggle Bypass  & Toggle Bypass\\
Middle Button				& Toggle Latch				& Tie \\
Middle Button (Long)			& Toggle Octave Accompaniment\\
Select Button				& Advance to Next User Arpeggio & Rest \\
						& {\it (only if currently in a user arpeggio)}\\
Select Button (Long)			& Menu & Save Arpeggio\\
Select + Middle Button (Long) 	& Set Latch and Enter Performance Mode \\
Left Knob					& Change Tempo  \\
						& {\it (press Select to set, Back to cancel)}\\
Right Knob				& Change Note Length & Scroll\\
						& {\it (press Select to set, Back to cancel)}\\
Keystrokes, CC, etc.			& Enter Arpeggio Chord & Enter Arpeggio Notes\\
						& {\it or} Transpose {\it in Performance Mode}\\
						& {\it or} Play Along {\it in Performance Mode}\\
\end{tabular}
}
\end{center}
~		% Give me a bit more space at the end of the page

\paragraph{Remote Control over CC}  If you have a remote control surface, you can directly access a number of the Arpeggiator's play or arpeggio creation capabilities by sending CC messages over Gizmo's Control MIDI Channel.    These operations are on top of the CC Control Midi operations (Section \ref{options}, see ``Control MIDI''), and in fact replicate a few of them.

\vspace{1em}
\noindent {\small
\begin{center}\begin{tabular}{@{}rlll@{}}
\it CC Number & \it CC Value & \it Creating a User Arpeggio	& \it Playing an Arpeggio \\
\hline\\[-0.5em]
88	&	Any	 & Enter Rest	& Toggle Latch\\
89	&	Any	 & Enter Tie       & \\%  Toggle Start/Stop Clock\\
90	&	Any	 & 	 	& Clear Latch\\
118	&	Any	&	& Set Latch and Enter Performance Mode\\
91	&	Any	&	& Advance to Next User Arpeggio\\
18	&	0--127	&	& Set Velocity (press Select to set, Back to cancel)\\
21	&	0--127	&	& Set Gizmo's Tempo (press Select to set, Back to cancel)\\
22	&	0--127	&	& Set Gizmo's Transpose (press Select to set, Back to cancel)\\
23	&	0--127	&	& Set Gizmo's Volume (press Select to set, Back to cancel)\\
24	&	0--127	&	& Set Gizmo's Note Speed (press Select to set, Back to cancel)\\
25	&	0--127	&	& Set Gizmo's Play Length (press Select to set, Back to cancel)\\
26	&	0--127	&	& Set Gizmo's Swing (press Select to set, Back to cancel)\\
27	&	0--127	&	& Set Performance Mode Type (press Select to set, Back to cancel)\\
30	&	0--127	&	& Set Octaves (press Select to set, Back to cancel)\\
\end{tabular}
\end{center}
}

\clearpage


\section {The Step Sequencer}
\label{stepsequencersec}

\noindent {\bf Note: }{\it Some features aren't available on the Uno due to lack\\
of memory.  These are called the``Advanced Step Sequencer''.}\vspace{1em}

\begin{wrapfigure}{r}{3in}
\vspace{-4.75em}\includegraphics[width=3in]{stepsequencer}
\vspace{-2em}\caption{\small Step Sequencer Display}
\vspace{-1em}
\label{stepsequencer}
\end{wrapfigure}

The Step Sequencer makes it easy to create multitrack loops of notes and other MIDI data.  You can have twelve 16-note tracks, eight 24-note tracks, six 32-note tracks, four 48-note tracks, three 64-note tracks, or two 96-note tracks.  You can later reduce the length of these tracks to arbitrary values.

\begin{wrapfigure}{r}{1in}
\vspace{-1em}\includegraphics[width=1in]{none.pdf}
\vspace{-1em}\caption{\small None}\vspace{-2em}
\label{none}
\end{wrapfigure}

\paragraph{Initialization} When you choose the Step Sequencer from the root menu, you will be asked to load a sequence from a slot.  You can do this, or you can create a new empty sequence by choosing \texttt{-~-~-~-} (meaning {\it none}).  Slots are shown by number, plus an \texttt{L} (for Load) and an optional \texttt{R} (for Recorder), \texttt{S} (for Step Sequencer), or \texttt{D} (for Drum Sequencer) if the slot already has a file from one of these applications.  This letter {\bf blinks} if the file isn't a Step Sequencer file to warn you that you might eventually overwrite a slot when you save.

If you have chosen {\it none}, or have chosen an empty slot or one presently filled by another application, you'll be asked to initialize the sequence first, choosing from 16, 24, 32, 48, 64, or 96 notes per track. You can also choose similar note lengths in a {\bf Monophonic} or {\bf Duophonic Synthesizer Format} ({\it Advanced Step Sequencer} only).  We discuss Mono/Duo options later. Then you are taken to the main Step Sequencer display.  

\begin{wrapfigure}{r}{1.8in}
\includegraphics[width=1.8in]{track.pdf}
\vspace{-2em}\caption{\small Track Numbers}
\label{tracknumber}
\end{wrapfigure}

\paragraph{Organization} Figure~\ref{stepsequencer} shows the general step sequencer display structure.  The {\it Tracks} region is where sequencer tracks are displayed and edited.  The {\it Track Number} area tells you the number of the currently-edited track.  Figure~\ref{tracknumber} explains how to interpret this.  The {\it Stopped} LED tells you if the step sequencer is currently stopped (or playing).  The {\it Track Uses a Default Velocity} LED tell you if all notes in the current track are fixed to the same velocity, or permitted to use different velocities.  The {\it Track Uses a Fader} LED is lit when the pre-track fader (volume) is set to something other than maximum volume for the current track.   Finally, the {\it MIDI Out Channel for Track} tells you what channel the current track is playing.  The patterns for the MIDI Out Channel for Track are the same as those in Figure~\ref{midichannelvalues}, except that MIDI Out does not support ALL (OMNI).  Other LEDs are discussed later.

Pressing the Select button will toggle whether the Step Sequencer is playing or stopped.  When you start playing the Step Sequencer, it waits for the clock to reach a measure boundary before beginning.  Note that if you start playing the Step Sequencer, but the clock isn't running, it won't play at all.  You'll need a clock (internal or external) to drive it.

You'll enter notes (or other kinds of sequenced data) in one of several {\it tracks}. You select your current track by turning the left knob.  In a given track, you choose the current note position by turning the right knob.  When the Step Sequencer is playing, the  tracks region will also display the {\it play position cursor}, a blinking vertical set of LEDs which indicate which notes in the tracks are currently playing.  This vertical group heads to the right, eventually wrapping around. 

Each track takes up one or two rows in the Tracks area: 16-note tracks take a single row, 24-note and 32-note tracks take two rows, 48-note tracks take three rows, and 64-note tracks take four rows.    Depending on the length of each track, you could have  up to twelve tracks: but the tracks display region is only six LEDs high.  This of course means that  the Tracks region scrolls as you turn the left knob.\footnote{This also means that can only see one 64-note track at a time.  BTW, you might be interested in knowing why Gizmo can't presently do 8- or 12-note tracks.  This is because their memory usage is too high\,---\,we're storing a lot of per-track information, and with 8-note tracks you can have 24 of them!  There's a different reason we don't have 128-note tracks: one track would take up the entire screen, including any footer information.  Also, you'd only have a single track, which seems to me to be not particularly useful.}

\vspace{10in}

\begin{wrapfigure}{r}{1.1in}
\hspace{\fill}\includegraphics[width=0.7in]{Patterns}\hspace{\fill}%
\vspace{-1em}
\caption{\small Four\ \ \ \ \ Bar Pattern.}\vspace{-1em}
\label{bars}
\end{wrapfigure}

The number of notes of the step sequencer (16, 24, 32, 48, 64, 96) is its {\it bar} or {\bf measure length}.  At the beginning of a new bar the step sequencer decides on which tracks to play.  You can set each track to play with a certain probability; or have the step sequencer choose from among certain tracks; or have tracks turn on and off according to a four-bar {\bf pattern}.  The current bar being played in the pattern is given as shown in Figure \ref{bars}.  Tracks can also be {\bf muted} or {\bf soloed}.

\paragraph{Modes} The Step Sequencer can be in one of three modes: the {\bf Edit Mode}, the {\bf Play Position Mode}, and the {\bf Performance Mode}.  In the Edit mode you can enter notes, rests, or ties one by one directly in positions of your choosing.  In Play Position Mode, when you play notes, they are entered right where the play position cursor is located.  This is good for doing on-the-fly recording.\footnote{I myself spend most of my time in Play Position Mode}  Finally in Performance Mode, when you play notes, they're not entered into the Step Sequencer: instead, they're routed to MIDI Out so you can play along with your recorded sequence; or they're interpreted as transposing the sequence as it plays.  Performance Mode also allows you to chain sequences and do other nifty things.

\paragraph{Edit Mode}
Typically you'll start in Edit Mode.  Somewhere on the screen there will be a single-pixel blinking {\it edit cursor} which determines where notes are entered if you play them.  The location of this cursor (track, note position) is determined by the dials.  

You enter notes by moving the edit cursor to the desired location, then playing keys.  You can also press the Middle Button to enter a rest, or Long Press the Middle Button to enter a {\it tie}.  A tie just means ``keep playing the previous note'': it essentially lengthens the previous note.  You can enter multiple ties in series to make a note as long as you like.

\paragraph{Sequencing Data {\it (Advanced Step Sequencer Only)}} The Step Sequencer allows you to store other kinds of data besides Note-On and Note-Off.  Specifically, you can set a given track to store and replay any one of (raw) CC, 14 Bit CC, NRPN, RPN, Pitch Bend, Channel Aftertouch, or Program Change commands by choosing {\bf Type} from the menu (see {\bf Menu Options} below).  If you select (raw) CC, 14 Bit CC, NRPN, or RPN, you will be further prompted to specify which parameter number you'd like to sequence (for example, CC\# 42 or NRPN\# 12312).

Non-note data is entered differently than note data.  To enter non-note events, send an event to the MIDI In Channel, then press the {\bf middle button}: the most recent sent event will then be recorded at the current position.  You can erase data at the current position (essentially send a rest) by long-pressing the middle button.\footnote{This may be confusing as rests are entered by simply {\it pressing} the middle button for note data, but I figured entering data was more important than entering rests in the non-note case.  Sorry.}

\begin{wrapfigure}{r}{1.25in}
\vspace{-1.5em}\hspace{\fill}\includegraphics[width=0.7in]{playpositionmode}\hspace{\fill}%
\vspace{-1em}
\caption{\small The Play Position Mode Cursor.  The two dots mark the current track.}
\vspace{-1em}
\label{playpositionmode}
\end{wrapfigure}

Some notes.  First, if you sent a Polyphonic Aftertouch event (on any key) it will be recorded instead as Channel Aftertouch.  Second, if your track is configured for (raw) CC, then you can send {\it any} raw CC value to it (all 127 parameter numbers), and it will also emit the same; but if your track is configured for 14-bit CC, then only the first 31 parameter numbers may be used.  Third, any Pitch Bend value can be stored except for 8191 (the high value).  Fourth, any 14 Bit CC, RPN, or NRPN value can be stored except for 16383 (again, the high value).  

\paragraph{Play Position Mode}

If you turn the right knob fully to the left, you will find that the cursor scrolls right off the left side of the screen.  At this point you are in {\bf play position mode}.  Here, if you press a key, it is entered at the play position cursor.  As long as you hold down the note, and don't press another, the sequencer will continue to add ties to stretch out the note.

In play position mode, the function of the Middle Button changes.  Now if you press the Middle Button, it will toggle muting the current track.  You can tell a track is muted because the LEDs at each end of the track are blinking.  Additionally, if you Long Press the Middle Button, it will erase all the notes on the current track (but not change its settings).

If the current track is muted (and you're not in Performance Mode as discussed below), Gizmo does something special: it plays any notes you send it on the track's channel: but it doesn't enter them into the track. This allows you to treat mute as a way to try out sounds on-the-fly without entering them into the sequencer.

You can tell you're in play position mode because the play position cursor has changed to include two dots which indicate the current track, as shown in Figure~\ref{playpositionmode}.  When stopped, the play position cursor will sit at the far right of the screen (so you'll only see one dot).

You can get out of play position mode just by turning the right knob so that the edit cursor is once again on-screen.

\paragraph{Performance Mode} If you simultaneously Long-Press the Select and Middle buttons when in either Edit or Play Position Mode, you'll enter Performance Mode.  This mode is meant to support performing with completed sequences.  Here, if you play notes or CC values different things might happen, depending on how you've configured things: either the notes will be entered on the track (which will be cleared first if a bar just transpired or the user just switched tracks), or the notes will be routed to some channel in MIDI Out, or the notes will be used to {\it transpose} the sequence as it's playing.  A track is transposed only if you have turned its ``Transpose'' option on (see Menu Options below).

In Performance Mode, a number of controls change their function  Pressing the Back Button doesn't exit the Step Sequencer: but rather, it exits Performance Mode back into whatever previous Mode you had used.  Pressing the Middle Button toggles muting of  the track.  Long-Pressing the Middle Button toggles Soloing: only your current track is played.  You can tell you're in Solo mode because every track {\it except} for the soloed one appears to be muted (LEDs at each end of the tracks are blinking).  The Left Knob changes tracks as usual, but the Right Knob changes the {\bf tempo}.  Long-pressing the Select button brings up the menu.  Pressing the Select button starts and stops playing as usual, but with a catch.  When you press the Select button to stop the sequence, in Performance mode it will instead schedule the sequence to be stopped at the end of the latest iteration.  If you press the Select button {\it again} while it's scheduled to stop, it will stop immediately.

In Performance Mode, sequences can be {\bf chained together}.  You can specify that a given sequence will {\bf repeat} for some number of bars, and then will either stop or will move on to {\it another sequence}.  Since the Mega can hold up to nine sequences, you can use repeating and chaining to make some fairly long songs.  Per-track patterns, repeating, and specifying the next sequence are done in the Menu Options (see later).

You can optionally chain two sequences together but set the first sequence to repeat {\it forever} in a {\tt LOOP}: in this case, you can still get to the next sequence (at the end of the bar) manually by long-press the Select and Middle buttons at the same time.

In Performance Mode, as in Play Position Mode, you'll always see the Play Position cursor.  When in Play Position Mode, a particular LED will also constantly blink to indicate the mode.  The current bar is also shown with a two-note pattern as shown at right.  Finally, if you have triggered the Sequencer to switch to a new sequence, then an LED is displayed to indicate this as well. See Figure \ref{stepsequencer}.

The behavior of {\bf Mute} and {\bf Solo} change in Performance Mode. The purpose of these behaviors is to make mute and solo more useful for syncing up with you in performance.

If you press solo, you won't {\it toggle} the solo.  Instead, you {\it schedule} the solo to be toggled at the beginning of the next bar.  If you press a second time, you undo your previous action.  So in short, pressing buttons go through this loop, which is known as {\bf advancing solo}:

\begin{itemize}
\item If you are currently {\it not} soloed: {\bf Schedule Solo} \(\rightarrow\) {\bf Undo}
\item If you are currently soloed: {\bf Schedule Un-solo} \(\rightarrow\) {\bf Undo}
\end{itemize}

\bump 

Similarly, if you press mute, you {\it schedule} the track to be muted (or not), at the next bar, for exactly one bar, and then to return to its present state.  If you press mute a second time, you schedule the track to be muted (or not) permanently at the next bar.  If you press mute a third time, you undo all this.  This is known as {\bf advancing mute}:

\begin{itemize}
\item If current track is {\it not} muted: {\bf Schedule Mute For One Bar} \(\rightarrow\)  {\bf Schedule Mute} \(\rightarrow\) {\bf Undo}
\item If current track is muted: {\bf Schedule Un-mute For One Bar} \(\rightarrow\)  {\bf Schedule Un-mute} \(\rightarrow\) {\bf Undo}
\end{itemize}

If you include a CC control surface (see {\it Remote Control over CC} below) you have the further option of advancing mute on any track individually.  This is useful to perform with Gizmo as a groove surface, where you trigger various tracks to play once or turn them on or off during the performance.

\paragraph{Monophonic and Duophonic Formats {\it (Advanced Step Sequencer Only)}}
These are alternative track formats you can choose when you initialize your sequence. These formats look the same as the regular layouts, but they don't play the same.  The purpose of these formats is to maximize your ability to make long sequences for a monophonic or duophonic sequencer, rather than short multitrack sequences for multiple synths.\footnote{You'll notice that this feels like a hacked-up, bad version of the Drum Sequencer's groups facility and you'd be right.  The Drum Sequencer was built from the ground up; but the Mono and Duo formats were added to the Step Sequencer long after the fact, so we can't do groups per se: and we don't have the memory to store group information anyway.  Instead this is the best   compromise I was able to make, knowing that its somewhat inconvenient to use.} When in Performance mode, the step sequencer will run through each track in turn rather than playing them all at the same time.  This allows you to, in some sense, create one long \(12 \times 16\) track rather than 12 separate 16-note tracks, for example.

If you have chosen a {\bf Monophonic} format, then the following things have changed in the step sequencer:

\begin{itemize} 
\item When in Edit or Play Position mode, the only track played is the current track.
\item The Patterns menu has been replaced with a menu called {\bf Repeats}, which indicates how a track is played in Performance mode.  Here you can set up a track to {\tt LOOP} forever, or to play 1, 2, 3, 4, 5, 6, 7, 8, 9, 12, 15, 24, 32, or 64 times, or you can mark the track as the {\tt END}.  This doesn't affect Edit or Play Position mode.
\item Specifically, in Performance Mode the first track will play {\it repeats} number of times: then the second track will play {\it repeats} number of times, and so on, until we reach an {\tt END} track or we finish all the tracks. At that point will the sequence terminate, loop back to the beginning, or go to a new sequence as usual.
\item There is only one MIDI channel.  Changing the channel on any track will change it on all tracks.
\item The Solo menu option no longer serves any purpose and does not function.
\item When in Play Position Mode, pressing the middle button will {\bf jump to the next track} to start playing (rather than mute).
\item When in Performance Mode, pressing the middle button will {\bf schedule the next track} to start playing (rather than schedule a mute).
\item When in Performance Mode, long-pressing the middle button will {\bf schedule the currently selected track} to start playing (rather than schedule a solo).
\item When in Performance Mode, the region reserved to display the current track channel will instead display the current track being played (which is not the same as the current selected track!).  Also the first LED will blink on the current track being played; though you may not see it if you have scrolled the track off the screen.
\end{itemize}

A {\bf Duophonic} format works very similarly, except that two tracks play at the same time in Performance Mode.  Duophonic is not available for 64 note tracks.  It differs from the Monophonic formats in the following ways;

\begin{itemize} 
\item In all modes, {\it pairs} of tracks are played rather than single tracks.
\item The {\bf Repeats} menu sets values for {\it pairs} of tracks.  
\item When in Performance Mode, the first track {\it pair} will play {\it repeats} number of times: then the second track {\it pair} will play {\it repeats} number of times, and so on.
\item When in Performance Mode, pressing the middle button will {\bf schedule the next track pair} to start playing.
\item When in Performance Mode, long-pressing the middle button will {\bf schedule the pair containing the currently selected track} to start playing.
\item When in Performance Mode, the region reserved to display the current track channel will instead display the {\it first track of the current track pair being played}. Also the first LED will blink on the current track {\it pair} being played.
\item In Edit or Play Position Mode, if you are editing the second track of a pair, the first LED of the first track in the pair will blink.
\end{itemize}



\paragraph{Menu Options}

If you Long Press the Select button, you'll call up the Step Sequencer's menu.  The Step Sequencer has a number of options:

\begin{itemize}
\item {\bf Solo} (or {\bf No Solo})\quad This plays only the current track, muting the others.  It has no effect in Mono/Duo.
\item {\bf Reset Track}\quad This wipes out and resets the track.  Unlike Long-Pressing the Middle Button while in Play Position mode, this doesn't just clear the track: it also resets all of its settings to their defaults.
\item {\bf Note Length (Track)}\quad This specifies the note length of notes in the track, as a percentage (0 is fully staccato and 100 is fully legato).   You can choose numerical values, or choose \texttt{DFLT}, which tells the Step Sequencer to use the default note length specified in the Options menu (see Section \ref{options}).
\item {\bf Out MIDI (Track)}\quad This specifies the MIDI Out Channel for notes in the track,    You can choose a value 1...16, or \texttt{- - - -} ({\it none}), or \texttt{DFLT}, which tells the Step Sequencer to use the default MIDI Out Channel specified in the Options menu (see Section \ref{options}).
\item {\bf Velocity (Track)}\quad This lets you specify a key velocity\footnote{By velocity we mean in the MIDI sense: how fast the key is struck; this basically translates to loudness.} that {\it all} notes in the track will use (as a value 1...127).  If you would instead prefer that each note in the track use its own independent key velocity as you had entered them, you can choose {\it none}.
\item {\bf Fader (Track)}\quad This lets you specify a volume modifier which is multiplied against the velocities of the notes in the track.  Possible values are 0...31, where 0 completely hushes the notes, 16 lets them play at their specified velocities, and 31 is roughly twice their velocity.  8 is one half-velocity, 4 is one quarter velocity, and so on.  In general, values larger than or smaller than 16 will make the notes louder or quieter respectively.
\item {\bf Type (Track)} {\it (Advanced Step Sequencer Only)}\quad Lets you specify the type of data stored in this track.  See discussion of {\bf Edit Mode} above.
\item {\bf Pattern} {\it (Non-Mono/Duo only)}\quad Lets you specify the four-bar pattern for this track in this group.  Your options are: \texttt{OOOO, O-O-, -O-O, OOO-, ---O, O--O, -OO-, OO--, --OO, OO-O, --O-, R1/8, R1/4, R1/2, ---X,} or\ \ {\tt XXXX}.  An {\tt O} indicates that the track will be played, and a {\tt -} indicates that it will be hushed.  The {\tt O} patterns are in pairs so you can swap the track being played with some other track.  The three patterns which start with {\tt R} hush at random each bar: the fraction indicates the probability that the track will be hushed.  An {\tt X} indicates {\it exclusive random:} if multiple tracks are designated as {\tt X} for a given bar, then exactly one of them will be played then (the others will be hushed).   {\tt ---X} is exclusive random for the fourth bar only and hushed otherwise; this makes possible random fills, and would be useful with {\tt OOO-}.   If only one track is {\tt X}, it's always played.  If two tracks are {\tt X}, they alternate.  Otherwise Gizmo tries to select a different track each time.

\item {\bf Repeats} {\it (Mono/Duo only)}\quad Lets you specify how many times a track will repeat in Performance Mode before going on to the next track: you can also mark a track as the End of a sequence.  Your options are {\tt LOOP}, 1, 2, 3, 4, 5, 6, 7, 8, 9, 12, 15, 24, 32, 64, or {\tt END}.

\item {\bf Edit}\quad A submenu for some simple cut/copy/paste-like operations:
\begin{itemize}
\item {\bf Mark}\quad Records the current track and edit position as the {\it mark position}.
\item {\bf Copy}\quad Copies the data at the mark position to the current track starting at the current edit position.  The copied data is wrapped around when it goes beyond the end of the current track.
\item {\bf Splat}\quad Copies the data at the mark position to the current track starting at the current edit position.  The copied data is {\it not} wrapped around when it goes beyond the end of the current track: it just stops there.
\item {\bf Move}\quad Works like Copy, except that the mark position track is then erased.
\item {\bf Duplicate}\quad Duplicate absolutely all of the data in the mark track to the current track (including note data, volume, MIDI out, etc.).  The specific mark (note) position is ignored.
\end{itemize}
\item {\bf Transpose (Track)} (or {\bf No Transpose (Track)})\quad This toggles whether this track responds to the user transposing the sequencer during performance mode.
%\item {\bf Clock Control} (or {\bf No Clock Control}) \quad This toggles {\it clock control mode}.  Normally, if you play or stop the sequencer, it will not send MIDI STOP or MIDI PLAY commands\,---\,if you want to start the MIDI clock you have to do so separately (see Section \ref{startingclock}).  However it's common for sequencers to issue these commands when you start or stop the sequencer, so as to also start or stop other sequencers listening over MIDI.  To do this, turn on clock control mode.  Note that this only has an effect if Gizmo is {\bf generating} a MIDI Clock (see ``MIDI Clock'' in Section \ref{options}).  Finally note that regardless of the settinmag of this options, if Gizmo is {\it using} an external MIDI Clock (again, see ``MIDI Clock'' in Section \ref{options}), the Step Sequencer will stop and start in response to incoming MIDI Clock commands.
\item {\bf No Echo} (or {\bf Echo}) {\it (Advanced Step Sequencer Only)}\quad This toggles {\it no-echo mode}.  Normally when you enter a note, it's automatically played so you get some feedback.  But in some circumstances this isn't desirable: for example, if you're entering {\it and} playing notes from the same keyboard, then this default behavior will cause two notes to be played whenever you enter a note.  {\bf No Echo} will turn this off.\footnote{Another closely related feature: if you're in play position mode and you have entered a note slightly before its time, Gizmo won't play it (the first time) when its time comes up, because you just played it yourself.  This isn't the case on the Uno: you'll hear a double-played note the first time.  There's not enough space on the Uno to add this feature, sorry.}

\item {\bf Rest Note} {\it (Advanced Step Sequencer Only)}\quad This selects a {\it rest note}, a key of your choice which, when pressed, enters a rest rather than a note.  If when selecting a rest note you instead press the Select button, this removes any rest note. 

\item {\bf Tie Note} {\it (Advanced Step Sequencer Only)}\quad This selects a {\it tie note}, a key of your choice which, when pressed, enters a tie rather than a note.  If when selecting a tie note you instead press the Select button, this removes any tie note. 

\item {\bf Length}\quad  Lets you set the length of the sequence to a custom value, or to \texttt{- - - -}, meaning to use the default initialized sequence length (16, 24, 32, 48, 64, or 96).  If the initialized length is 16, 24, or 32, then you can customize the length to anything less than it.  If the initialized length is 48, 64, or 96, then you can reduce the length to no less than 31 fewer notes.

\item {\bf Performance}\quad  This brings up the Performance Mode sub-menu.  You can select these menu items from any mode (not just Performance Mode), but they will only impact on Performance Mode:

\begin{itemize}
\item {\bf Keyboard}\quad  This menu option lets you specify what happens when you press keys or send other MIDI information during Performance Mode.  Your options are:

 \renewcommand\labelitemiii{$\diamond$}
\begin{itemize}
	\item {\texttt{-~-~-~-}}\quad Played notes are entered into the current track.  If the user has just changed the current track, then upon playing the first note the track is first cleared.
	\item {0} \quad Play along with the sequencer: MIDI data is routed out the default MIDI Out channel.
	\item {1--16} \quad Play along with the sequencer: MIDI data is routed out the channel specified.
	\item {\texttt{TRAN}} \quad When you press a key, this transposes sequenced notes (on those tracks which have been set to be transposable).
\end{itemize}

\item {\bf Repeat Sequence}\quad This specifies how many bars the sequence plays before it stops.  The default is {\tt LOOP} (forever), but you also have 1, 2, 3, 4, 5, 6, 8, 9, 12, 16, 18, 24, 32, 64, or 128 bars as options.
\item {\bf Next Sequence}\quad This menu option specifies {\it which} sequence the sequencer should switch to when it has completed the given sequence.  Your options are {\tt - - - -} (stop), or 0...9.
\end{itemize} 

\item {\bf Save} \quad This saves the sequence. You then choose the slot to save in.  Slots are displayed by number, plus an \texttt{S} (for Save), plus optionally an \texttt{R} or \texttt{S} or \texttt{D} (for Recorder or Step Sequencer or Drum Sequencer) if the slot is already used by that application.  This letter {\bf blinks} if the file isn't the type of your current application to warn you that you will be overwriting the other application's slot.

\item {\bf Options} \quad This brings up the Options menu, which enables you to do...
\end{itemize}

\paragraph{More Step Sequencer Options}

Beyond the settings above, the step sequencer responds to a number of options in the Options Menu.  See Section \ref{options} for detailed information on each of the options.  The Step Sequencer responds to the following options:


\vspace{1em}
\begin{tabular}{llllll}
Tempo& Note Speed& Swing & Note Length&Transpose&Click\\
In MIDI& Out MIDI&Control MIDI&MIDI Clock&Volume\\
\end{tabular}

\vspace{1em}
Of particular use are Tempo, Note Speed, Swing, and especially Click.


\paragraph{Controls Summary}  The Step Sequencer's controls can be a bit complex.  Here's what they do in each mode.  Note that once you're changing the tempo in the menu, you can also tap the middle button to adjust the tempo in addition to moving the knob.

\vspace{1em}
\hspace{-3.5em} {\small
\begin{tabular}{@{}rlll@{}}
\it Control					& \it Edit Mode	& \it Play Position Mode	& \it Performance Mode\\
\hline\\[-0.5em]
Back Button				& Exit  & Exit & Leave Performance Mode\\
Back Button (Long)			& Toggle Bypass & Toggle Bypass & Toggle Bypass\\
Middle Button				& Rest & Toggle Mute Track & Schedule Mute Track\\
{\it Mono/Duo}&&Next Track&Schedule Next Track/Unschedule Track\\
Middle Button (Long)			& Tie & Clear Track & Schedule Solo\\
{\it Mono/Duo}&&&Schedule Track\\
Select Button				& Start/Stop & Start/Stop & Start/Schedule Stop/Stop\\
Select Button (Long)			& Menu & Menu & Menu\\
Select + Middle Button (Long) 	&  Performance Mode &  Performance Mode & Schedule Next Sequence\\
Left Knob					& Change Track & Change Track & Change Track\\
Right Knob				& Change Note Position & Enter Edit Mode & Change Tempo\\
						& {\it or} Enter Play Position Mode && {\it (Press Back/Select when Done)}\\
Keystrokes, CC, etc.			& Enter at Edit Cursor&Enter at Play Cursor & Enter at Play Cursor {\it or}\\
						&				  &				   &Transpose {\it or} Route Out\\
\end{tabular}
}\\

\paragraph{Remote Control over CC}  If you have a remote control surface, you can greatly enhance the Step Sequencer's editing and performance chops by sending CC messages over Gizmo's Control MIDI Channel:

\begin{itemize}
\item In any mode you can {\bf jump directly to any track,} clear the current track, and directly edit the length, midi out, velocity, fader, transposability, and pattern of the current track.  You can also change a variety of sequencer or note settings and perform edits.

\item In Edit Mode or Play Position Mode you can also toggle the mute of any track independently and toggle solo.

\item In Performance Mode you can {\bf advance the mute of any track independently} and you can advance the solo.  You can also trigger a jump to the next sequence.

\item These operations are on top of the CC Control Midi operations (Section \ref{options}, see ``Control MIDI''), namely start/stop/continue, back button, middle button, select button, bypass, and left/right knob.
\end{itemize}

\vspace{1em}
\noindent {\small
\begin{tabular}{@{}rll@{}}
\it CC Number & \it CC Value	& \it Operation\\
\hline\\[-0.5em]
64 ... 75	&	Any		& Toggle Mute (or Schedule) Mute for track \(n - 63\), that is, track 1...12\\
76 ... 87	&	Any		& Select Track \(n - 75\), that is, track 1...12\\
88	&	Any		& Toggle (or Schedule) Solo\\
89	&	Any		& Toggle Transposable Track\\
90	&	Any		& Clear Track\\
91	&	Any		& Schedule Next Sequence\\
92	&	Any		& Toggle Click Track\\
93	&	Any		& Mark\\
94	&	Any		& Copy\\
95	&	Any		& Splat\\
116	&	Any		& Move\\
117	&	Any		& Duplicate\\
16	&	0--127		& Set Track Note Length (press Select to set, Back to cancel)\\
17	&	0--127		& Set Track MIDI Out (press Select to set, Back to cancel)\\
18	&	0--127		& Set Track Velocity (press Select to set, Back to cancel)\\
19	&	0--127		& Set Track Fader (press Select to set, Back to cancel)\\
20	&	0--127		& Set Track Pattern (press Select to set, Back to cancel)\\
21	&	0--127		& Set Gizmo's Tempo (press Select to set, Back to cancel)\\
22	&	0--127		& Set Gizmo's Transpose (press Select to set, Back to cancel)\\
23	&	0--127		& Set Gizmo's Volume (press Select to set, Back to cancel)\\
24	&	0--127		& Set Gizmo's Note Speed (press Select to set, Back to cancel)\\
25	&	0--127		& Set Gizmo's Play Length (press Select to set, Back to cancel)\\
26	&	0--127		& Set Gizmo's Swing (press Select to set, Back to cancel)\\
27	&	0--127		& Set Performance Keyboard Mode (press Select to set, Back to cancel)\\
28	&	0--127		& Set Performance Repeat (press Select to set, Back to cancel)\\
29	&	0--127		& Set Performance Next Sequence (press Select to set, Back to cancel)\\
30	&	0--127		& Set Sequence Length (press Select to set, Back to cancel)\\
\end{tabular}
}

\vspace{1em}
\noindent Note that the Set... CCs transfer to the relevant menu, then act as proxies for the left knob.  This means that once you've transferred to a menu, {\it any} of them can be used to set the menu, along with the left knob CC.  This can be confusing.

Normally, if you send a CC Number 66...75, it'll toggle the mute no matter what value you send.  But some controllers have a feature where the button you press lights up if non-zero CC values are sent and turns off when zero CC values are sent.  You could capitalize on this feature to have your controller send a non-zero value to unmute and a zero value to mute, and then add the the following line to the file {\tt All.h}:

\begin{verbatim}
#define INCLUDE_STEP_SEQUENCER_CC_MUTE_TOGGLES
\end{verbatim}

Note that this doesn't change the Advance Mute in Performance Mode behavior, just the non-Performance Mute Toggle.  The Advance Mute behavior remains as it is: pushing a button just triggers the behavior to happen in the future.

\clearpage



\section {The Drum Sequencer}
\label{drumsequencersec}

\noindent {\bf Note: }{\it The Drum Sequencer can't run on the Arduino Uno.}\vspace{1em}

\begin{wrapfigure}{r}{3in}
\vspace{-3.5em}\includegraphics[width=3in]{drumsequencer}
\vspace{-1em}\caption{\small Drum Sequencer Display}\vspace{-2em}
\label{drumsequencer}
\end{wrapfigure}

The Drum Sequencer is similar in some respects to the Step Sequencer, but it has features and tradeoffs more oriented towards drum machine style sequencing:

\begin{itemize}
\item The Step Sequencer plays only one multitrack sequence at a time (except in Mono or Duo format).   But the Drum Sequencer can store up to 15 of them (known as {\bf groups}) in a single sequence to make a drum rhythm for a whole song.  These groups can vary in length and speed.  The term {\bf sequence} is reserved for all the groups together as a whole.
\end{itemize}

\begin{itemize}
\item The Drum Sequencer can string these (up to 15) groups in a {\bf chain} 20 groups long; the same group can show up multiple times.  Each step in the chain (known as a {\bf transition}) states the group and how many times to repeat it before going on to the next; and there are some randomization options as well.
\item The Drum Sequenecer has sufficient memory for these things because it doesn't store nearly the amount of per-step information the Step Sequencer does. The Step Sequencer stores in its steps {\it notes} with pitches and velocities; or {\it control changes} like CC or NRPN.  But in the Drum Sequencer each track is tied to a single {\bf drum note} played at a certain velocity on a certain MIDI channel.  Thus each step in the track is just an ``on'' or ``off'', indicating that the drum should be triggered.
\item The Step Sequencer both plays notes (MIDI Note On) and releases them (MIDI Note Off).  The Drum Sequencer only plays notes (MIDI Note On).  It never releases them: drums typically don't need that.
\item The Drum Sequencer offers many more {\bf tracks} and {\bf initialization formats} than the Step Sequencer.
\end{itemize}

\paragraph{Basic Organization} A drum sequence is organized as follows:

\begin{itemize} 
\item The sequence is divided into \(\leq 15\) {\bf groups,} depending on the initialization format.\footnote{In terminology of other drum sequencers, notably Roland-style drum machines, these are commonly called {\bf patterns}.  Why aren't we using this terminology?  Because Gizmo already uses {\it pattern} to mean something else.} Each group is a repeating ``sequence'' in and of itself in the Step Sequencer sense of the term.  Each group has a maximum length (number of steps) which you specify up front when you choose the drum sequence format\,---\,8, 16, 32, or 64\,---\,but you can shorten individual group lengths to less than this.  Each group also has a {\it speed} which may be a multiple or fraction of the standard tempo.
\item Each group has \(\leq 20\) rows of steps, again depending on the initialization format.  A thread of rows through all the groups\,---\,row \#5 running through groups 0...15, say\,---\, is called a {\bf track}.  Each track is associated with a {\bf drum note}, which is a MIDI note pitch, note velocity, and MIDI channel.  When a step on that track is played, Gizmo will send a Note On message with that pitch and velocity to that channel.  A global accent track is available, and you can make per-drum-note accent tracks. Tracks can also be muted and soloed.
\item Each row in each group can have its own four-bar {\it pattern} in the Step Sequencer sense of the term.
\item Just like the Step Sequencer, the Drum Sequencer has a {\bf Performance Mode}.  In performance mode, the sequencer will (among other things) play groups in a certain order according to a {\bf chain} of up to 20 {\bf transitions}.\footnote{Again, traditionally, the term for this is probably a {\it chain} of {\it patterns}.}  Each transition specifies a group to play, plus how many times to play it before transitioning to the next one.  Transitions can also be set up to play random groups and to do small loops.  In performance mode you can trigger a premature transition manually, as well as a fill.
\end{itemize}

\begin{wrapfigure}{r}{1in}
\includegraphics[width=1in]{none.pdf}
\vspace{-2em}\caption{\small None}\vspace{-1em}
\end{wrapfigure}

\paragraph{Initialization} When you choose the Drum Sequencer from the root menu, you will be asked to load a sequence from a slot.  You can do this, or you can create a new empty sequence by choosing \texttt{-~-~-~-} (meaning {\it none}).  Slots are displayed by number, plus an \texttt{L} (for Load) and an optional \texttt{R} (for Recorder), \texttt{S} (for Step Sequencer), or \texttt{D} (for Drum Sequencer) if the slot already has a file from one of these applications.  This letter {\bf blinks} if the file isn't a Drum Sequencer file to warn you that you might eventually overwrite a slot when you save.

If you have chosen {\it none}, or have chosen an empty slot or one presently filled by another application, you'll be asked to initialize the sequence first by specifying its {\it format}.  Your 16 options\footnote{Why only multiples of 8 for group length?  You're choosing the {\it maximum} group length, and this data has to be packed into (8-bit) bytes.  However you can later reduce the length of any of the groups to any value less than maximum.  See the \textbf{Menu}.} are:

\vspace{0.5em}\begin{center}{\small\noindent\begin{tabular}{@{}rrrr@{}}
&\it Group Length&\it Number of&\it Number of\\
&\it in Steps&\it Groups&\it Tracks\\
\hline\\[-0.9em]
1&8&10&20\\
2&16&6&20\\
3&32&3&20\\
4&8&13&16\\
5&16&8&16\\
6&32&4&16\\
7&64&2&16\\
8&8&15&12\\
 \end{tabular}%
\hspace{8em}\begin{tabular}{@{}rrrr@{}r@{}}
&\it Group Length&\it Number of&\it Number of\\
&\it in Steps&\it Groups&\it Tracks\\
\cline{1-4}\\[-0.9em]
9&16&11&12&\ \ \(\longleftarrow\) {\it default}\\
10&32&6&12\\
11&64&3&12\\
12&16&15&8\\
13&32&9&8\\
14&64&5&8\\
15&32&12&6\\
16&64&10&4\\
 \end{tabular}}\end{center}
 \vspace{-0.5em}
 
\begin{wrapfigure}{r}{1.8in}
\vspace{-1em}
\includegraphics[width=1.8in]{20val.pdf}
\vspace{-1em}\caption{\small Track and Group / Transition Numbers (Tracks and Transitions can go to 20; Groups go to 15)}
\vspace{-2em}
\label{trackgrouptransitionnumber}
\end{wrapfigure}

\vspace{1em}Each option is displayed on-screen as {\it steps/groups/tracks}.  Why these choices in particular? Because they squeeze the most out of Gizmo's very limited memory.   After you have chosen a format, you will be asked to provide one more piece of information: a {\bf base drum note}.  Drum synthesizers typically have all their notes laid out in order, starting with one lowest note.  For example, the Novation Drumstation places its TR-808 Kick at C0 (MIDI note key 24), then the TR-808 Rimshot at C\#0, then the TR-808 Snare at D0, and so on.  If you give C0 as your base note, then Gizmo will set the drum note pitch of track 1 to C0, the pitch of track 2 to C\#0, the pitch of track 3 to D0, and so on, and so will set up each track to play a different drum. This can help greatly in laying out the drums for your sequence.  (You can of course modify all of this afterwards).

\paragraph{Screen Layout} The screen layout, shown in Figure~\ref{drumsequencer}, is much like the one in the Step Sequencer, except that the screen is showing you only one group at a time.  The {\it Group or Transition Number} area tells you the group being displayed.  The {\it Tracks} region is where sequencer notes are displayed and edited in that group: each horizontal row corresponds to a track.  The {\it Track Number} area tells you the number of the currently-edited track.  Figure~\ref{trackgrouptransitionnumber} explains how to interpret this.

Many of the remaining LEDs are similar to ones in the Step Sequencer.  The {\it Sequence Iteration} LED tells you which bar in a four-bar repeating pattern the is being played (see Figure~\ref{bars2}).  The {\it Performance Mode} LED tells you if you're in performance mode.  The {\it Next-Transition Trigger} LED lights up if, when in performance mode, you have manually triggered the sequence to transition at the end of playing this group.  The {\it Track is Playing} LED tells you if the current track is now playing according to its pattern.  The {\it Region} LED tells you steps will be toggled, set, or cleared, when you play certain notes (we'll get back to that). And finally, the {\it Stopped} LED is lit when the drum sequencer is not running.

\paragraph{The Four Modes} The original Step Sequencer had three modes: the Drum Sequencer has four. Three modes are determined by the horizontal position of the cursor on the screen, which you can change by turning the right knob.  Turn the knob to the extreme left and the cursor is replaced with a special bar cursor as you enter {\bf Play Position Mode}.  Turn the knob to the extreme right and the cursor is replaced with a different bar cursor as you enter {\bf Group Mode} (this is the new one).  Turn the knob somewhere in the middle and you'll find yourself in {\bf Edit Mode}.  Finally, pressing a special button sequence enters {\bf Performance Mode}.  In each of these modes the function of the buttons and (sometimes) the knobs works differently. 

\paragraph{Edit Mode, and Setting Up a Track} Move the right knob somewhere in the middle to enter edit mode.  This mode works much like it did in the Step Sequencer: you can enter notes right at the single blinking cursor (but there are now other options too).  Moving the right knob to the right or left changes the step position of the cursor in the group (note that if you have more than 16 steps, they'll wrap around).  Moving the left knob up to the right or left changes the track you're editing in the group.  You'll see the cursor move up and down, and you'll also see the track number indicator change to tell you what track you're on.

Before we continue in Edit Mode, let's set up the first track.  Make sure that you have selected track 1 by moving the left knob to the far left.  Long-press the Select button, you'll enter the {\bf Menu}.  From here, choose {\bf Track}.  From there, select {\bf Out MIDI} to pick the MIDI channel that this track will send notes to.  You can choose a value 1...16, or \texttt{----} ({\it none}), or \texttt{DFLT}, which tells the Step Sequencer to use the default MIDI Out Channel specified in the Options menu (see Section \ref{options}).  The Track menu also sports the {\bf Velocity} option, where you can choose the {\it note velocity} from 1...8.  The default value is 6.  See {\bf Velocities} later for information on how to interpret the 1...8 values as MIDI velocities.  

If you like, you can then {\it distribute} this information to all other tracks by choosing the {\bf Distribute} option.  You can manually change track information on other tracks by moving the left knob in Edit mode (or in Play Position Mode) to change the track you'd like to work on, then selecting the appropriate menu option.

Exit the menu, and we're back in Edit mode.  But let's first make one more change before we enter data: let's select the drum note to be played when a step is triggered.  To do this, long-press the Middle button.  You will be asked to play a {\tt NOTE}, and that note will become the {\bf new drum note} for that track.  

\enlargethispage{0.5em}
Let's say you don't know what note to choose.  What to do?  When you see {\tt NOTE}, press the Back button.   You will enter a screen which shows you the current note the track is using.\footnote{Gizmo displays the {\it note}, and {\it octave}, like C\#4.  Octaves start at 0 in MIDI (A0 is the first note); but manufacturers differ in how they display octaves: for example, Yamaha adds 2, so a Yamaha C\#0 is a MIDI/Gizmo C\#2; and a MIDI/Gizmo C0 is a Yamaha C -2.}   While the note is being displayed, you can try out other notes by pressing any key you like, and it'll get played directly on your drum synthesizer.  When you're ready, you can go back to {\tt NOTE} by pressing the Select button, and from there, you can enter the note of your choice.  Or you can press the Back button to escape all of this.

\begin{wrapfigure}{r}{4in}
\vspace{-0.5em}
\hspace{\fill}\includegraphics[width=4in]{EditModeKeys}
\vspace{-1em}
\caption{\small Edit Mode key input, starting at Middle C (MIDI 60).  White keys toggle one of 16 steps each.  C\# sets these keys to correspond to steps 1...16 (region ``A'').  D\# sets them to correspond to steps 17...32 (region~``B'').  F\#~sets them to correspond to steps 33...48 (region ``C'').  G\# sets them to correspond to steps 39...64 (region ``D'').  The default is region ``A''.  A\#~toggles the current step ({\it ``here''}) at the edit cursor.  Holding down a high key as shown while playing a key will enter it in the next track instead.  This is useful if you have set up the next track as an accent track (see {\bf Accents}).  Playing a key below middle C will just play the current drum sound. (You can change the Middle C to a different C from the menu).}
\vspace{-1em}
\label{editmodekeys}
\end{wrapfigure}

Now we've set our track's MIDI channel, velocity, and drum note, and we're back in Edit mode and ready to enter data.  In Edit mode data can be entered at the edit cursor, like the Step Sequencer, by pressing the Middle button; but usually it's entered in another way.  Namely, the Drum Sequencer is set up more like traditional drum machines: keys correspond to individual steps.  Your edit cursor specifies the current track.  When you press a white key (starting at middle C, as shown in Figure~\ref{editmodekeys}), a step on that track is toggled on/off.  You can also press the key A\# to toggle the step exactly at the edit cursor (like the Middle Button).  

\begin{wrapfigure}{r}{1.3in}
\hspace{\fill}\includegraphics[width=1.3in]{drumregion}\hspace{\fill}%
\vspace{-1em}
\caption{\small Drum Region and Operation Patterns.}
\label{drumregionpatterns}
\vspace{-1em}
\end{wrapfigure}


This all works fine if you have an 8- or 16-step bar, but if you specified a longer group (32 or 64 steps), you have to jump through an extra hoop.  To toggle steps 17...32, for example, press the D\# key, which shifts the {\bf drum key region} from 1...16 to 17...32.  Now pressing a white key will toggle steps from 17...32.  Similarly to toggle steps 33...48, press the F\# key.  And to toggle steps 49...64, press the G\# key.  To go back to steps 1...16, press the C\# key.  The current drum region is specified by the Drum Region LEDs, as shown in Figure~\ref{drumregionpatterns} (ignore the Operation column).  Holding down a high key as shown while playing a key will enter it in the next track instead.  This is useful if you have set up the next track as an accent track (see {\bf Accents}).  Finally, any note less than middle C just plays the current drum note rather than entering it.

In Edit mode you can also {\bf start the sequencer playing} or {\bf stop} it by pressing the Select button.  This works much like the Step Sequencer in that only the current group is played (and repeatedly).  When you start playing the Drum Sequencer, it waits for the clock to reach a measure boundary before beginning.  Note that if you start playing the Drum Sequencer, but the clock isn't running, it won't play at all.  You'll need a clock (internal or external) to drive it.

If you long-press the Select button, you enter the {\bf menu}, whose options we'll discuss later.    If you long-press both the Middle and Select buttons, you'll enter {\bf Performance Mode}, also discussed later.

\begin{wrapfigure}{r}{1.25in}
\vspace{-1em}
\hspace{\fill}\includegraphics[width=0.7in]{playpositionmode}\hspace{\fill}
\caption{\small The Play Position Mode Cursor.  The current track is marked by the two dots.}
\vspace{-4em}
\label{playpositionmode2}
\end{wrapfigure}



\paragraph{Play Position Mode}  This mode is entered from Edit mode by twisting the right knob to the far left.  The edit cursor will go off-screen and will be replaced with the {\it play position mode cursor}, a bar with two dots as shown in Figure~\ref{playpositionmode2}.  Play Position Mode works similarly to the Step Sequencer,  but note entry is very different.   Once again, instead of entering notes, keys you play will correspond to doing operations on the currently-playing steps on different tracks. The three operations are:

\enlargethispage{1em}

\begin{itemize}
\item {\bf Toggle} a step: change it from on to off, or from off to on.  This is the default.
\item {\bf Set} a step: turn it on.
\item {\bf Clear} a step: turn it off.
\end{itemize}

\begin{wrapfigure}{r}{4in}
\vspace{-1em}
\hspace{\fill}\includegraphics[width=4in]{PlayPositionModeKeys}
\vspace{-1em}
\caption{\small Play Position Mode key input, starting at Middle C (MIDI 60).  White keys toggle, set, or clear the currently-played step on one of 20 tracks.  C\# sets these keys to {\it toggle} their steps.  D\# sets them to {\it set} (turn on) their steps.  F\#~sets them to {\it clear} (turn off) their steps.  The default is operation is toggle.  A\#~toggles/steps/clears the currently played step ({\it ``here''}) at the currently selected track.  Playing a key below middle C will just play the current drum sound. (You can change the Middle C to a different C from the menu). }
\label{playpositionmodekeys}
\end{wrapfigure}

The current operation is chosen by  pressing a black key.  If you press C\#, the operation is set to {\it Toggle} (the default).  If you press a D\#, the operation is set to {\it Set}.  If you press a F\#, the operation is set to {\it Clear}.  The Drum Region LEDs will light up to indicate the current operation, as shown earlier in Figure~\ref{drumregionpatterns}.  Pressing white keys will then perform the current operation on the currently-played step of one of twenty tracks, as shown in Figure~\ref{playpositionmodekeys}.  If you press A\#, then the operation is performed on the currently-played step of the currently-selected track.  (Unlike in Edit mode, pressing the Middle button doesn't do the same thing as pressing A\#).  ally, any note less than middle C just plays the current drum note rather than entering it.

For obvious reasons, and given its name, Play Position Mode really only works properly when you are playing the sequence.  Just like in Edit mode, you start playing (or stop playing) by pressing the Select button.  Again, this works much like the Step Sequencer in that only the current group is played (and repeatedly); see the Edit Mode for information about how this interacts with the MIDI clock.   Also like Edit mode, if you long-press the Select button, you'll be taken to the {\bf menu}.    Just as in the Step Sequencer, pressing the Middle button mutes or unmutes the current track.    And likewise, long-pressing the Middle button clears the track (within the current group only), but with a twist: if the track is {\it already} clear, there is a menu option to optionally randomize notes in the track when clearing a second time.  Finally, long-pressing the Select and Middle buttons together enters {\bf Performance Mode}.

\paragraph{Velocity Curves}  The Drum Sequencer's velocities are simply the numbers 1...8.  How do these translate to MIDI velocities 0...127?  There are two options.  We can either interpret 1...8 as {\it linear} or {\it exponentially increasing} values: which you prefer would depend on what fits better with the velocity curve of your drum synthesizer.  The default is a linear curve, but you may prefer an exponential curve if it makes accents more consistent. 

{\small
\begin{center}\begin{tabular}{r|rrrrrrrr}
\it Drum Sequencer Velocity& 1 & 2 & 3 & 4 & 5 & 6 & 7 & 8\\
\it MIDI Velocity (Linear) & 16 & 32 & 48 & 64 & 80 & 96 & 112 & 127\\
\it MIDI Velocity (Exponential) & 25 & 32 & 40 & 50 & 64 & 80 & 100 & 127\\
\end{tabular}\end{center}
}

Notice that with the exponential curve, increasing (accenting) by +1 changes the MIDI velocity, and perhaps the underlying volume by +25\%; increasing by +2 changes by +60\%; increasing by +3 changes by +100\%; increasing by +4 changes by +150\%; and so on.  That's convenient.  But your drum machine might {\it already} have an exponential curve built in, and so then you'd want the linear velocity curve.  You can switch between curves with a menu option.


\paragraph{Accents}  The Drum Sequencer can only have one note velocity per track.  To play accents (louder beats than normal), you need to set up another track to do the accent notes for a given track, or you can have a 808-style {\bf global accent track}.  The former is more flexible, but the latter is easier and doesn't eat up as many tracks.

Let's start with the first choice.  To add accents to a given track, you just set up a second track\,---\,normally the very next track\,---\,with the same drum note and MIDI channel, but a different velocity, and play accented beats into that.  There's a helpful menu option called {\bf Accent Track}, on page~\pageref*{accenttrack}, which does exactly this, with an accent velocity of +2.  There's a little trick to make entering accents more convenient: if you hold down a low key or high key, as shown in Figure~\ref{editmodekeys}, while playing another note, then that note will instead be entered in the next track (notionally your accent track), not the current track. 

\begin{wrapfigure}{r}{1.25in}
\vspace{-1em}\hspace{\fill}\includegraphics[width=0.7in]{Patterns}\hspace{\fill}
\caption{\small Four Bar Pattern LEDs.}
\vspace{-2em}
\label{bars2}
\end{wrapfigure}

Now to the global accent track.  The very last track in your sequence can be specially set up not to play beats, nor accents per se, but to {\it define which notes are accents for all other tracks.}  Thus if there is a beat in this track, then when we come to that beat, {\it all other track drum-beats} will automatically be accented by a certain amount.  To do this, just set the final track's MIDI channel to one of the ``accent channels'' {\tt A +1} through {\tt A +6}.  Now when a drum note is turned on in this track no note is played: but other track notes are accented, having their velocities increased by +1 to +6 respectively.  The global accent track's track velocity, mute, and solo do nothing; and you can't copy, swap, or distribute track info to or from a global accent track.



\begin{wrapfigure}{r}{1.25in}
\vspace{-1em}
\hspace{\fill}\includegraphics[width=0.7in]{rightmode}\hspace{\fill}
\caption{\small The Group Mode Cursor.  The current track is marked by the two dots.  Compare to the Play Position Mode cursor.}
\vspace{-2em}
\label{groupmode}
\end{wrapfigure}



\paragraph{Patterns}

In the Step Sequencer, each track can be set up to follow some four-bar {\bf pattern} as specified by a {\bf menu} option.  In the Drum Sequence it's the same, except that each track {\it in each group} can have its own separate pattern.   The pattern LEDs indicate which of the four bars is currently being played, as indicated in Figure~\ref{bars2}.

\paragraph{Group Mode}  The Drum Sequencer has up to 15 {\it groups}, each its own sequence.  You can string together a list of 20 groups with other information to form a full song.  But first you need to edit the groups, and to do that you need to get around from group to group.  That's what Group Mode is for.

Group Mode is entered by twisting the right knob to the far right, which moves the edit cursor to the end of the track.  The cursor is replaced with a bar cursor similar to the one in Play Position Mode, but slightly different.  It's shown in Figure~\ref{groupmode}.

In Group Mode the buttons {\it and} the left knob work differently than in the other modes.  Let's start with the left knob.  If you turn this knob, it will scroll through the {\bf groups} rather than through the {\it tracks}.  Doing so will also reset the sequencer play position to the beginning.  %Press the Select button to confirm your choice, or the Back button the cancel.  


\begin{wrapfigure}{r}{1in}
\vspace{-1em}
\includegraphics[width=1in]{sure.pdf}
\caption{Sure?}
\label{exit}
\vspace{-1em}
\end{wrapfigure}

As usual, pressing the Select button starts/stops the group as in other modes, long-pressing the Select button enters the {\bf menu} as usual, and similarly long-pressing the Select and Middle buttons together will enter {\bf Performance Mode}.  However, pressing the middle button will {\bf increment} the group (go to the next group), wrapping to the first group when you have gone beyond the last group.  Additionally, long-pressing the Middle button will {\bf clear the group}, much as long-pressing the Middle button in Play Position mode will clear the track.  Because clearing an entire group is quite destructive, Gizmo will first ask you {\tt SURE?}

%The buttons are largely used to move to different Groups as well.  Pressing the Select button doesn't start/stop like in the other modes; instead, it will increment the group, that is, it will switch to editing the next group. Incrementing wraps around to the first group when you go beyond the last group.   Pressing the Middle button will decrement the group: and it wraps around to the last group when you go below the first group.  
%Long-pressing the Middle button will {\bf clear the group}, much as long-pressing the Middle button in Play Position mode will clear the track.  Because clearing an entire group is quite destructive, Gizmo will first ask you {\tt SURE?}  Pressing the Select button will confirm your action; pressing the Back button will go back.  Long-pressing the Select button enters the {\bf menu} as usual, and similarly long-pressing the Select and Middle buttons together will enter {\bf Performance Mode}.

When you play keys in Group Mode, it works like Play Position mode.

\paragraph{Editing the Transition Chain}

Once you have finished editing your groups, you can use the {\bf Chain} menu option to string them together into a {\bf chain of transitions} from group to group.  You can then play them in the order specified in the chain.   A group can appear more than once in a chain, that is, you can play the same group more than once in different parts in your song.  A chain can be up to 20 transitions long.

A {\bf standard transition} holds three things:

\begin{itemize}
\item The transition number (1...20)
\item Which {\it group} is playing (1... {\it Max groups}, up to 15, the amount depends on your initialization format)
\item How many times the group's playing repeats (\texttt{LOOP}, 1, 2, 3, 4, 5, 6, 7, 8, 9, 12, 16, 24, 32, 64, or \texttt{BIG~LOOP})
\end{itemize}

We'll get to {\tt BIG LOOP} in a moment.  Plain old {\tt LOOP} indicates that the group should play repeatedly forever, never transitioning from then on, unless you manually schedule a transition for it (you can do this in Performance Mode).  

There are also {\bf special transitions}, chosen by selecting {\tt - - - -} for your group number.  These include:

\begin{itemize}
\item Play 1, 2, 3, or 4 times, or repeat forever (\texttt{LOOP}).  Each time, randomly select from among groups 1--2 which group to play.  These are denoted by {\tt R2-1} ... {\tt R2-4} or {\tt R2-L}.
\item Play 1, 2, 3, or 4 times, or repeat forever (\texttt{LOOP}).  Each time, randomly select from among groups 1--3 which group to play (or 1--2 if there are only 2 groups).  These are denoted by {\tt R3-1} ... {\tt R3-4}  or {\tt R3-L}.
\item Play 1, 2, 3, or 4 times, or repeat forever (\texttt{LOOP}).  Each time, randomly select from among groups 1--4 which group to play (or 1--3 if there are only 3 groups; or 1--2 if only 2).  These are denoted by {\tt R4-1} ... {\tt R4-4} or {\tt R4-L}.
\item Mark this transition as the {\texttt{END}}.  When at the end of the transitions, the Drum Sequencer will stop, repeat the whole chain, or continue to the next sequence depending on what you specify in the {\bf Performance Menu} options.  If none of your transitions is {\bf end}, then the end will be the end of the chain.   The very first transition in a chain is not permitted to be {\bf end}.
\end{itemize}

\begin{figure}[t]
\centering\includegraphics[width=6.5in]{BigLoop}
\caption{Making looped groups with {\tt BIG LOOP}, manual transitions, and an optional repeating sequence.}
\vspace{-1em}
\label{bigloop}
\end{figure}

And now to {\tt BIG LOOP}.  When in performance mode (discussed below) you can manually force a transition to occur at the end of the current bar, even if the number of repeats hasn't been completed yet.  This is useful in the following way: you could set up Transition 1 to be (say) group 1 repeated forever (LOOP), then Transition 2 to be group 2 repeated forever, then Transition 3 to be group 3 repeated forever, etc.  When playing the chain, you'll be in group 1 until you decide to force the transition, which would take you to group 2, and then another manual forced transition at the desired time would go to group 3.

This is all fine and good, but another common need would be to loop though several groups in a row forever until manually forced to transition out of them.  For example, imagine you wanted to loop forever through groups 1, 7, and 3 as a sequence, then loop forever through group 2, then loop forever through groups 3, 4, and 6 as a sequence.  This is where {\tt BIG LOOP} comes in.  When a transition is set to {\tt BIG LOOP}, then after playing its group {\it once}, it will hunt back though previous transitions until it finds the first transition {\it after} some other infinite loop (a BIG LOOP, a LOOP, or a random LOOP).  If it finds none, it will use the very first transition.  It then loops back to this transition.  As shown in Figure~\ref{bigloop}, this can be used to chain together multi-transition groups separated by manual transitions.\footnote{BIG LOOP is infinite.  You can't chain together groups with finite ``big loops'': instead, you should use multiple sequences.}

One weakness of BIG LOOP is that the final transition in the loop will be played exactly 1 time.  That's the way it is: Gizmo doesn't have the storage to do otherwise. If you need its group played \(n\) times, set up the previous transition to be that same group played \(n-1\) times.  Other transitions in the loop may be repeated multiple (non-infinite) times, as shown in Figure~\ref{bigloop}, though most commonly they'd be 1 time each as well.


\paragraph{Performance Mode}  The function of Performance Mode is to make it easy to perform a chain of transitions; it works much like Performance Mode does in the Step Sequencer. No, we've not discussed how to edit transitions yet\,---\,we'll get to that when we discuss the menus.  Performance Mode is entered by long-pressing the Select and Middle buttons together while in any of the other three modes.  In Performance Mode, the cursor will look the same as the Play Position cursor.

Pressing the Select button does a start/stop, but not in the same way it does in the other modes.  Instead of starting playing the given group, this button will {\it start the transition chain}, and so begin playing the group associated with the first transition.  When the transition is finished, Performance Mode will then continue on to the next transition, and so on.   Pressing the Select button while playing will stop and reset to the beginning of the chain when the latest iteration of is complete.  There is a performance menu option to specify whether stopping will occur at the last note of the current iteration, or the first note of the {\it next} iteration.  If you press the Select button a second time before stopping has occurred, then it will stop and reset immediately.  There is also a menu option to pause performance without resetting.  

In Performance Mode, the left knob changes the current track, but the right knob changes the {\bf tempo}.  After changing the tempo, pressing Select will choose the tempo while pressing Back button will cancel.  

Performance mode handles the keyboard in much the same way as the Step Sequencer.  As determined by a {\bf menu} option, you can treat the keyboard just like Play Position Mode; or you can route the key notes out any MIDI channel; or you can select the {\tt PICK} option, which allows you to schedule specific transitions by pressing keys.  Unlike the Step Sequencer, you cannot transpose with the keyboard.

\begin{wrapfigure}{r}{4in}
\vspace{-0.5em}
\hspace{\fill}\includegraphics[width=4in]{TransitionKeys}
\vspace{-1.5em}
\caption{\small Transitions scheduled by key using {\tt PICK} in Performance Mode. (You can change the Middle C to a different C from the menu).}
\vspace{-1em}
\label{transitionkeys}
\end{wrapfigure}

In Performance mode you can chain together not only groups but entire sequences.  You can specify that a given sequence will {\bf repeat} for some number of iterations (through all its transitions), and then will either stop or will move on to {\it another sequence}.  This is also handled as a menu option.

Again, just as in the Step Sequencer, in Performance mode, you can schedule various things to occur at the next iteration of the group.  If you press the Middle button, you can {\bf schedule a mute} (or unmute) to affect the current track.  If you long-press the Middle button, you can {\bf schedule a solo} (or unsolo) on this track.  If you long-press the Select and Middle buttons together, you can {\bf schedule a transition} or {\bf schedule playing the next sequence} (the sequence you had selected to transition to at the end of the transitions) to happen immediately after the current iteration has finished.  Which of these happens depends on a setting in the Performance mode menu.\footnote{If you would like to do either independently, you can do so via MIDI CC.}

Muting and solo schedule in the same way as the Step Sequencer, but it's worth repeating it here. If you press solo, you won't {\it toggle} the solo.  Instead, you {\it schedule} the solo to be toggled at the beginning of the next bar.  If you press a second time, you undo your previous action.  So in short, pressing buttons go through this loop, which is known as {\bf advancing solo}:

\begin{itemize}
\item If you are currently {\it not} soloed: {\bf Schedule Solo} \(\rightarrow\) {\bf Undo}
\item If you are currently soloed: {\bf Schedule Un-solo} \(\rightarrow\) {\bf Undo}
\end{itemize}

If you press the Middle button during Performance mode, one of two things happens (depending on a setting in the Performance mode menu): either you schedule a {\bf fill}, or you schedule a {\bf track to be muted}.\footnote{If you would like to do both fills and mutes, you can do so via MIDI CC.}

If you have scheduled a fill, then starting at the bar, Gizmo will not play the current group, but rather will play the {\bf fill group} once.  You specify the fill group as a Performance mode menu option.  The current group's speed, pattern, and length still take precedence: but the fill group's drum steps will be used instead.  Once the current group's bar length has been finished, play will revert to the current group.  Note that even while the fill group is playing, the current group will still be displayed. You can cancel the fill prior to it starting by pressing the Middle button again.  

Similarly, you can schedule the track to be muted (or not), at the next bar, for exactly one bar, and then to return to its present state.  If you press mute a second time, you schedule the track to be muted (or not) permanently at the next bar.  If you press mute a third time, you undo all this.  This is called {\bf advancing mute}:

\begin{itemize}
\item If current track is {\it not} muted: {\bf Schedule Mute For One Bar} \(\rightarrow\)  {\bf Schedule Mute} \(\rightarrow\) {\bf Undo}
\item If current track is muted: {\bf Schedule Un-mute For One Bar} \(\rightarrow\)  {\bf Schedule Un-mute} \(\rightarrow\) {\bf Undo}
\end{itemize}

If you include a CC control surface (see {\it Remote Control over CC} below) you have the further option of advancing mute on any track individually.  This is useful to perform with Gizmo as a groove surface, where you trigger various tracks to play once or turn them on or off during the performance.

\paragraph{Menu Options}

If you Long Press the Select button at any time, you'll call up the Drum Sequencer's menu:%  The Drum Sequencer has many menu options:

\begin{itemize}
\item {\bf Solo} (or {\bf No Solo}) {\it All modes but Performance Mode}\quad This plays only the current track, muting the others.
\item {\bf Pause} {\it Performance Mode only}\quad This pauses the Drum Sequencer rather than stopping it.  Starting again will start from the current position rather than from the beginning of the transition chain.
\item {\bf Mark}\quad This sets the {\bf mark} to the current group and current track.  The mark is a location in the sequence, and it's used for a number of copy and swap operations discussed below.

\bump

\item {\bf Local}\quad{\it Submenu for dealing with a given track in a given group}
\begin{itemize}
\item {\bf Unroll}\quad Copies the current 8-, 16, or 32-beat segment of the local track to the next segment, or to all other segments (with \texttt{All}). 
\item {\bf Pattern}\quad Specifies a four-bar pattern for this track in this group.  Options are: \texttt{OOOO, O-O-, -O-O, OOO-, ---O, O--O, -OO-, OO--, --OO, OO-O, --O-, H1/8, H1/4, H1/2, ---X,} or {\tt XXXX}.  An {\tt O} means that the track will be played, and a {\tt -} means that it will be hushed.  The {\tt O} patterns are in pairs so you can swap the track being played with another track.  The patterns that start with an {\tt H} {\it hush} individual notes at random: the fraction is the probability of hushing.  this is different from the Step Sequencer, which hushes whole tracks.  An {\tt X} indicates {\it exclusive random:} if multiple tracks are designated as {\tt X} for a given bar, then exactly one of them will be played then (the others will be hushed).   {\tt ---X} is exclusive random for the fourth bar only and hushed otherwise; this makes possible random fills, and would be useful with {\tt OOO-}.  If only one track is {\tt X}, it's always played.  If two tracks are {\tt X}, they alternate.  Otherwise Gizmo tries to select a different track each time. 
\item {\bf Randomize}\quad Specifies the probability (1--100) that when you clear an already empty track, each step will instead be randomly turned on.  If \texttt{- - - -}  then the track is just left empty (the default).
\item {\bf Copy Track}\quad Copies the track specified by the {\it mark} onto the current track,  just within this group.  The mark is then cleared.
\item {\bf Swap Tracks}\quad Swaps the track specified by the {\it mark} with the current track, just within this group.  The mark is then cleared.
\end{itemize}

\item {\bf Track}\quad{\it Submenu for dealing with entire sequence-wide tracks.  Note: to change the Drum Note, Long-Press the Middle button in Edit Mode.}
\begin{itemize}
\item {\bf Velocity}\quad This lets you specify the velocity (volume) of the MIDI note, hence drum, that this track sends messages to.  This is a value 1...8.   The default value is 6.  Beware that some drum synthesizers may be configured to play the same velocity no matter what is sent to them.   If this track is the {\bf global accent track} then the velocity value is ignored (see Out MIDI below).
\item {\bf Out MIDI}\quad This specifies the MIDI Out Channel for steps in the track,    You can choose a value 1...16, or \texttt{- - - -} ({\it none}), or \texttt{DFLT}, which tells the Drum Sequencer to use the default MIDI Out Channel specified in the Options menu (see Section \ref{options}).  

The final track can also be set to be the {\bf global accent track}.  To do this, set its {\bf Out MIDI} not to a MIDI value, but to the degree to which each track's velocity increases to perform an accent. These are marked with an {\tt A} followed by the increase in value.  If any track's velocity value is pushed beyond 8, it is clamped to 8.  I regard {\tt A +2} as a good accent choice.

\item {\bf Copy Whole Track}\quad This copies the entire track (all groups) specified by the {\it mark} onto the current track, including the track note, velocity, and channel.  The mark is then cleared.
\item {\bf Swap Whole Tracks}\quad This swaps the entire track (all groups) specified by the {\it mark} with the current track, including the track note, velocity, and channel.  The mark is then cleared.
\item {\bf Distribute Track Info}\quad Copies the MIDI channel, and velocity (but not MIDI note) from the current track to all other tracks.  
\item {\bf Accent Track}\quad \label{accenttrack}A convenience method for making per-drum accent tracks.  This copies the MIDI note, velocity, and channel from the current track to the very next track (if there is a next track).  It then changes the next track's velocity to the nearest velocity that is approximately 60\% louder (increasing the velocity by 2)  Not to be confused with the {\bf global accent track} discussed earlier.
\item {\bf Default Velocity}\quad Sets the default initial velocity for all tracks in new sequences.
\item {\bf Linear} {\it or} {\bf Exponential Curve}\quad Sets the global velocity curve used for all tracks in sequences.
\end{itemize}


\bump

\item {\bf Group}\quad{\it Submenu for dealing with entire groups} 

\begin{itemize}
\item {\bf Unroll}\quad Copies the current 8-, 16, or 32-beat segment of the entire group to the next segment, or to all other segments (with \texttt{All}). 
\item {\bf Length}\quad Sets the length of this group.  This can be made shorter but not longer than the standard group length specified when you first formatted the sequence. 
\item {\bf Speed Multiplier}\quad Changes the speed of the group relative to the tempo.   The speed can be set to \(1\times\) (default), \(2\times\), \(4\times\), or \(1/2\times\) the current  {\bf note speed} (see the {\bf Options}).  

Unfortunately some of these speeds may be {\bf invalid} depending on the current note speed type due to limitations of the MIDI clock.  If a multiplier is invalid, you can still select it but the {\bf stopped} light will blink when playing (and it'll sound wrong). Critically, note that a 16th note can be doubled in speed but not quadrupled, which is a shame.\footnote{The technical reason for this: Gizmo's pulses are based on MIDI clock.  A 16th note is 6 MIDI clock beats.  You can't divide 6 by 4. }  You might try changing to an eighth note and doubling the tempo to permit \(4\times\). Here's a table of valid note speed multipliers:

\vspace{0.25em}
\newcommand\cm{\checkmark}
{\small
\begin{tabular}{@{}rrr@{\ }r@{}}
&\multicolumn{3}{@{}c@{}}{\it Speed Multiplier}\\
{\it Current Note Speed}&\(2\times\)&\(4\times\)&\(1/2\times\)\\
\hline
Eighth Triplet (Triplet 64th Note)	&	&	&\cm	\\
Quarter Triplet (Triplet 32nd Note)&\cm	&	&\cm	\\
Thirty-Second Note&		&	&	\cm\\
Half Triplet (Triplet 16th Note)&\cm	&\cm	&	\cm\\
Sixteenth Note&\cm	&	&	\cm\\
Triplet&\cm	&\cm	&\cm	\\
Eighth Note&\cm	&\cm&\cm	\\
Quarter Note Triplet&\cm	&\cm	&	\cm\\
Dotted Eighth Note&\cm	&	&	\cm\\
\end{tabular}
\hfill
\begin{tabular}{@{}rrr@{\ }r@{}}
&\multicolumn{3}{@{}c@{}}{\it Speed Multiplier}\\
{\it Current Note Speed}&\(2\times\)&\(4\times\)&\(1/2\times\)\\
\hline
Quarter Note&\cm	&\cm	&\cm	\\
Half Note Triplet&\cm	&\cm	&	\cm\\
Dotted Quarter Note&\cm	&\cm	&\cm	\\
Half Note&\cm	&\cm	&	\cm\\
Whole Note Triplet&\cm	&\cm	&	\cm\\
Dotted Half Note&\cm	&\cm	&\cm	\\
Whole Note&\cm	&\cm	&\cm	\\
Dotted Whole Note&	\cm	&\cm	&	\\
Double Whole Note&\cm		&\cm	&	\\
\end{tabular}
}
\vspace{0.25em}

%	Eighth Triplet (Triplet 64th Note)	&1\\
%	Quarter Triplet (Triplet 32nd Note)	&2\\
%	Thirty-Second Note				&3\\
%	Half Triplet (Triplet 16th Note)		&4\\
%	Sixteenth Note					&6\\
%	Triplet						&8\\
%	Eighth Note					&12\\
%	Quarter Note Triplet				&16\\			% add
%	Dotted Eighth Note				&18\\			% do we include this?
%	Quarter Note					&24\\
%	Half Note Triplet				&32\\			% Was: Quarter Note Tied to Triplet.   Delete?
%	Dotted Quarter Note				&36\\
%	Half Note						&48\\
%	Whole Note Triplet				&64\\			% Was: Half Note Tied to Two Triplets.  Delete?
%	Dotted Half Note				&72\\
%	Whole Note					&96\\
%	Dotted Whole Note				&144\\
%	Double Whole Note				&192\\



\item {\bf Stamp Group}\quad Copies the current group to the next one, then switches to that next group.    The mark is then cleared.  If there is no next group (if we're at the last group), then nothing is done and \texttt{CANT} is displayed.
\item {\bf Copy Group}\quad Copies the  {\it mark} group to the current group.  The mark is then cleared.
\item {\bf Swap Groups}\quad Swaps the {\it mark} group with the current group.  The mark is then cleared.
\end{itemize}

\item {\bf Chain}\quad{\it Submenu for editing the chain of transitions}
\end{itemize}

% this hack, including the weird stuff in the wrapfigure, is meant to indent the entire paragraph.

\setlength{\leftskip}{2.5em}

\begin{wrapfigure}{r}{3in}
\vspace{-1em}
\hspace{-2.5em}\includegraphics[width=3in]{SelectTransition}
\hspace{-2.5em}\caption{\small Selecting a Transition.~~~~~~~~~~~~~~~~~~~~}
\label{selecttransition}
\end{wrapfigure}

\noindent This submenu works differently from the others.  

When you select {\bf Chain}, you're first asked to select a {\bf transition} in the chain.  
Transitions are by default displayed as shown in Figure~\ref{selecttransition}: as you scroll through the transitions (left knob) the current transition is displayed at bottom.  Additionally, the transition's group is displayed both as a number at left and as a range, and the transition's repeats is shown at right (\(\infty\) for \texttt{LOOP}, and {\tt BL} for \texttt{BIG LOOP}), and as a range.  Random group transitions are shown as {\tt R}{\it num}~{\it repeats}, where {\it num}, the number of groups involved, is 2, 3, or 4, and {\it repeats} is 1, 2, 3, 4, or \(\infty\). If the group is an end group, the screen will show {\tt END}.   

\bump

\begin{wrapfigure}{r}{2.2in}
\hspace{-2.5em}\includegraphics[width=2.2in]{GroupRepeats.pdf}
\hspace{-2.5em}\caption{\small Groups and Repeats~~~~~~~~~~~~~~~~~~~~}
\label{grouprepeats}
\end{wrapfigure}

Don't remember which group is which?  Long-pressing the Middle Button will toggle toggle viewing the current screenshot of the group in the transition.    The group number and repeats will still appear as ranges, as shown in Figure~\ref{grouprepeats}.

You can use the transition chain display to just look through the transitions, then press the Back button to go back; or you can {\bf select} a transition to perform an {\bf operation} on it, one of:

 \begin{itemize}
\item[]~\vspace{-2em}
\begin{itemize}
	\item {\bf Edit}\quad Edits the selected transition.  You will be first asked to pick a group for the transition (you'll see a \texttt{G}), or \texttt{DFLT} to select the current group.  You can also select \texttt{- - - -} to indicate extra options.  If you selected a group or \texttt{DFLT}, you will then be asked to specify the number of repeated iterations for the group (you'll see an \texttt{R}).  Your choices are \texttt{LOOP}, 1, 2, 3, 4, 5, 6, 7, 8, 9, 12, 16, 24, 32, 64, or \texttt{BIG LOOP}.  If you instead selected \texttt{- - - -}, you'll be then asked to select among extra options.  These options are: \texttt{END}, R2-1, R2-2, R2-3, R2-4, R2-L, R3-1, R3-2, R3-3, R3-4, R3-L, R4-1, R4-2, R4-3, R4-4, and R4-L.  The various {\it R-} options indicate loops with random group selection. They are:\\
	\vspace{1em}

\end{itemize}


\setlength{\leftskip}{2.3em}
\setlength{\rightskip}{-16.9em}

%\begin{center}
\hspace{6.5em}\begin{tabular}{r@{~~~}r|rrrrr}
		&&\multicolumn{5}{c}{\it Number of Repeats}\\
				&&1&2&3&4&LOOP\\
				\hline
		\it Randomly select&1, 2&R2-1&R2-2&R2-3&R2-4&R2-L\\
		 \it from among &1, 2, 3&R3-1&R3-2&R3-3&R3-4&R3-L\\
		\it these groups&1, 2, 3, 4&R4-1&R4-2&R4-3&R4-4&R4-L\\
\end{tabular}
\vspace{1em}
%\end{center}

You can cancel at any time, and nothing will be changed.  Note that if your initialized format has fewer groups than permitted by the random group selection you have chosen, then (of course) random group selection will select from among just your available groups.

The default pattern of groups and repeats is (\(N\) is the maximum group):

\hspace{8em}\begin{tabular}{r@{~~~}|rrrrrrr}
		&\multicolumn{7}{c}{\it Transition}\\
				&1&2& ... & \(N\) & \(N+1\)& ... &20\\
				\hline
		 \it Group &1&2&...&\(N\)&End& ... &End\\
		\it Repeats&Loop&1&...&Big Loop&&&\\
\end{tabular}
\vspace{1em}

The idea is that this transition pattern loops on group 1 forever by default, but if you change its repeats to 1, then it'll loop on all the groups in order.

Again: don't remember which group is which?  If you long-press the Middle Button, instead of being displayed group numbers, you can toggle viewing the current screenshot of the group (except for \texttt{DFLT} and \texttt{- - - -}).  The group number will appear as a range.  See Figure~\ref{grouprepeats}.

\begin{itemize}
	\item {\bf Mark}\quad Sets the selected transition to be the {\bf mark transition} (similar to but not to be confused with the mark discussed earlier), to be used in the copy, swap, and move commands below.
	\item {\bf Add}\quad Inserts a new transition just before the selected transition, then edits the new transition as with {\bf Edit}.  The highest transition (\#20) is eliminated to make room.    The mark transition is then cleared. You can back out at any time, and nothing will be changed, including the insertion.
	\item {\bf Delete}\quad Deletes the selected transition, shifting all higher transitions down one slot.  The empty slot now at the topmost transition (\#20) is set to \texttt{END}.  The mark transition is then cleared.
	\item {\bf Copy}\quad Copies the mark transition, inserting it before the selected transition, and shifting all higher transitions up one slot to make room.  The topmost transition  (\#20) is eliminated to make room.  The mark transition is then cleared.  If the mark had not yet been set, you'll be told {\tt CANT}.
	\item {\bf Swap}\quad Swaps the mark transition with the selected transition.  The mark transition is then cleared.   If the mark had not yet been set, you'll be told {\tt CANT}.
	\item {\bf Move}\quad Moves the mark transition from its current location to just before the current transition.\footnote{To move the mark transition to after the final transition (\#20), you could move it to just before the final transition, then swap it.} The mark transition is then cleared. If the mark had not yet been set, you'll be told {\tt CANT}.
	\item {\bf Go}\quad Jumps to the group associated with the selected transition.
	\end{itemize}

%\item {\bf Clock Control} (or {\bf No Clock Control}) \quad This toggles {\it clock control mode}.  Normally, if you play or stop the sequencer, it will not send MIDI STOP or MIDI PLAY commands\,---\,if you want to start the MIDI clock you have to do so separately (see Section \ref{startingclock}).  However it's common for sequencers to issue these commands when you start or stop the sequencer, so as to also start or stop other sequencers listening over MIDI.  To do this, turn on clock control mode.  Note that this only has an effect if Gizmo is {\bf generating} a MIDI Clock (see ``MIDI Clock'' in Section \ref{options}).  Finally note that regardless of the setting of this options, if Gizmo is {\it using} an external MIDI Clock (again, see ``MIDI Clock'' in Section \ref{options}), the Step Sequencer will stop and start in response to incoming MIDI Clock commands.

\setlength{\leftskip}{0em}
\item {\bf Performance}\quad{\it Submenu for options that affect Performance Mode beyond the Chain options} 
\begin{itemize}
\item {\bf Keyboard}\quad  This menu option lets you specify what happens when you press keys or send other MIDI information during Performance Mode.  Your options are:

 \renewcommand\labelitemiii{$\diamond$}
\begin{itemize}
	\item {\texttt{-~-~-~-}}\quad Played notes are entered as if in Play Position Mode.
	\item {0} \quad Play along with the sequencer: MIDI data is routed out the default MIDI Out channel.
	\item {1--16} \quad Play along with the sequencer: MIDI data is routed out the channel specified.
	\item {\texttt{PICK}}\quad Pressing various notes on the keyboard, starting at Middle C, will cause Gizmo to schedule a jump to that transition at the end of the bar.  See Figure~\ref{transitionkeys} on page~\pageref*{transitionkeys}.
\end{itemize}

\item {\bf Repeat Sequence}\quad This menu option specifies how many times the sequence as a whole should play before it stops.  The default is {\tt LOOP} (forever), but you also have 1, 2, 3, 4, 5, 6, 7, 8, 9, 10, 12, 16, 24, 32, or 64 repeats as options.
\item {\bf Next Sequence}\quad This menu option specifies {\it which} sequence the sequencer should switch to after completing the given sequence including all its repeats.  Your options are {\tt - - - -} (stop), or 0...9.
\item {\bf Stop}\quad This toggles how the Drum Sequencer stops in performance mode.  If you press the Select button to stop the Drum Sequencer, it will either stop at the end of the current iteration ({\bf \(\bm \uparrow\) Stop At End}) or it will stop on the first note of the {\it next} iteration ({\bf \(\bm \downarrow\) Stop At Beginning}).
\item {\bf Do Fill} or {\bf Do Mute}\quad This toggles whether the Drum Sequencer will schedule a fill or a mute when the Middle Button is pressed during Performance mode.  The default is Do Fill.  This is a Drum Sequencer wide option, rather than being stored with your sequence.
\item {\bf Fill Group}\quad This specifies which group will be used for fills.  The default is 1, but you probably will want to change that.
\item {\bf Do Sequence} or {\bf Do Transition}\quad This toggles whether the Drum Sequencer will schedule the next sequence or the next transition when the Middle\(+\)Select buttons are long-pressed during Performance mode.  The default is Next Transition.  This is a Drum Sequencer wide option, rather than being stored with your sequence.
\end{itemize} 

\setlength{\leftskip}{0em}

\item {\bf Save} \quad This saves the sequence. You then choose the slot to save in.  Slots are displayed by number, plus an \texttt{S} (for Save), plus optionally an \texttt{R} or \texttt{S} or \texttt{D} (for Recorder or Step Sequencer or Drum Sequencer) if the slot is already used by that application.  This letter {\bf blinks} if the file isn't the type of your current application to warn you that you will be overwriting the other application's slot.

\item {\bf Center} \quad This lets you choose a different C, other than middle C, as the base note for keyboard control of the Drum Sequencer.  If you select a note other than C, you'll be told {\tt CANT}.

\item {\bf Options} \quad This brings up the Options menu, which enables you to do...
\end{itemize}


\setlength{\leftskip}{0em}

\paragraph{More Drum Sequencer Options}

Beyond the settings above, the step sequencer responds to a number of options in the Options Menu.  See Section \ref{options} for detailed information on each of the options.  The Drum Sequencer responds to the following options:


\begin{center}
\begin{tabular}{lllllllllll}
Tempo& Note Speed& Swing & Click&
In MIDI& Out MIDI&Control MIDI&MIDI Clock&Volume\\
\end{tabular}
\end{center}

Of particular use are Tempo, Note Speed, Swing, and especially Click.  The Drum Sequencer also responds to the {\it Transpose} option (because all applications do) but this is a nuisance as it changes the drum notes.  I strongly suggest you set Transpose to 0 in Options when using the Drum Sequencer.


\paragraph{Controls Summary}  The Drum Sequencer's controls can be a bit complex.  Here's what they do in each mode.  Note that once you're changing the tempo, you can also tap the middle button to adjust the tempo in addition to moving the knob.

\vspace{1em}\hspace{-6em}
\noindent {\small
\hspace{-2em}\begin{tabular}{@{}rllll@{}}
\it Control					& \it Edit Mode	& \it Play Position Mode	& \it Group Mode & \it Performance Mode\\
\hline\\[-0.5em]
Back Button				& Exit  & Exit & Exit &Leave Performance Mode\\
Back Button (Long)			& Toggle Bypass & Toggle Bypass &Toggle Bypass & Toggle Bypass\\
Middle Button				& Toggle Step & Toggle Mute Track & Next Group &Schedule Fill or Mute\\
Middle Button (Long)			& Set Drum Note & Clear/Randomize Group Track & Clear Group &Schedule Solo\\
Select Button				& Start/Stop Group & Start/Stop Group & Start/Stop Group &Start/Schedule Stop/Stop\\
Select Button (Long)			& Menu & Menu & Menu &Menu\\
Select + Middle Button (Long) 	& Performance Mode &  Performance Mode &  Performance Mode &Schedule Transition or Sequence\\
Left Knob					& Change Track & Change Track & Change Group &Change Track\\
						&  &  & {\it (Back/Select when Done)} &\\
Right Knob				& Change Step {\it or} &  Edit Mode &  Edit Mode&Change Tempo\\
						& Play Position Mode && &{\it (Back/Select when Done)}\\
						& {\it or} Group Mode && &\\
\end{tabular}
}
\vspace{-1em}

\paragraph{Remote Control over CC}  Like the Step Sequencer, the Drum Sequencer responds to a variety of CC messages over Gizmo's Control MIDI Channel.  These operations are on top of the CC Control MIDI operations (Section \ref{options}, see ``Control MIDI''), namely start/stop/continue, back button, middle button, select button, bypass, and left/right knob.

Note that the Set... CCs transfer to the relevant menu, then act as proxies for the left knob.  This means that once you've transferred to a menu, {\it any} of them can be used to set the menu, along with the left knob CC.  This can be confusing.

One option is only available over CC: you can't do it directly with buttons or knobs on the Arduino:

\begin{itemize}
\item {\bf Previous Group}\quad This is just like Next Group, except that it moves backwards through the groups rather than forwards.
\end{itemize}

Also note that {\bf Save} (CC 25) and {\bf Schedule Fill} (CC 22) are in unusual positions.


\vspace{1em}
\noindent {\small
\hspace{-3em}\begin{tabular}{@{}rll@{}l@{}}
\it CC Number & \it CC Value	& \it Operation\\
\hline\\[-0.5em]
64 ... 83	&	Any	& Go to Track \(\text{\it CC} - 63\), that is, track 1 ... 20\\
84	&	Any		& Set Drum Note\\
85	&	Any		& Previous Group&{\it Only available via CC}\\
86	&	Any		& Next Group\\
87	&	Any		& Toggle (or Schedule) Mute&{\it Only separate from Fill (22) via CC}\\
88	&	Any		& Toggle (or Schedule) Solo\\
89	&	Any		& Clear Group\\
90	&	Any		& Clear Track in Group\\
91	&	Any		& Schedule Transition&{\it Only separate from Next Sequence (120) via CC}\\
92	&	Any		& Toggle Click Track\\
93	&	Any		& Set Mark\\
94	&	Any		& Copy Track (Local)\\
95	&	Any		& Swap Tracks (Local)\\
112	&	Any		& Copy Whole Track\\
113	&	Any		& Swap Whole Tracks\\
114	&	Any		& Copy Group\\
115	&	Any		& Swap Groups\\
116	&	Any		& Stamp Group\\
117	&	Any		& Accent Track\\
118	&	Any		& Distribute Track Info\\
119	&	Any		& Pause\\
120	&	Any		& Schedule Next Sequence&{\it Only separate from Next Transition (91) via CC}\\
16	&	0--127		& Set Group Speed Multiplier (Select to set, Back to cancel)\\
17	&	0--127		& Set Track MIDI Out (Select to set, Back to cancel)\\
18	&	0--127		& Set Track Velocity (Select to set, Back to cancel)\\
19	&	0--127		& Set Group Length (Select to set, Back to cancel)\\
20	&	0--127		& Set Track Pattern (Select to set, Back to cancel)\\
21	&	0--127		& Set Gizmo's Tempo (Select to set, Back to cancel)\\
22	&	Any		& Schedule Fill&{\it Only separate from Mute (87) via CC}\\
23	&	0--127		& Set Gizmo's Volume (Select to set, Back to cancel)\\
24	&	0--127		& Set Gizmo's Note Speed (Select to set, Back to cancel)\\
25	&	Any		& Save (Select to set, Back to cancel)\\
26	&	0--127		& Set Gizmo's Swing (Select to set, Back to cancel)\\
27	&	0--127		& Set Performance Keyboard Mode (Select to set, Back to cancel)\\
28	&	0--127		& Set Performance Repeat (Select to set, Back to cancel)\\
29	&	0--127		& Set Performance Next Sequence (Select to set, Back to cancel)\\
30	&	0--127		& Schedule Transition (Select to set, Back to cancel)\\
31	&	0--127		& Change Group (Select to set, Back to cancel)\\
\end{tabular}
}

\clearpage

\section {The Note Recorder}

\begin{wrapfigure}{r}{3in}
\vspace{-1.5em}\includegraphics[width=3in]{recorder.pdf}
\vspace{-2em}\caption{\small Recorder Display}\vspace{-1em}
\label{recorder}
\end{wrapfigure}

The Note Recorder lets you record and play back approximately 64 notes\footnote{Why {\it approximately} 64? Because the recorder is recording MIDI NOTE ON and NOTE OFF messages.  If you provided 64 pairs of these messages, it'd take up exactly all the available memory.  But you could plausibly send a few more NOTE ON than NOTE OFF messages, in which case at the end of the recording all the remaining outstanding notes will be terminated.  This would allow you to squeeze in a few (perhaps 16) more notes in rare circumstances.  But I'd not rely on it regularly.  BTW, if you looked carefully you would have noted that the Note Display in Figure~\ref{recorder} has space for 16 more notes too.} spread over 21 measures.\footnote{Why 21?  Because notes are stored with a timestamp of eleven bits, resulting in \(2^{11} = 2048\) timesteps.  A single timestep is 1/24 of a quarter note (the resolution of MIDI Clock), and a quarter note is 1/4 of a measure, so we have \(2048 / 4 / 24 = 21 \frac{1}{3}\) measures.  We round that down to 21.}  You can save up to two recordings on an Arduino Uno and nine recordings on a Mega.

The Note Recorder screen has two major parts.  The {\it Note Display} section indicates the current note number being played (and hints at the remaining notes available).  The {\it Measure Display} section indicates the current measure being played (and again hints at the remaining measures available).  When recording a song, each of these sections will contain a single lit LED indicating the current note or measure position.  When playing a song, the Note Display and Measure Display will each also contain an additional lit LED indicating the position of the end of the song.

\begin{wrapfigure}{r}{1in}
\includegraphics[width=1in]{none.pdf}
\vspace{-2em}\caption{\small None}\vspace{-1em}
\label{none}
\end{wrapfigure}

When you choose the Note Recorder from the root menu, you will be asked to load a recording from a slot.  You can do this, or you can create a new empty recording by choosing \texttt{-~-~-~-} (meaning {\it none}).  Slots are displayed by number, plus an \texttt{L} (for Load) and an optional \texttt{R} (for Recorder), \texttt{S} (for Step Sequencer), or \texttt{D} (for Drum Sequencer) if the slot already has a file from one of these applications.  This letter {\bf blinks} if the file isn't the type of your current application to warn you that you might eventually overwrite a slot when you save.

\begin{wrapfigure}{r}{1.5in}
\includegraphics[width=1.5in]{recorderstatus.pdf}
\vspace{-2em}\caption{\small Recorder Status LEDs}\vspace{-4em}
\label{recorderstatus}
\end{wrapfigure}

After this you will enter the Note Recorder.  The Note Recorder is always in one of four states, indicated the patterns at right:

\begin{itemize}
\item Stopped
\item Playing
\item Counting off beats prior to Recording
\item Recording
\end{itemize}

When you enter the Note Recorder, the state is initially Stopped.  To start recording a song, Long Press the middle button.  The recorder will count off four beats for you, then start recording.  To start playing the song you recorded, press the middle button.  To stop playing (or stop recording), you also press the middle button.

If you are recording, you can alternatively press the Select button.  This will schedule the recording to end at the next measure boundary, then immediately start playing the recording.  If you press the Select button why playing, it'll likewise end playing at the next measure boundary and immediately start playing the recording again.  If you press the Select button after you've already pressed it, but before the measure boundary has been reached, cancels the previous Select.

The {\bf Menu} can be entered by Long-Pressing the Select button.  Here you have the option to {\bf Save} your recording.  Gizmo will ask you to select a slot to save in, then it will exit the Recorder.  Slots are displayed by number, plus an \texttt{S} (for Save), plus optionally an \texttt{R} or \texttt{S} or \text{D} (for Recorder or Step Sequencer or Drum Sequencer) if the slot is already used by that application.  This letter {\bf blinks} if the file isn't the type of your current application to warn you that you will be overwriting the other application's slot.

\paragraph{Performance Options}  
When the Note Recorder is playing, you have many of the same options as Performance Mode in the Step Sequencer and Drum Sequencer.  These options are also available in the Menu.  First, you can indicate how many iterations the Note Recorder will {\bf Repeat} playing the recording: one of {\tt LOOP} (forever), or 1, 2, 3, 4, 5, 6, 8, 9, 12, 16, 18, 24, 32, 64, or 128 times.  Second, you can indicate what the Note Recorder will do {\bf Next} after finishing its iterations: it can either {\tt END} or load another saved Recording.  Third, while the Note Recorder is playing, you can play along on the {\bf Keyboard}.  This will either play out the same MIDI channel (\texttt{- - - -}), or out a MIDI channel of your choosing.

\paragraph{More Recorder Options}  
You can also choose the Options Menu from within the Menu.  The Recorder responds to a number of options:

\vspace{1em}
\begin{tabular}{llll}
Tempo&Transpose&Volume&Click\\
In MIDI& Out MIDI&Control MIDI&MIDI Clock\\
\end{tabular}

\vspace{1em}
Of particular use to you will be Tempo and Click.  Click gives you an audible click track: and the count-in is also clicked for you.  This is much easier to play along with than the LED lights.

\paragraph{Recording and Quantization}

The Recorder records nothing but MIDI NOTE ON and NOTE OFF messages.  It presently quantizes these messages, rounding them up or down to the nearest MIDI Clock pulse, that is, 1/24 of a quarter note.\footnote{This is pretty low-resolution, but there you have it.  Space is tight.}

\paragraph{Controls Summary}  Here's what the recorder's controls do in different contexts.


\begin{center}
\vspace{1em}
\noindent {\small
\begin{tabular}{@{}rll@{}}
\it Control					& \it Action\\
\hline\\[-0.5em]
Back Button				& Exit\\
Back Button (Long)			& Toggle Bypass \\
Middle Button				& Stop / Play \\
Middle Button (Long)			& Record\\
Select Button				& Stop and Schedule Play \\
Select Button (Long)			& Menu\\
Select + Middle Button (Long) 	& - \\
Left Knob					& - \\
Right Knob				& - \\
\end{tabular}
}
\end{center}
~		% Give me a bit more space at the end of the page



\clearpage

\vspace{2em}
\begin{figure}[h!]
\begin{center}
\includegraphics[width=3in]{GeneralGauge.pdf}~\includegraphics[width=3in]{NoteGauge.pdf}
\caption{\small (Left) General Gauge Display, and (Right) Note Gauge Display}
\label{gaugedisplays} \end{center}
\vspace{-3em}\end{figure}


\section {The MIDI Gauge}


\begin{wrapfigure}{r}{2in}
\vspace{-1.5em}\includegraphics[width=2in]{midichannel.pdf}
\vspace{-2em}\caption{\small MIDI Channel Values}
\label{midichannelvalues}
\end{wrapfigure}


The MIDI gauge listens to incoming MIDI messages and displays them on-screen.  This only occurs if the MIDI IN channel is something other than NO CHANNEL (See {\bf Options\(\boldsymbol\rightarrow\) In MIDI}, in Section \ref{options}).



There are three kinds of MIDI messages for purposes of display.  {\it General} messages take up the whole screen with a single message.  {\it Note} messages split the screen between pitch and velocity of the given note (on, off, polyphonic aftertouch).  Finally {\it Frequent} messages are simply displayed by lighting a specific LED to indicate their presence. 


Messages that are associated with a MIDI channel will have their channel number displayed at bottom left (see Figure~\ref{gaugedisplays}~[Left]).  Note that if the MIDI IN channel is something other than OMNI (ALL Channels), then only messages associated with the MIDI IN channel will be displayed.






\paragraph{General Gauge Display}  These are simple displays which take up the entire screen.

\begin{itemize}
\item {\bf Pitch Bend}\quad The 14-bit value is displayed
\item {\bf Channel Aftertouch}.  \texttt{AT} is displayed, followed by the value.
\item {\bf Program Change}\quad \texttt{PC} is displayed, followed by the value.
\item {\bf Control Change (CC)}\quad If the CC parameter is for 7-bit values, then the value is shown first, then text is scrolled with the value, then \texttt{CC}, then the parameter.  If the CC parameter is for 14-bit values, the MSB of the value is shown first, then text is scrolled with the value, then \texttt{CC}, then the parameter, then in parentheses the 14-bit (MSB+LSB) value.
\item {\bf Registered Parameter Number (RPN)}\quad The value (7-bit MSB only) is shown first, then text is scrolled with the value (MSB), then \texttt{RPN}, then the parameter (14-bit), then in parentheses the 14-bit (MSB+LSB) value.   {\bf Note:} RPN NULL is never displayed\,---\,it obscures NRPN values on the gauge and so is not helpful.
\item {\bf Non-Registered Parameter Number (NRPN)}\quad The value (7-bit MSB only) is shown first, then text is scrolled with the value (MSB), then \texttt{NRPN}, then the parameter (14-bit), then in parentheses the 14-bit (MSB+LSB) value.
\item {\bf Channel Mode}\quad The channel mode parameter and value are shown as a text message (such as \texttt{RESET ALL CONTROLLERS}).  In the case of \texttt{MONO ON} the channel count is shown first.  The Channel Mode parameters are:
\begin{center}\begin{tabular}{@{}rlll@{}}
{\it Parameter}&{\it Value}&{\it Meaning}\\[0.1em]
\hline\\[-0.9em]
120&Normally 0&All Sound Off\\
121&Normally 0&Reset All Controllers\\
122&0=Off, 1=On&Local Control Off ({\it or} On)\\
123&Normally 0&All Notes Off\\
124&Normally 0&Omni Off\\
125&Normally 0&Omni On\\
126&\(C\)&Mono On ({\it with \(C\) channels, shown first})\\
127&Normally 0&Poly On\\
\end{tabular}\end{center}
\item {\bf System Exclusive}\quad is {\it not} indicated: Gizmo does not detect it.
%\texttt{SYSX} is displayed.\footnote{Note that the version of the MIDI library on which Gizmo relies presently doesn't handle large System Exclusive messages well.}
\item {\bf Song Position}\quad \texttt{SPOS} is displayed.
\item {\bf Song Select}\quad \texttt{SSEL} is displayed.
\item {\bf Tune Request}\quad \texttt{TREQ} is displayed.
\item {\bf Start}\quad \texttt{STRT} is displayed.
\item {\bf Continue}\quad \texttt{CONT} is displayed.
\item {\bf Stop}\quad \texttt{STOP} is displayed.
\item {\bf System Reset}\quad \texttt{RSET} is displayed.
\end{itemize}


\begin{wrapfigure}{r}{1.6in}
\vspace{-37em}\includegraphics[width=1.6in]{notefunction.pdf}
\vspace{-2em}\caption{\small Note Function}
\label{notefunction}
\end{wrapfigure}


Note messages are displayed with both the {\bf pitch} (at left) and the {\bf velocity} (at right) of the note value, plus the particular kind of message in question: {\bf Note On, Note Off,} and {\bf Polyphonic Aftertouch}.

\begin{wrapfigure}{r}{1.8in}
\vspace{-32em}
\includegraphics[width=1.8in]{octave.pdf}
\vspace{-2em}\caption{\small Octave Values.  Three common octave numbering schemes shown: Yamaha, SPN (so-called {\it Scientific Pitch Notation}), and MIDI.  SPN is the most common on keyboards.}\vspace{-5em}
\label{octave}
\end{wrapfigure}



\paragraph{Note Display}


How exactly do you distinguish between a Note On, Note Off, or Polyphonic Aftertouch message then?  Figure~{\ref{gaugedisplays}}~[Right] shows a four-LED region, in orange, where a certain pattern is displayed to indicate this.  The patterns are shown in Figure~\ref{notefunction}.

Note pitches are displayed by showing the note (such as B$\flat$), with the octave value directly below it.  MIDI has eleven different octaves (0--10): Middle C is the bottom note of octave 5.  The octave number is displayed as shown in Figure~\ref{octave}.  As usual, the MIDI In Channel is shown below that.


\paragraph{Frequent MIDI Display}  These are messages which are too fast to display usefully in most cases.  In each case, a specific LED will be toggled (see Figure~\ref{gaugedisplays}).  The messages are:

\paragraph{}\vspace{-2em}\begin{itemize}
\item {\bf Active Sensing}
\item {\bf MIDI Time Code}
\item {\bf MIDI Clock}
\end{itemize}


\clearpage
\section {The Controller}
\label{controller}

The Controller allows you to assign to each of two buttons and four potentiometers any NRPN, RPN, Control Change (CC), Program Change (PC) value, Pitch Bend, or Channel Aftertouch.%, or control voltage.  
These assignments are stored in Flash memory and survive reboots and power cycling.  You can also set up an eight-stage loopable envelope to do complex control changes, such as emulating the Waldorf Microwave XT's ``wave envelope'', or doing unusual LFOs.  Finally, the Controller sports a sophisticated Random LFO to do control changes.

The Controller interacts with Bypass mode differently than other applications.  In bypass mode, the Controller passes all the incoming MIDI data to MIDI out as usual.  But it will still emit values when you press the buttons or more the knobs.  This is because it's typical that you'd want to use Controller to (for example) manipulate the VCF on an analog synthesizer while playing it via a remote keyboard.  To do this, you'd plug the keyboard into Gizmo, plug Gizmo into the synthesizer, and turn on bypass mode.

%Optional control voltage is sent out one of two Digital-to-Analog Converters or DACs you may have attached, which it refers to as DAC A (I2C address 0x62) or DAC B (I2C address 0x63).\footnote{Gizmo assumes you're DACs of this kind: https:/\!/www.adafruit.com/product/935\quad These DACs can be connected via I2C, send 0--5V, and can be set to either I2C address 0x62 or 0x63.  If you need more than 5V, you'll need to wire up your own op-amp.}

	\begin{description}

	\item{\bf Go}
	
	This enters the controller proper.  Turning the left or right knobs, or pressing the Select or Middle buttons, will cause them to issue MIDI messages as assigned (below).  If you press the Back button, you can exit the controller.
	
	When you turn a knob or press a button, and the knob/button has had its control type set to something other than OFF, then the appropriate MIDI message is sent, and the value of the message is displayed as a number on-screen.  Messages are only sent if the MIDI OUT channel is something other than NO CHANNEL (See {\bf Options\(\boldsymbol\rightarrow\) Out MIDI}, in Section \ref{options}).
	
	Several protocols (NRPN, RPN, some CC) can accept 14-bit numbers, that is, values in the range 0...16383.  However Gizmo's potentiometers have a maximum resolution of 10 bits, that is, 0...1023.   The potentiometers in fact send 10 bits of data: their top 7 bits are the MSB, and their bottom 3 bits become the upper 3 bits of the LSB (the bottom 4 bits are padded with zeros).   While only the MSB is displayed on-screen, this partial LSB is being sent as well to give you a bit more control.  %For control voltage, all ten bits will be mapped to 0--5V.\footnote{The DACs are actually 12-bit, but what can you do: the pots are only 10 bit.}  \ For Pitch Bend, the ten bits will be mapped to \(-8192...8191\).

	{\bf \textit{Important Note}\quad} If you have set the Left Knob to send out PC, it works differently than other protocols: changing the knob does not send the PC value.  Rather, to send the PC value currently set by the knob, you press the Middle Button.  Similarly, if you have set the Right Knob to send out PC, then the PC value will be sent when you press the Select Button.  This is because few if any synthesizers can survive a stream of constant PC values.
	
	\item{\bf Left Knob} and {\bf Right Knob}\quad (\texttt{L KNOB} and \texttt{R KNOB})
	
	These submenus let you select your control type and parameter number for the left and right potentiometers.
	
	In each, you are first given the option of what kind of {\bf control type} (OFF, CC, NRPN, RPN, PC, Pitch Bend, or Aftertouch) % A Voltage, or B Voltage) 
you would like to manipulate with the left knob.  If OFF, then turning the knob will not do anything.

	After you have selected a control type, if you have selected CC, NRPN, or RPN, you will be then asked to select a {\bf controller parameter number}.  (Otherwise, for PC and OFF, you will go back to the top-level Controller Menu).  CC parameter numbers may range 0 ... 119.  NRPN and RPN parameter numbers may range 0 ... 16383. When you have completed this, you will be sent back to the top-level Controller menu.

	\item{\bf Middle Button} and {\bf Select Button}\quad (\texttt{M BUTTON} and \texttt{R BUTTON})
	
	Thees submenus let you select your control type, parameter number, and on/off values for the middle and select buttons.
	
	In each, you are first given the option of what kind of {\bf control type} (OFF, CC, NRPN, RPN, PC, Pitch Bend, or Aftertouch) %A Voltage, or B Voltage) 
you would like to manipulate with the middle button.  If OFF, then pressing the button will not do anything.

	After you have selected a control type, if you have selected CC, NRPN, or RPN, you will be then asked to select a {\bf controller parameter number}.  (Otherwise ,for PC and OFF, you will go back to the top-level Controller Menu).  CC parameter numbers may range 0 ... 119.  NRPN and RPN parameter numbers may range 0 ... 16383.
	
	When you have completed this, you will be then asked to enter the value sent when the button is {\bf pressed}.  This value must be 0...127 (the Controller does not send 14-bit values).  Afterwards, you will be similarly asked to enter the value sent when the button is {\bf pressed again} (the buttons act as toggles).  When you have completed this, you will be sent back to the top-level Controller menu.
	
	Alternatively, instead of choosing a value 0...127 for the controller value number, you can choose to have the button {\bf increment} or {\bf decrement} its parameter when pressed.  This {\bf only makes sense if your parameter type is RPN or NRPN}, which have this facility.  Pressing the middle button decrements its parameter; pressing the select button increments its parameter.   
	
	{\bf \textit{Important Note}\quad} If you have selected PC for your Left Knob, then the M BUTTON will not operate as you have specified here: instead it will serve only to send the PC value (see {\bf Go} above).  Similarly, if you have selected PC for your Right Knob, then the R BUTTON will not operate as specified here.
	
	\item{\bf Wave}
	
	This submenu controls the 8-stage Wave Envelope (see Section \ref{waveenvelope} next).

	\item{\bf Random}
	
	This submenu controls the Random LFO (see Section \ref{randomlfo}).
	
	\item{\bf PC}
	
	This submenu allows you to load, save, and send out out bulk PC and Bank changes to many synthesizers at once (see Section \ref{bulkpc}).

	\item{\bf Analog Port 2} and {\bf Analog Port 3}\quad (\texttt{A2} and \texttt{A3})
	
	These submenus work exactly like the {\bf Left Knob} and {\bf Right Knob} menus, but they are for additional potentiometers or other devices which you can attach to ports A3 and A3, respectively, on your Arduino.  For example, I use them to get information from a joystick.\footnote{{\it Why not A4 and A5 too?\quad} Because A4 and A5 are used for I2C, which the screen uses.  You could of course add in A6...A12 too.}
	
	\end{description}
	
\paragraph{Warning} Gizmo could {\it theoretically} react to knob changes every 4 MIDI bytes read at full MIDI speed.  However an RPN/NRPN command will take up to 12 bytes (or 18 if \texttt{INCLUDE\_SEND\_NULL\_RPN} is turned on in the code), and a CC command could take up to 6 bytes.  This means that turning a knob may cause Gizmo to generate RPN/RPN/CC messages faster than it can send them.  Eventually the Arduino's write-buffer will fill with unwritten messages and the Arduino will pause until it's sent stuff out.  This could cause Gizmo to miss an incoming MIDI byte.  So if you're sending control commands while (say) at the same time passing data through, this may be an issue: though I have never had it happen to me.

\subsection{The 8-Stage Wave Envelope}
\label{waveenvelope}

The Controller also sports an {\bf 8-stage envelope} to control any single control type (CC, NRPN, RPN, PC, Pitch Bend, Channel Aftertouch)%, Voltage)
.  This envelope is modeled after the 8-stage ``Wave Envelope'' found on Waldorf Microwave synthesizers, designed to make cool modulations through a wavetable.  For some reason the Waldorf Blofeld lacks this envelope, and I wrote this for Gizmo to give the Blofeld a rough approximation.  But you may find it useful in other contexts as well!  You can also use this envelope to easily do an LFO with a triangle wave, a sawtooth wave, or a pulse wave, among others.

\begin{wrapfigure}{r}{1in}
\includegraphics[width=1in]{none.pdf}
\vspace{-2em}\caption{\small Off}\vspace{-1em}
\end{wrapfigure}

 This is not an ADSR envelope: there is no sustain and no release.  Rather, the envelope has eight {\it stages}, each with a {\it control value} and a {\it time interval}.  As a key is held down, Gizmo iterates through all of the stages one by one.  Each stage starts at the stage's value and lasts for its time interval, gradually interpolating until it reaches the value of the next stage.  Your envelope doesn't have to have all eight stages: you can cut it short by setting a stage time interval to {\bf off} (\texttt{-~-~-~-}).  This has different effects depending on whether you're looping or doing a one-shot envelope (see below).


\begin{figure}[t]
\begin{center}
\begin{tabular}{@{}ll@{}}
\includegraphics[height=1.7in]{OneShot.pdf}&
\includegraphics[height=1.7in]{OneShotShort.pdf}\\
~(a) Gated/Faded/Triggered, 8 stages&
~(b) Gated/Faded/Triggered, 5 stages\\\\
\includegraphics[height=1.7in]{Looped.pdf}&
\includegraphics[height=1.7in]{LoopedShort.pdf}\\
~(c) Looped/Free, 8 stages&
~(d) Looped/Free, 4 stages
\end{tabular}
\end{center}
\caption{Examples of the 8-Stage Envelope in Action.}
\end{figure}

\paragraph{Modes}  There are five envelope modes:

\begin{itemize}
\item {\bf Gated}\quad This is the default.  The envelope will begin at the the stage 1 value when a key is pressed, and then start sweeping.  An envelope has expired when it reaches (not completes) stage 8, or when it reaches a stage whose time interval is ``off''.  When you release all the keys, or when the envelope has expired, then the control value will stay fixed at its current value until restarted.  You restart the envelope by releasing all the keys and then pressing a new one. 
\item {\bf Faded}\quad This is the exactly same as Gated, except that the envelope also restarts each time a new note is pressed, and (only) when restarted in this different way, instead of resetting to Value 1 and moving towards Value 2 again, it will move towards Value 2 from wherever it left off.  This prevents the sudden jump to Value 1 and is useful for long legato pads.
\item {\bf Triggered}\quad This is like Gated, except that the envelope continues even if you have released all the keys.  You restart the envelope by releasing all the keys and then pressing a new key.  This might be useful for sounds with long releases.
\item {\bf Looped}\quad The envelope will begin at the the stage 1 value when a key is pressed, and then start sweeping.  An envelope has expired when it {\it completes} stage 8, or when it {\it reaches} (not completes) a stage whose time interval is ``off''.  When an envelope has expired, Gizmo loops to stage 1 again.  If you release all the keys, then the control value will stay fixed at its current value until restarted.  You restart the envelope by releasing all the keys and then pressing a new key.
\item {\bf Free}\quad This is like Looped, except that envelope is constantly running and looping at all times.  The envelope cannot be stopped or restarted.
\end{itemize}

Please note that if you're using Gated, Faded, or Triggered, then the time interval for stage 8 serves no function.  In Looped or Free, it indicates the length of stage 8 before returning to stage 1.  

\paragraph{Values}  All stage values are 0...127.  However the interpolation is 14-bit if you're doing NRPN, RPN, %, Voltage, 
or Pitch Bend.  Otherwise the interpolation is 7-bit.

\paragraph{Intervals} A stage time interval can any \(n\) from 0...254 (or {\bf off} as discussed earlier).  The envelope can be {\bf synced} to the MIDI clock (or {\bf unsynced}). When synced, a value of \(n\) is a 32nd note, that is, \(n/8\) of a beat.  So if you want your interval to last a 4-beat measure, set \(n=32\).

When unsynced (the default), \(n\) represents approximately \(n/16\) steps per second, which is roughly the same as what \(n\) is in synced mode when the tempo is 120 BPM.    At \(n=254\) (the maximum), the interval length is 15.875 seconds.

\paragraph{Display When Running}  If the envelope is expired or hasn't started yet,  \texttt{-~-~-~-} is displayed.  Otherwise it displays two numbers.  The left number is the MSB of the current control value.  The right number is the stage (1...8).

\paragraph{Menu Options}  The menu options in the {\bf Wave} submenu are:

\begin{itemize}
\item {\bf Go}\quad Start the envelope.
\item {\bf Control}\quad Specify the envelope control type.  If CC, NRPN, or RPN, additionally specify the parameter number. 
\item {\bf Envelope}\quad Set the 8 envelope values and time intervals.  Remember that an interval can also be \texttt{-~-~-~-}, which indicates that that is the final stage.  The final entry in this submenu is {\bf Clear}, which resets all the values to 0 all the intervals to \texttt{-~-~-~-}. 
\item {\bf Mode}\quad Specify the mode (Gated, Faded, Triggered, Looped, or Free).
\item {\bf Sync} {\it or} {\bf Unsync}\quad Sync the envelope to MIDI clock or unsync it.
\end{itemize}

\paragraph{Fun with LFOs}  You can use the Wave Envelope to make a variety of LFOs.  For example, you can make Sawtooth, Triangle, and various Square wave LFOs as follows.  First set the Mode to Looped or Free.  Then set the Envelope values as shown in the table below:

\begin{itemize}
\item \(x\) is the minimum value of your wave (0...127).
\item \(y\) is the maximum value of your wave (0...127, \(x \leq y\)).
\item \(n\) is the wavelength (a stage time interval), (0...254).  So for example if you want a synced LFO to repeat every measure, you could set \(n=32\).
\end{itemize} 

\begin{center}
\begin{tabular}{@{}r|l@{\hspace{1.5em}}l@{\hspace{1.5em}}l@{\hspace{1.5em}}ll@{}}
		& \it Sawtooth  	& \it Triangle 			& \it Square 	& \it Square, \(p\)\% Pulse Width		& \it Sawtooth \(\rightarrow\) Triangle \(\rightarrow\) Square!\\
Value 1	& \(x\)		& \(x\)				& \(x\)			& \(x\)								&			\(x\)				\\
Length 1	& \(n\)		& \(n/2\)			& \(n/2\)				& \(n \times p / 100\)						&			\(n\)				\\
Value 2	& \(y\)		& \(y\)			& \(x\)				& \(x\)								&			\(y\)				\\
Length 2	& 0			& \(n/2\)			& 0					& 0									&			\(0\)				\\
Value 3	& anything		& anything			&\(y\)		&\(y\)								&			\(x\)				\\
Length 3	& \texttt{-~-~-~-}& \texttt{-~-~-~-}	& \(n/2\)				& \(n \times (100 - p / 100)\)				&			\(n/2\)			\\
Value 4	&			&  				&\(y\)				&\(y\)								&			\(y\)				\\
Length 4	& 			&				& 0					& 0									&			\(n/2\)			\\
Value 5	& 			&				& anything				& anything						&			\(x\)				\\
Length 5	& 			&				& \texttt{-~-~-~-}		& \texttt{-~-~-~-}						&			\(0\)				\\
Value 6	& 			&				& 					& 									&			\(y\)				\\
Length 6	& 			&				& 					& 									&			\(n/2\)			\\
Value 7	& 			&				& 					& 									&			\(y\)				\\
Length 7	& 			&				& 					& 									&			\(0\)				\\
Value 8	& 			&				& 					& 									&			\(x\)				\\
Length 8	& 			&				& 					& 									&			\(n/2\)			\\

\end{tabular}
\end{center}


\paragraph{Hints}  This envelope isn't like the envelopes in your synthesizer: those are per-voice.  This envelope controls {\it all} the voices playing at the same time\,---\,after all, it's just manipulating a single control (perhaps a CC parameter).  So if you play a key, then release, and your sound has a long release, then when you play a new key, it'll suddenly change the control value for the old note along with the new one.  Thus this might work best with sounds with fairly short releases.

The envelope updates its interpolated values at no more than 100 times a second, so as not to overwhelm the MIDI stream.  It's possible that your device can't handle changes that fast.  You can change things with the \texttt{WAVE\_COUNTDOWN} constant found in the file \texttt{Control.h}. It's by default set to 32.  Increasing it will slow things down.

If the envelope feels stepped or discretized, changing \texttt{WAVE\_COUNTDOWN} won't help things.  Rather, it's likely the fact that you're doing low-resolution CC (7-bit).  There's little you can do about that, it's the nature of MIDI.

Last: if you're wondering why you can't select 0 as a length, this is because you're only turning the left knob.  Turn the right knob to get finer-grained choices.

\subsection{The Random LFO}
\label{randomlfo}

Gizmo's Random LFO modifies a control value over time.  The Random LFO permits both random walks and sample-and-hold style random values of different kinds.  In general, it operates by repeatedly choosing a new (random) target value, then over a specified length of time, either (random walk) gradually moving that value towards the target, or (sample and hold) going to the new value immediately.   Specifically the LFO has, count 'em, {\it eight modes}:

\begin{itemize}
\item {\bf Gated}\quad This is the default.  When a key is pressed and no other keys are down, the LFO chooses a new random start position and new random end position, and sets the control value to the start position.  Then over a specified period of time it gradually shifts the control value to the end position.  When the time is over, it picks a new random end position, then gradually shifts the control value to that one, and so on.  When the last key is released, the LFO stops, and the current control value is maintained.
\item {\bf Triggered}\quad This is like Gated, except that the LFO does not stop when the last key is released.
\item {\bf Not Reset} \footnote{Yeah, it's not a good name.}\quad This is like Gated, except that when a new key is pressed after all keys are released, the LFO doesn't choose a new random start position; it uses the existing control position.  This would be the obvious choice for situations such as the LFO controlling pitch.
\item {\bf Free}\quad  This LFO is free-running and not affected by keystrokes.  The LFO immediately chooses a new random start position and new random end position, and sets the control value to the start position.  Then over a specified period of time it gradually shifts the control value to the end position.  When the time is over, it picks a new random end position, then gradually shifts the control value to that one, and so on.
\item {\bf Sample and Hold (SH) Gated}\quad This is like Gated, except that the control value only changes when we have selected a new random end position.
\item {\bf Sample and Hold (SH) Triggered}\quad This is like Triggered, except that the control value only changes when we have selected a new random end position.  
\item {\bf Sample and Hold (SH) Not Reset}\quad This is like Not Reset, except that the control value only changes when we have selected a new random end position.
\item {\bf Sample and Hold (SH) Free}\quad  This is like Free, except that the control value only changes when we have selected a new random end position.
\end{itemize}

There are several parameters which control the randomness of the LFO:

\paragraph{Length} The length of time before selecting a new random value is handled similarly to the stage-interval time in the 8-stage wave envelope.  Specifically it can be any value \(n\) from 0...254.  However, these values are three times faster than in the 8-stage wave envelope (because fast LFOs are more useful, and long time lengths for an LFO are not).  

The envelope can be {\bf synced} to the MIDI clock (or {\bf unsynced}). When synced, a value of \(n\) is \(n/24\) of a beat.  So if you want your interval to last a 4-beat measure, set \(n=96\).  When unsynced (the default), \(n\) represents approximately \(n/48\) steps per second, which is roughly the same as what \(n\) is in synced mode when the tempo is 120 BPM.    At \(n=254\) (the maximum), the interval length is 5.2917 seconds.

Just beyond 254, you can also set \(n\) to \texttt{HIGH}.  This makes the length infinity: until it is reset, the LFO will not change its current value.


\paragraph{Range} The degree of randomness in selecting new random values.  This is a value \(n\) from \(1...127\).  When a new random end position must be chosen, we take the old end position (which will become the new start position) and add a random amount between \(-n...n\).  This attempt is repeated until we find a valid random end position value.   This this ranges from \(n=1\) very gradual random walks to \(n=127\) complete random selection.

\paragraph{Initial Value}  This is the starting point of the LFO, and is any value \(0...127\).  When the LFO is reset, we choose a new starting point by taking this value and generate a random number within the {\it range} of this value.

\paragraph{Display and Control When Running}  If the envelope is expired or hasn't started yet,  \texttt{-~-~-~-} is displayed.  Otherwise it displays two numbers.  The left number is the MSB of the current control value.  The right number the MSB of the target end position. 

You can also dynamically change the current LFO {\bf Length} with the left pot, and the LFO {\bf Range} with the right pot.  Additionally, directly beneath each displayed number is a range showing (at a resolution of 16) the current Length (left) and Range (right).  When you exit the LFO, the length and range are restored to the values to which they had originally been set (by you).

These two ranges are shown because you can control

\paragraph{Menu Options}  The menu options in the {\bf Random} submenu are:

\begin{itemize}
\item {\bf Go}\quad Enter the LFO.  
\item {\bf Control}\quad Specify the envelope control type.  If CC, NRPN, or RPN, additionally specify the parameter number. 
\item {\bf Mode}\quad Specify the LFO mode.
\item {\bf Range}\quad Specify the LFO range.
\item {\bf Initial Value}\quad Specify the LFO initial value.
\item {\bf Length}\quad Specify the LFO length.
\item {\bf Sync} {\it or} {\bf Unsync}\quad Sync the LFO to MIDI clock or unsync it.
\end{itemize}

\paragraph{Hints}
The longer the length, the less random the LFO will sound.  So if you want a blast of random noise, set the length to 0.

Just like the wave envelope, the LFO updates its interpolated values at no more than 100 times a second, so as not to overwhelm the MIDI stream.  It's possible that your device can't handle changes that fast.  You can change things with the \texttt{WAVE\_COUNTDOWN} constant found in the file \texttt{Control.h}. It's by default set to 32.  Increasing it will slow things down.  If the envelope feels stepped or discretized, changing \texttt{WAVE\_COUNTDOWN} won't help things.  Rather, it's likely the fact that you're doing low-resolution CC (7-bit).  There's little you can do about that, it's the nature of MIDI.  Try increasing your range.

Last: if you're wondering why you can't select 0 as a length, this is because you're only turning the left knob.  Turn the right knob to get finer-grained choices.

\subsection{Bulk PC and Bank Changes}
\label{bulkpc}

If you're performing, it'd be nice to change patches on several synthesizers at once, and to do so multiple times during a performance.  Gizmo can send different bulk Bank (CC \#0) and Program Changes up to twelve times in a performance.  Each time is called a {\it stage}, and a single change can send different Bank and Program changes to all 16 of your MIDI channels.  You can save this bulk change plan to any of Gizmo's slots and load it later.

\begin{wrapfigure}{r}{1in}
\vspace{-1em}
\includegraphics[width=1in]{exit.pdf}
\caption{Exit?}
\vspace{-1em}
\end{wrapfigure}

When you enter the Bulk PC/Bank change facility, the first thing you'll be asked to do is load a plan from a slot or initialize a new one with \texttt{-~-~-~-}.  After that, you're given the option to {\it perform} with the stages in the existing plan (via \texttt{GO}), to {\it edit} a stage (by choosing its number), or finally to {\it save} the edited plan.  If you try to exit, you'll be asked if you really want to quit before saving with {\texttt{EXIT?}}.

\paragraph{Editing a Stage}
If you have chosen a stage to edit, you'll be then presented with 16 MIDI channels. Each channel can be given a different PC and Bank change.  Select a channel, and you'll next be asked which Bank Change to do (0--127, or no Bank Change with \texttt{-~-~-~-}).  After setting this up, you'll next be asked which Program Change to do (0--127).   Then you'll be returned to the Bulk PC/Bank Change main menu.

\paragraph{Performing}
If you have chosen to perform the plan, you'll be then presented with 12 stages. Select a stage and Gizmo will first send the Bank Changes (or not) to all 16 channels as you specified, then wait for 1 second (while it displays \texttt{DONE}), then send the Program Changes to all 16 channels as you specified, then return to the Performance menu so you can pick another stage.

\clearpage

\section {The Keyboard Splitter}
\label{splitter}

\begin{wrapfigure}{r}{3in}
\vspace{-1.5em}\includegraphics[width=3in]{split}
\vspace{-2em}\caption{\small Splitter Layout}
\label{splitter}
\end{wrapfigure}

The Splitter allows lets you (1) split your keyboard into two zones, each controlling a different MIDI channel; or (2) split your keyboard into {\it three} zones, two controlling different MIDI channels and the third (middle) zone playing both MIDI channels simultaneously\footnote{This is commonly known as {\it layering.}}; or (3) fade or balance your entire keyboard such that if you play a note quietly, one channel will predominate, but if you play loudly, the other channel will predominate.

The two MIDI channels in question are the Default MIDI Out Channel (defined in the Options, see Section \ref{options}), and an Alternative MIDI Out Channel which you specify in the Splitter.  If you're splitting your keyboard, you'll also specify up to two notes, {\bf Split Note A} and {\bf Split Note B}  (Split Note B is optional), and whether Control Change, Aftertouch, and the like go to the left split region or the right split region.

\begin{itemize}
\item  If you want to just split your keyboard, specify only Split Note A at the split point.  Notes in the left (lower) region will go our the Alternative MIDI Channel, and notes in the right (upper) region will go out the Default MIDI Channel.

\item If you want your keyboard to play {\bf both} channels, make Split Note A be the lowest note on your keyboard, and Split Note B be the highest note on your keyboard.    Notes played on the keyboard will be sent out both the Alternative MIDI and Default MIDI Channels. 

\item If you want your keyboard to have {\bf three regions}, a left-only region, a right-only region and a middle region where both are played, make Split Note A be the lower note in the middle region and Split Note B be the higher note.   Notes in the left (lower) region will go our the Alternative MIDI Channel, notes in the right (upper) region will go out the Default MIDI Channel, and notes in the middle region will go out both channels.

\item Note that it is possible to also give you keyboard three regions where in the middle region {\bf neither} channel is played.  This is probably not very useful.  But it happens when you define Split Note A to be {\it above} (to the right of) Split Note B.

\item Finally, if you want to fade or balance your keyboard so that notes on the two channels are played with opposite velocities, you simply choose this option (see below).  You don't need to specify either Split Note A or B, but you will need to define the alternative MIDI Channel.
\end{itemize}

\begin{wrapfigure}{r}{1.5in}
\vspace{-3em}\includegraphics[width=1.5in]{Fade}
\vspace{-2em}\caption{\small Cumulative Velocity of MIDI Out when fading.  Default is {\color{red}red}; Alternative is {\color{blue} blue}.}
\vspace{-3em}
\label{splitter}
\end{wrapfigure}

\paragraph{Fade Behavior}  The current fade behavior function is:
\[
\begin{split}
\text{\it Default MIDI Velocity} &= Velocity - \text{\it Velocity} \times \left( \text{\it Velocity} + 1 \right) / 128\\
\text{\it Alternative MIDI Velocity} &= \text{\it Velocity} \times \left( \text{\it Velocity} + 1 \right) / 128
\end{split}
\]

This is a fancy way of saying that the total velocity of both channels is the same as the input velocity, and that the ratio of the Alternative velocity versus the Default velocity increases linearly as you get closer to 127.  I don't know if this is the right approach, and certainly different keyboards synthesizers have different and often nonlinear velocity mappings.  But there you have it.  If you'd like to play with it, it's in the function \texttt{handleNoteOn(...)} in \texttt{TopLevel.cpp}.

As you play notes in the fader, their velocity will appear on-screen.

\paragraph{Defining Notes and Channels}

If you press the Select Button, you'll be asked to enter a note.  This note will be Split Note A.  By default this note is Middle C.  You can back out if you don't want to enter the note.

If you press the Middle Button you'll be asked to enter a note.  This note will be Split Note B.  By default this note is off.  You can back out if you don't want to enter the note.  You can also turn Split Note B off again by once again pressing the Middle Button, at which point you'll be presented with \texttt{-~-~-~-}.  You can also back out if you don't want remove Split Note B.

\begin{wrapfigure}{r}{1in}
\vspace{-1em}\includegraphics[width=1in]{none}
\vspace{-2em}
\end{wrapfigure}

You define the Alternative MIDI Out channel by Long-Pressing the Select Button.  The Alternative MIDI Channel must be a value 1--16 (you can't choose Off).

The Splitter will display your two Split Notes, as shown in Figure \ref{splitter}.

\paragraph{Routing and Fading}

Long-pressing the Middle Button cycles through three options:
\begin{itemize}
\item Split/Layer the Keyboard, and route all CC, PC, Channel Aftertouch, and NRPN/RPN to the {\it right} (upper) split channel, that is, to the Default MIDI Out channel.  Pitch Bend is also routed to the right channel if you're doing a simple split, but if you're layering, it's routed to both channels.  Polyphonic Aftertouch is always routed to the channel handling the note in question, as are all Note On and Note Off messages.  The screen will show the Split A note, and will light the ``left'' LED (see Figure \ref{splitter})..
\item Split/Layer the Keyboard, and route all CC, PC, Channel Aftertouch, and NRPN/RPN to the {\it left} (upper) split channel, that is, to the alternative MIDI Out Channel.  Pitch Bend is also routed to the left channel if you're doing a simple split, but if you're layering, it's routed to both channels.  Polyphonic Aftertouch is always routed to the channel handling the note in question, as are all Note On and Note Off messages.  The screen will show the Split A note at left and the Split B note at right, and will light the ``left'' LED (see Figure \ref{splitter}).
\item Fade/Balance the Keyboard.  CC, PC, Channel Aftertouch, and NRPN/RPN is routed to the alternative MIDI Out Channel.  Pitch Bend, Polyphonic Aftertouch, all Note On and Note Off messages (though with inverted velocities as discussed earlier) are routed to both channels.  The screen will display \texttt{FADE}.
\end{itemize}


\clearpage
\begin{figure}[h!]
\vspace{-1em}\includegraphics[width=6.5in]{thru.pdf}
\vspace{-1.25em}
\caption{Flow graph of the Thru Facility.}
\vspace{-0.5em}
\end{figure}

\section{The Thru Facility}
\label{thru}

The {\bf Thru Facility} passes MIDI through the box, but if MIDI data comes in via certain channels, it applies various optional transformations to it before it sends it out:

\begin{itemize}
\item First, you can {\bf merge} two incoming channels: the Default MIDI In channel and a {\bf Merge Channel}.  Additionally you can also merge {\it all} channels by simply setting the Default MIDI In channel to OMNI.  All channels other than Default MIDI and Merge are passed through, unless you {\bf block} them.

\item Note data and Control data (CC/PC/RPN/NRPN/Pitch Bend) are then handled differently. We'll start with Note data.   Note data is passed through an option that allows you to {\bf debounce notes}.   The objective here is to counteract the situation where you hit a drum pad, and instead of staying down, it quickly bounces up, then down again, resulting in a buh-dum sound.\footnote{For example, I have debouncing issues with the Arturia Beatstep.}  It works like this.  When you first press a note, Gizmo will require that this note be no less than \(N\) milliseconds long.  Furthermore, if the same note is pressed within these \(N\) milliseconds, Gizmo will ignore it.     Large values of \(N\) catch more bounces but prevent you from playing the pad rapidly, and will also prevent you from making short stabs.  Thus this would be better for one-shot sounds (drums, etc.) which play in their entirety regardless of when the note-off occurs.

\item You can then {\bf distribute} incoming notes, one by one, to separate MIDI channels, rotating the distribution round-robin.  Thus if you have multiple copies of the same monosynth, you could distribute notes to the various synths and basically treat them as if they were a polysynth.\footnote{This is an idea directly stolen from the MIDIPal.}

\item Further, you can have Gizmo add {\bf extra notes}, that is, pile up notes you're playing multiple times extremely rapidly.  For example, if you play the note G, Gizmo might send four G notes out essentially at the same time.  This is useful for fattening up a synth but not making it fully unison (mono).  For example, I have an Oberheim Matrix 1000 with six voices.  If I instruct Gizmo to provide two extra notes, then every time I press a note, Gizmo emits {\it three} copies of that note to the Oberheim, causing three voices to play together.  This basically turns the Matrix into a fatter two-voice polysynth.  {\bf Important Note}: Polyphonic Aftertouch is not repeated.\footnote{\label{serialbuffer}Because it overwhelms the Arduino's serial output, which blocks, causing Gizmo to miss messages, resulting in stuck notes.}

\item You can also perform {\bf chord memory}.  Here you first specify a chord, and then afterwards, every time you play a note, Gizmo will instead emit that chord, transposed so that its lowest note corresponds with the note you played.  This is a classic facility available on a number of synthesizers, including early ones such as the Oberheim OB-Xa, Sequential Circuits Prophet 600, Korg PolySix, and so on.\footnote{This idea was also directly stolen from the MIDIPal.}   {\bf Important Note}: Polyphonic Aftertouch is not chorded.\footnote{See Footnote \ref{serialbuffer}.}

\item Control Data (CC/PC/RPN/NRPN/Pitch Bend) follows a different path: a copy is sent to every channel that Gizmo is presently distributing to.  
%First, you're given the option of {\bf mapping CC to NRPN Data}.  Some devices (for example, the Matrix 1000 with 1.20 ROM) expect NRPN, but few DAWs (grrr) send NRPN; furthermore, most DAWs send raw CC values in total violation of MIDI spec.  At present Gizmo can convert all 128 raw CC numbers to their equivalent NRPN parameter numbers.

\item CC/RPN/NRPN from the Merge Channel is neither merged nor passed through, but is simply ignored.  This is because it is nontrivial to merge it.

\item Data from channels other than Default MIDI In and Merge is simply passed through.  {\bf Warning.}  If your Default MIDI Out is not the same as your Default MIDI In (that is, you're routing from one channel to another) you may have conflicts with other incoming data being routed to its equivalent out channel.

\item Finally, if you are distributing incoming notes, control data will also be {\bf distributed}, but instead of sending it round-robin to each of the relevant channels, control data is sent to {\bf all of them in parallel}.

\end{itemize}

I might add more unusual transformers later.  Anyway, here are your options:

	\begin{description}
	
	\item{\bf Go}\\
		Once you have set up your Thru options (see below), if you choose {\bf Go}, the Thru facility starts and you can start playing.  You'll see the word \texttt{PLAY} on the LED.

	\item{\bf Extra Notes}\\
		This lets you select how many {\it additional} simultaneous note-on (and note-off) messages Gizmo should send down a given channel in response to you playing a note.
		\begin{description}
		\item{\bf \textit{Choose}}: No Extra Notes, or 1...31 \hspace{3in}\smash{\includegraphics[width=1in]{none.pdf}} 
		\end{description}

	\item{\bf Distribute Notes}\\
		This lets you select the number of MIDI channels, beyond the Default MIDI Out channel, over which to distribute notes.  Let's say you selected 2, and the Default MIDI Out channel was \(M\).  This means that three MIDI channels will be used: \(M\), \(M+1\), and \(M+2\).  This wraps around: if, say, \(M = 15\), then it wraps around: the channels used would be 15, 16, and 1).
		
		If you played the five notes A, B, C, D, E, Gizmo would send A down channel \(M\), then send B down channel \(M + 1\), then send C down channel \(M + 2\), then send D down channel \(M\) again (wrapping around), then E down channel \(M+1\) again, and so on.  This would let you essentially use Mono devices listening in on the three MIDI Channels as a 3-voice polyphonic device.
		
		\begin{description}
		\item{\bf \textit{Choose}}: No Extra Channels, or 1...15 \hspace{2.8in}\smash{\includegraphics[width=1in]{none.pdf}} 
		\end{description}

	\item{\bf Chord Memory} or {\bf No Chord Memory}\\
		This will show the text \texttt{CHRD}, and you can then specify a chord (of up to eight notes).  Thereafter if you play a single note, the chord will be played instead, transposed so that its root note corresponds with the note you play. (If you play more than one note, more than one chord will be played).  If you have already chosen a chord, then selecting this option again will turn off chord memory.
				
		\begin{description}
		\item{\bf \textit{Choose}}: a chord of up to eight notes.
		\end{description}
		
	\item{\bf Debounce}\\
		This lets you choose the number of milliseconds between successive identical MIDI notes (the \(N\) value discussed earlier).  100 milliseconds seems to work well for me.
		\begin{description}
		\item{\bf \textit{Choose}}: No Debouncing, or 1...255 (milliseconds)\\\rule[0em]{0em}{0em}\hspace{\fill}\smash{\includegraphics[width=1in]{none.pdf}} 
		\end{description}

	\item{\bf Merge Channel}\\
		Here you select the alternate MIDI-In channel (if any) to merge with the Default MIDI In channel
		\begin{description}
		\item{\bf \textit{Choose}}: No Channel, Channels 1...16, or ALL Channels (OMNI)\hspace{0.25in}{\includegraphics[width=1in]{none.pdf}}~{\includegraphics[width=1in]{all.pdf}}
		\end{description}

%	\item{\bf CC-NRPN} or {\bf No CC-NRPN}\\
%		This will toggle whether raw CC is converted to equivalent NRPN values.

	\item{\bf Block Others} or {\bf Unblock Others}\\
		This lets you choose whether MIDI data on channels other than the Default MIDI In channel and the alternate MIDI-In channel should be blocked entirely or passed through unaltered.
				
	\end{description}


\clearpage

\section {The Measure Counter}
\label{measure}

\begin{wrapfigure}{r}{3in}
\vspace{-1.5em}\includegraphics[width=3in]{measure}
\vspace{-2em}\caption{\small Measure Counter Layout}
\vspace{-1em}\label{measurecounter}
\end{wrapfigure}

	The measure counter\footnote{The Measure Counter was added to Gizmo at the request of \texttt{Inkog}, a member of http:/\!/gearslutz.com and early adopter of Gizmo.} is basically a stopwatch which counts elapsed beats, measures (bars), and phrases (up to 127 phrases).  Alternatively it can count eighths of a second,\footnote{\label{footnotetenths}Why not tenths of a second?  Because \(10 \times 60 \times 128\) is larger than \(2^{16}\), so it won't fit in a 16-bit integer.} seconds, and minutes (up to 127 minutes) via the internal clock or external MIDI clock.  If the minutes or phrases exceed 127, then \texttt{HI} is displayed instead.
	
	To change what a measure (bar) or phrase means, long-press the Select button to bring up the Menu.  Here you can choose the {\bf Beats Per Bar} (a value from 1...16)\footnote{Why not 32, like Bars Per Phrase uses?  Because \(32 \times 32 \times 128 > 2^{16}\).  So basically for the same reason as Footnote \ref{footnotetenths}.} and the {\bf Bars per Phrase} (1...32), and as usual, call up the {\bf Options Menu} to change the {\bf Tempo}, etc.
	
\begin{wrapfigure}{r}{2in}
\includegraphics[width=2in]{measurebeats}
\vspace{-2em}\caption{\small Beats or 1/8 Second Ticks}\vspace{-3em}
\label{midichannelvalues}
\end{wrapfigure}


	Pressing the Middle Button starts and and restarts the stopwatch.  Long-pressing the Middle Button pauses the stopwatch (pressing the Middle Button afterwards will continue).  Additionally, external MIDI Start, Stop, and Continue commands will affect the stopwatch: Stop will pause it, Continue will continue from last pause, and Start will reset the stopwatch and start it fresh.

	Normally the measure counter shows beats, measures, and phrases.  But if you press the Select button, Measure will instead start displaying elapsed time, so it's basically a stopwatch.  More specifically, rather than counting beats, measures (bars), and phrases, Measure will count eighths of a second, seconds, and minutes.  The response to buttons is the same, as is the response to MIDI clock directives.

	You can also control the Measure Counter via MIDI: if you send a MIDI Play, it will reset the Measure Counter.  Note that if you send a MIDI Stop, and Gizmo is listening to external MIDI clock, then the Measure Counter will be halted (and MIDI Continue will continue the Counter).  However sending a MIDI Stop will not presently halt the stopwatch.

\clearpage
\section {Synthesizer MIDI Helpers}
\label{synthesizer}

Synthesizers often have weak spots in their MIDI implementations.  This application contains various submenus, one per synthesizer, designed to provide an assist for that particular model.\footnote{Why {\it these particular models?}  Because I own them. If you'd like me to write a helper for another machine here, send me a synth!}

\paragraph{A Note on Performance} Often these MIDI helpers are converting CC to NRPN, or converting NRPN to Sysex, and in the process they're emitting something bigger than they're getting (for example, an NRPN message could be between 3 and 5 times larger than a CC message).  This is a problem.  For example, if it receives CC at a very fast rate, Gizmo won't be able to send out corresponding NRPN fast enough and eventually Gizmo's serial input buffer will overflow.  To counter this, Gizmo sends out messages at a fixed rate (typically about 100 per second), and drops some on purpose if they arrive faster than that.   Gizmo will always send the most recent packet.  This is determined by some constant \texttt{\textit{SYNTH\_NAME}\_COUNTDOWN} in the appropriate file \texttt{synth/\textit{SynthName}.h} (lower values are faster rates).


\subsection{Waldorf Blofeld Module / Keyboard}  The Blofeld exposes parameters through CC, not NRPN; this means that not all of its parameters can be accessed (there too many of them).  Gizmo maps NRPN messages to corresponding Blofeld Sysex parameter change messages, which allows you to manipulate all of the Blofeld's parameters over NRPN.  The NRPN parameter numbers in question are 0...382 (MSB + LSB), corresponding to the Sysex parameters shown in Section 3.1 (``SDATA - Sound Data'') of the text file found in \texttt{docs/synth/WaldorfBlofeldSysex.txt} in the Gizmo distribution.  Values are MSB only.  Additionally, changing the value of NRPN parameter number 400 will set the buffer which these parameters effect, namely, 0 (Sound Mode Edit Buffer or Multi Instrument Edit Buffer 0), or 1--15 (Multi Instrument Edit Buffers 1...15).  The default is 0.

\paragraph{Display}  If you are setting the ID, then \texttt{ID} is displayed, along with the ID value.  Otherwise, just the set value is displayed.

\paragraph{Caveats} While you are sending NRPN this way, you can also send some CC messages, but not for CC numbers 6, 32--63, or 98--101: you'll need to send the NRPN equivalents for those.  The Blofeld seems to drop some Sysex packets if you send them too fast.  Significantly increasing \texttt{WALDORF\_BLOFELD\_COUNTDOWN} helps things but at the cost of lag and discretization.  It's probably better to just turn your encoder dial more slowly.

\subsection{Kawai K4 / K4r}\quad The K4 is only accessible via Sysex.  Gizmo maps CC messages to corresponding Kawai K4 Sysex parameter change messages, which allows you to tweak all of the K4's (non-Multi) parameters over CC.  The K4 has 89 parameters, broken into Single (0--69), Drum (70--81), and Effect (82--88) parameters.  You can send any of these parameters as a CC message number 0...88.  

Some of these parameters allow not 128 but 256 different values.  You can toggle Gizmo to sending values in the range 128...255 rather than 0...127 (or not) with a CC message sent to parameter number 100.

Finally, the K4's parameters 0--69 require a {\it source} value (0...4, representing s1, s2, s3, or s4 respectively); while parameters 70--81 require a {\it key note} (one of 0...60), and finally parameters 82--88 require a {\it submix} or {\it output channel} (a value 0...7.  You specify the source, key note, or submix/output channel by sending it as a value to CC parameter 101.

For more information, see the PDF file \texttt{docs/synth/KawaiK4CorrectedMIDIImplementation.pdf}.  In particular, look at to the breakout of parameters as specified in Section 5-6 and the parameter list proper in Section 6 (pay attention to the ``Parameter No'' column). 

\subsection{Oberheim Matrix 1000}  {\color{red} This code is not yet tested}\quad This machine is dramatically opened up by new firmware (Gligli's v1.16, and Bob Grieb's v1.20), which adds quasi-real-time parameter changes and NRPN support.\footnote{I strongly suggest you install Bob Grieb's, it's only \$30.  http:/\!/tauntek.com/Matrix1000Firmware.htm\quad By the way, if you have a Matrix 6 or 6R, check out http:/\!/tauntek.com/Matrix6Firmware.htm though it doesn't support NRPN.}  Gizmo's facility supports two items:

\begin{itemize}
\item If you are using NRPN to control the 1000, Gizmo will let you do the same to program its mod matrix.  To do this, first the {\bf matrix slot number} (0--9) as NRPN parameter 102.  After that, you can set the source (0--20) as NRPN parameter 103, the destination (0--32) as NRPN parameter 104, and the value (1--127, interpreted as \(-63\)---\(+63\)) as NRPN parameter 105.
\item Gizmo also lets you control the device via CC if your DAW doesn't support NRPN.  The CC parameter is simply the Matrix parameter number, except for parameter numbers 6 (which is CC 106), 38 (which is CC 107) and 98 (which is CC 108).  The CC value for Matrix parameters 0...98 is always the Matrix value plus 64 [this is how it's done in the NRPN mapping too].  You can also program the mod matrix just like the previous bullet, only using CC parameters rather than NRPN parameters.
\end{itemize}

\paragraph{Display} Just the set value is displayed.  If you set the mod matrix slot, that is displayed.  If you change any mod matrix element (source, destination, value), the current value is displayed.


\subsection{Korg Microsampler}  Anyone who owns one of these machines knows that its MIDI is a disaster.  Pitch Bend is steppy, only certain notes can be played, if MIDI START is received the pattern sequencer immediately starts up, and if the device is set to emit MIDI clock (via ``auto'' or ``internal'') then MIDI clock pulses are sent but not MIDI START, STOP, or CONTINUE.  These last two items can be overcome with judicious use of Gizmo's MIDI clock filter.  But on top of it all, the Microsampler's parameters can only be controlled via sysex,\footnote{And not all of them: you can't change the start or end trim positions, for example.  Furthermore, no {\it events} can be controlled by sysex.  So there's no sysex message to tell the Microsampler to start recording, or to normalize the sound.} and it's entirely undocumented. 

Gizmo's helper for the Microsampler converts NRPN to sysex.  There are four basic kinds of parameters.  First there are {\it bank parameters}, such as BPM.  These are changed straightforwardly with one NRPN command each.  Second, there are {\it pattern parameters} such as Pattern Length.  These require a pattern number in addition to the parameter number.  To change these, you either first set the pattern number \(n\) via NRPN 50 and then set the parameter via its NRPN \(p\), or you can change the parameter  directly by changing NRPN number \(p + 100 \times n\).  Third there are {\it sample parameters} such as Sample Level.  Similar to pattern parameters, these require a sample number in addition to the parameter number, and similarly, you can either first set the sample number \(n\) via NRPN 51 and then set the parameter via its NRPN \(p\), or you an change the parameter directly by changing NRPN number \(p + 100 \times n\).  

Finally, there are {\it FX parameters}.  These require not two but {\it three} variables: the FX type, the FX Parameter Number, and the parameter value proper.  To set these, you first set the FX Type (NRPN 31).  This switches the FX Type on the machine.  Then you either set the FX parameter number \(n\) via NRPN 52 and then set the parameter via its NRPN \(p\), or you can change the parameter directly by changing NRPN number \(p + 100 \times n\).  Note that the number of FX parameters varies depending on the FX type chosen.

The FX type can be set independently, as can the parameters emitted by Control 1 and Control 2 (via NRPN 32 and 33 respectively).  Note that the parameters emitted by these are a subset of possible parameters for a given FX type.

The full sysex documentation to the best of my understanding, plus the Gizmo NRPN parameter mapping, can be found in the file \texttt{docs/synth/KorgMicrosamplerSysex.txt}.


\paragraph{Display} If you are setting the Sample Number, Pattern Number, or Effects Parameter Number, then \texttt{S}, \texttt{P}, or \texttt{FX} are displayed respectively, plus the number.  Otherwise only the set value is displayed.

\subsection{Yamaha TX81Z} {\color{red} This code is not yet tested, and has known bugs}\quad The TX81Z's parameters can only be changed via sysex.  Gizmo maps NRPN messages to corresponding TX81Z Sysex parameter change messages, which allows you to tweak all of them over NRPN.   One oddity in Yamaha's Sysex is that it has MIDI channels.  Gizmo will set this channel to Gizmo's MIDI Out channel.

The TX81Z has a great many parameters, split into nine different categories, each assigned a number \(c\) as shown:  (0) {\it VCED,} (1) {\it ACED,} (2) {\it PCED,}  (3) {\it Remote Switch,}  (4) {\it Octave Micro Tuning,} (5) {\it Full Micro Tuning},  (6) {\it Program Change Table,} (7) {\it System Data,} and (8) {\it Effect Data}.  In all cases, to set a parameter number \(p\) from a given category~\(c\), you choose NRPN number (MSB+LSB) \(c \times 200 + p\). The parameter numbers for categories VCED, ACED, PCED, and Remote Switch are shown on pages 71--73 of the file \texttt{docs/synth/YamahaTX81ZManual.pdf}.   System Data and Effect Data are set similarly, but the meaning of the parameters is unknown to me.  Values are 7-bit (MSB) only.

For Octave and Full Micro Tuning, the parameter number \(p\) is the key number.  The value of this number is 14-bit (MSB+LSB), where the MSB is the note to assign the key and the LSB is a fine-tuning of the note.  Gizmo deviates from the TX81Z standard here to make programming easier: the fine-tuning ranges from 1...63, where 1 is \(-31\), 32 is 0, and 63 is +31.

Because the TX81Z has more than 128 patches, Program Change Table lets you (weirdly) map PC commands (0--127) to specific patches (0--184).  See page 68 of the file for information on how these 185 patches are organized.  In NRPN, the parameter number \(p\) specifies the program change number, and the value (MSB+LSB) is the patch number.

\paragraph{Display} Just the set value is displayed.

\paragraph{Caveats}  The TX81Z doesn't respond well to fast parameter changes.    Significantly increasing the value of \texttt{YAMAHA\_TX81Z\_COUNTDOWN} may help this but at the cost of lag and discretization.  It's probably better to just turn your encoder dial more slowly.



\clearpage

\section {The Sysex Dump Facility}
\label{sysex}

	The sysex facility allows Gizmo to load and emit sysex dumps of individual arpeggios or of slots (which can hold either step sequences or note recordings).
	
	The facility is simple.  First, you're asked to choose whether you want to upload/download {\bf Slots} or {\bf Arpeggios}.  After this, you're asked which slot or arpeggio {\bf number} you want to upload/download.  After you have selected one, the sysex facility enters its {\bf transfer mode}.
	
Initially the transfer mode will display \texttt{-~-~-~-} indicating that nothing has happened yet.  When you press the {\bf Select Button}, Gizmo will do a sysex dump of your requested slot or arpeggio.  You can send this dump as many times as you like.  Correspondingly, if you send a slot or arpeggio to Gizmo while in transfer mode, Gizmo will automatically read and save it.  (Note that Gizmo will only save an arpeggio if you had selected an arpeggio, and likewise only save a slot if you had selected a slot: otherwise it will display \texttt{SYSX} on receiving the wrong type sysex dump).

After either sending or receiving a sysex dump, Gizmo will display an ever-increasing integer.  The value of this integer means absolutely nothing: it exists so that when it changes you can tell that something was just successfully received or sent.  If Gizmo receives a bad sysex dump, it will display \texttt{FAIL}.

Gizmo's sysex dump format is simple.  Bytes 0--6 are as follows:\quad\texttt{0xF0 0x7D 'G' 'I' 'Z' 'M' 'O'}.\quad  Byte 7 is the version number (presently \texttt{0x00}). Byte 8 is the sysex type (\texttt{0x00} for slots, \texttt{0x01} for arpeggios).  Next, Gizmo provides a memory dump of the bytes forming a single slot or arp structure (\texttt{struct~{\textunderscore}slot} or \texttt{struct~{\textunderscore}arp} in Gizmo's code).  Each byte in this memory dump is nybblized into two bytes in the sysex stream: that is the high 4 bits of the byte are sent first, followed by the low 4 bits.  For example, if a memory byte was binary 01101001, then the sysex would contain the two bytes 0000{\bf 0110} and 0000{\bf 1001} in that order.  Next, Gizmo has a checksum, which is simply the low 7 bits of the sum of all the nybbles.  Finally comes \texttt{0xF7}.   This structure is not really meant to be parsed, though a cursory reading of the Gizmo code should make it clear what the memory bytes mean.

\paragraph{Installation Note} The Sysex facility is not enabled by default, because in order to use it you also need to make a small modification to the MIDI library.  You can't use the Sysex facility on the Uno because there's not enough RAM.  See the {\tt All.h} file for installation instructions.

\clearpage
\section {Options}
\label{options}

	Options sets global parameters for the device.  These parameters are stored in Flash memory and survive a power cycle.  Some options can be also accessed from certain other applications as a convenience.  
	
	\begin{description}


	\item{\bf Tempo}\\
	If Gizmo is following its own internal clock rather than relying on an external MIDI clock , this specifies how fast a quarter note is.  (See {\bf Options\(\boldsymbol\rightarrow\)MIDI~Clock}).  When not in {\bf Bypass Mode}, the tempo is shown by the pulsing on/off of the {\bf Beat / Bypass Light} (see Figure~\ref{HighLevelGizmoLayout}). 
	
\begin{description}
		\item{\bf \textit{Choose}}: 1 ... 999 Beats Per Minute.  Note that this is a large number, and so may require you to choose it with the left potentiometer, then fine-tune it with the right potentiometer.\footnote{Gizmo can go lots faster than 999: in theory it could go clear to 31200 or so.  But then your synthesizer would explode and we wouldn't want that.}
		\item{\bf \textit{Alternatively}}: Tap the Middle Button.  The BPM will be set to the rate you tap.
   		\item{\bf \textit{Note}} At the borders (near 1 or 999) you may find it harder to precisely set the BPM due to anti-potentiometer noise code in Gizmo. Just use the right knob to fine-tune your value. 
		\end{description}
		
	\item{\bf Note Speed}\\
		Various applications (arpeggiators, step sequencers)
		produce notes at a certain rate relative to the tempo.
		For example, though the tempo may specify that a
		quarter note is set to 120 Beats Per Minute, the 
		arpeggiator might be generating eighth notes and
		so is producing notes at twice that speed.  You specify the 
		note speed here.
		
		Note speed is shown by pulsing the {\bf Note Pulse Light} (Figure~\ref{HighLevelGizmoLayout}).

\vbox{		\begin{description}
			\item{\bf \textit{Choose}}:
			\begin{tabbing}
			Eighth Triplet (Triplet 64th Note)\hspace{3em}\=\(\myfrac{1}{24}\)\hspace{1em}\=Beat\\
			Quarter Triplet (Triplet 32nd Note)\>\(\myfrac{1}{12}\)\>Beat\\
			Thirty-Second Note\>\(\myfrac{1}{8}\)\>Beat\hspace{0.7in}\=\smash{\includegraphics[width=2in]{NoteSpeed1.pdf}}\\
			Half Triplet (Triplet 16th Note)\>\(\myfrac{1}{6}\)\>Beat\\
			\\
			Sixteenth Note\>\(\myfrac{1}{4}\)\>Beat\\
			Triplet\>\(\myfrac{1}{3}\)\>Beat\\
			Eighth Note\>\(\myfrac{1}{2}\)\>Beat\>\smash{\includegraphics[width=2in]{NoteSpeed2.pdf}}\\
			Quarter Note Triplet\>\(\myfrac{2}{3}\)\>Beat\\
			\\
			Dotted Eighth Note\>\(\myfrac{3}{4}\)\>Beat\\
			Quarter Note\>\(1\)\>Beat\\
			Dotted Quarter Note\>\(1\ \, \myfrac{1}{2}\)\>Beats\>\smash{\includegraphics[width=2in]{NoteSpeed3.pdf}}\\
			Half Note\>\(2\)\>Beats\\
			\\
			Dotted Half Note\>\(3\)\>Beats\\
			Whole Note\>\(4\)\>Beats\\
			Dotted Whole Note\>\(6\)\>Beats\>\smash{\includegraphics[width=2in]{NoteSpeed4.pdf}}\\
			Double Whole Note\>\(8\)\>Beats
			\end{tabbing}
		\end{description}
		}


%	\begin{center}\begin{tabular}{@{}lr@{}}
%	{\it Note Speed}&{\it Divisor}\\
%					\hline\\[-0.9em]
%	
%	Eighth Triplet (Triplet 64th Note)	&1\\
%	Quarter Triplet (Triplet 32nd Note)	&2\\
%	Thirty-Second Note				&3\\
%	Half Triplet (Triplet 16th Note)		&4\\
%	Sixteenth Note					&6\\
%	Triplet						&8\\
%	Eighth Note					&12\\
%	Quarter Note Triplet				&16\\			% add
%	Dotted Eighth Note				&18\\			% do we include this?
%	Quarter Note					&24\\
%	Half Note Triplet				&32\\			% Was: Quarter Note Tied to Triplet.   Delete?
%	Dotted Quarter Note				&36\\
%	Half Note						&48\\
%	Whole Note Triplet				&64\\			% Was: Half Note Tied to Two Triplets.  Delete?
%	Dotted Half Note				&72\\
%	Whole Note					&96\\
%	Dotted Whole Note				&144\\
%	Double Whole Note				&192\\
%	\end{tabular}\end{center}


	\item{\bf Swing}\\
			Swing, or {\bf syncopation}, is the degree to which every other note is delayed.  0\% means no swing at all.  100\% means so much swing that the odd note plays at the same time as the next even note.  That's a lot.
		\begin{description}
		\item{\bf \textit{Choose}}: 0\% ... 100\% (larger percentages are more swing, that is, longer delay every other note)
		\end{description}

		Swing is only applied if the Note Speed is set to thirty-second, sixteenth, eighth, quarter, or half-notes.

	\item{\bf  Transpose}\\
		Here you can state that any notes generated by Gizmo be {\bf transposed} up or down by as much as 60 notes. If a note is transposed to the point that it exceeds the MIDI range, it is not played.  When appropriate, some applications (such as the MIDI Gauge) ignore this option. 
		\begin{description}
		\item{\bf \textit{Choose}}: \(-60\) ... 60 
		\end{description}

	\item{\bf  Volume}\\
		Like Transpose, here you can state that any notes generated by Gizmo have their volume changed by multiplying their MIDI note velocity by some value.  If a resulting MIDI note velocity exceeds the maximum (127), it is bounded to 127.
			
			
			\begin{description}
			\item{\bf \textit{Choose}}:
			\begin{tabbing}
			1/8\\
			1/4\\
			1/2\hspace{3.65in}\smash{\includegraphics[width=1.7in]{fractions2.pdf}}\\
			1\qquad (default)\\
			2\\
			4\\
			8
			\end{tabbing}
		\end{description}

	\item{\bf In MIDI}\,\footnote{{\it Why aren't these called MIDI In and MIDI Out?}  Because then they'd be indistinguishable on the menu screen before the text started scrolling.}\\
		Many applications expect notes and other controls to come in via a specific input MIDI channel.  You specify it here.
		\begin{description}
		\item{\bf \textit{Choose}}: No Channel, Channels 1...16, or ALL Channels (OMNI)\hspace{0.25in}\smash{\includegraphics[width=1in]{none.pdf}}~\smash{\includegraphics[width=1in]{all.pdf}}
		\end{description}
	\item{\bf Out MIDI}\\
		Many applications emit notes etc. via a specific output MIDI channel.  You specify it here.  Other applications can emit notes on several different channels, in which case this value determines the default channel used.
		\begin{description}
		\item{\bf \textit{Choose}}: No Channel, or Channels 1...16\hspace{0.5in}\smash{\includegraphics[width=1in]{none.pdf}}
		\end{description}
	\item{\bf Control MIDI}\\
		Gizmo can be controlled via NRPN or CC messages.  You specify the channel here on which it will listen for them.
		\begin{description}
		\item{\bf \textit{Choose}}: No Channel, or Channels 1...16\hspace{0.5in}\smash{\includegraphics[width=1in]{none.pdf}}
		\end{description}
		
		The default\footnote{If you need to change these parameters, see constants (such as \texttt{CC\_LEFT\_POT\_PARAMETER}) in the file \texttt{MidiInterface.h}.  Remember, you'll want the Pot parameters to be 14-bit.} CC and NRPN Messages are (note that CC 14 and 15 are 14-bit: their LSB are sent with 46 and 47):
		
		\noindent\hspace{-1.5em}\begin{tabular}{@{}rr|lllp{2.4in}@{}}
		\multicolumn{2}{c}{\it Parameter}\\
		{\it CC}&{\it \hspace{-0.8em} NRPN}&{\it Min  Value}&{\it Max Value}&{\it MSB/LSB}&{\it Description}\\[0.1em]
		\hline\\[-0.9em]
		14&0	&0	& 1023 & MSB + LSB (46)& Turn the Left Potentiometer\\
		15&1	&0	& 1023 & MSB + LSB (47) & Turn the Right Potentiometer\\
		102&2	&0 (released)	&Not 0 (pressed) & MSB& Press/Release the Back Button\\
		103&3	&0 (released)	&Not 0  (pressed) & MSB & Press/Release the Middle Button\\
		104&4	&0 (released)	&Not 0 (pressed) & MSB & Press/Release the Select Button\\
		105&5	&Any Value	&Any Value & MSB& Toggle Bypass\\
		106&6	& Any Value &Any Value & MSB& Unlock Potentiometers.  When an NRPN message is received, normally the on-board potentiometers are locked so that turning them has no effect.  Pressing an on-board button unlocks them: so does sending this NRPN message.\\
		107&7	&	Any Value &Any Value & MSB& Start Clock.  If Gizmo's clock is stopped, resets and starts it.  When appropriate, sends a MIDI START message.\\
		108&8	& Any Value	&Any Value & MSB& Stop Clock.  If Gizmo's clock is playing, stops it.  When appropriate, sends a MIDI STOP message.\\
		109&9	&	Any Value &Any Value & MSB & Continue Clock.    If Gizmo's clock is stopped, continues it.  When appropriate, sends a MIDI CONTINUE message.\\
		127&10	&	Any Value &Any Value & MSB & Go Down Menu.   In menus and numerical or glyph displays, pressing the Middle Button goes up.  Pressing this CC/NRPN instead goes down.\\
%		110&---	& 0	&127 & --- & Turn the Left Potentiometer (relative).  Values \(<\) 64 will more and more decrease the left potentiometer.  Values \(\>\) 64 will more and more increase it.\\
%		111&---&	0 &127 & --- &  Turn the Right Potentiometer (relative).  Values \(<\) 64 will more and more decrease the right potentiometer.  Values \(\>\) 64 will more and more increase it.\\
		\end{tabular}

\vspace{1em}

		These operations are in addition to those discussed in the Step Sequencer (Section \ref{stepsequencersec}).

		You can also start/stop the clock via by Long-Pressing the Middle button while in the Options menu.  And you can start/continue the clock by Long-Pressing the Select button while in the Options menu.   See also {\bf Starting, Stopping, and Continuing the Clock} in Section \ref{startingclock}.
		
	\item{\bf MIDI Clock}\\
		Gizmo can respond to a MIDI clock, ignore it, or emit its own MIDI clock.  The use of the {\bf Options\(\boldsymbol\rightarrow\)Tempo} setting will depend on the setting chosen here.

\vbox{
		\begin{description}
			\item{\bf \textit{Choose}}:
			\begin{tabbing}
			Ignore\hspace{0.5in}\=Use an internal clock but let any external MIDI clock pass through.\\
			Use\>Use an external MIDI clock and also let it pass through.\\
			Consume\>Use an external MIDI clock but don't let it pass through.\\
			Generate\>Use an internal clock and emit it as a MIDI clock.\\
			\>~~~~~~Don't let any external MIDI clock pass through.\\	
			Merge\>Like Generate, but also emit when Bypass is on.\\	
			Block\>Use an internal clock but don't emit it as a MIDI clock.\\
			\>~~~~~~Don't let any external MIDI clock pass through.\\
			\end{tabbing}
		\end{description}
}
\vspace{-1em}
		You can also start/stop the clock via by Long-Pressing the Middle button while in the Options menu.  And you can start/continue the clock by Long-Pressing the Select button while in the Options menu.   See also {\bf Starting, Stopping, and Continuing the Clock} in Section \ref{startingclock}.

	\item{\bf Divide}\\
	If your MIDI Clock option is set to {\bf Use}, {\bf Ignore}, {\bf Merge}, or {\bf Generate} (that is, something which causes Gizmo to emit MIDI Clock), then you have the option of slowing down the emitted MIDI clock by {\it dividing} it by some constant \(N\)  For example, if your clock is presently running at 120 BPM, and you're doing Generate, and \(N=3\), then Gizmo's emitted MIDI clock will be at 40BPM.   Or if you're doing Use and the incoming MIDI Clock is at 100BPM, and \(N=2\), then Gizmo will adjust this to 50BPM when sent out.  Note that Gizmo's internal applications will use the original clock speed.  Note that {\bf Bypass Mode} disables all this nonsense.
	
Why is this useful?  Here's an example scenario.  You have a sequencer on Device A which is playing a song at 120 BPM, and you want to slave another sequencer, on Device B, to it which plays its part at half that speed.  You can do this by having Device A send to Gizmo, and Gizmo send to Device B with a divided clock.

			\begin{description}
					\item{\bf \textit{Choose}}: 2 ... 16 (the value for \(N\)) when using {\bf Divide}, or None\hspace{1.45in}\smash{\includegraphics[width=1in]{none.pdf}}
			\end{description}

		
	\item{\bf Click} or {\bf No Click}\\
		This toggles the note played by Gizmo to provide a click track for applications such as the Step Sequencer or Recorder.  If you select {\bf Click}, you will be asked to play a {\bf note}.  The note pitch and velocity form the click.  If you select {\bf No Click}, this turns off the click track.
		
\hspace{\fill}\smash{\\\includegraphics[width=1in]{note.pdf}}\vspace{-1.5em}

	\item{\bf Screen Brightness}\\
		This changes the brightness of the LED matrix.
		\begin{description}
		\item{\bf \textit{Choose}}: 1...16 (higher values are brighter)
		\end{description}
		
		
	\item{\bf  Menu Delay} (Mega Only)\\
		Gizmo often will display text (such as \texttt{SWING}) by filling the screen with the first few letters (in this case, \texttt{SWI}), then {\it delaying} for a certain interval, then scrolling through the whole word horizontally. You can specify how long that delay is.
	
		
		\begin{description}
			\item{\bf \textit{Choose}}:
			\begin{tabbing}
			0\hspace{2em}\=Seconds\hspace{0.5in}\=(scroll immediately)\\
			1/4\>Seconds\\
			1/3\>Seconds \>(really 3/8 seconds)\hspace{1in}\=\smash{\includegraphics[width=2in]{fractions.pdf}}\\
			1/2\>Seconds\\
			1\>Second\>(default)\\
			2\>Seconds\\
			3\>Seconds\\
			4\>Seconds\\
			8\>Seconds\\
			\(\infty\)\>Seconds\>(never scroll)\>\smash{\includegraphics[width=0.6in]{infinity}}\\
			S\>\>(1 second, then half-speed {\bf Slow} scroll)
			\end{tabbing}
		\end{description}

	\item{\bf  Auto Return} (Mega Only)\\
		In some applications, such as the Arpeggiator and Step Sequencer, you can immediately enter and change various parameters just by twisting a real or MIDI CC knob.  Normally you have to press Select or Back to exit.  But if you set Auto Return to some value (in seconds) other than None, you will automatically be exited after that number of seconds.

			\begin{description}
					\item{\bf \textit{Choose}}: 1 ... 16 (the number of seconds) or None\hspace{1.45in}\smash{\includegraphics[width=1in]{none.pdf}}
			\end{description}

%	\item{\bf  \textit{Mega:}\quad CV+Velocity \textit{or} CV+Aftertouch \textit{or} No CV}\\
%		Toggles whether Gizmo will send control voltage and gate information.  If {\bf CV+Velocity}, then control voltage will be sent out DAC A (I2C address 0x62), velocity will be sent out DAC B (I2C address 0x63), and digital pin 8 will be set HIGH (5V) when a note is played (or LOW (0V) when a note is released), thus acting as a Gate.\footnote{As mentioned in an earlier footnote, Gizmo assumes you're DACs of this kind: https:/\!/www.adafruit.com/product/935} If {\bf CV+Aftertouch}, then the same is done except that aftertouch will be sent out DAC B instead of velocity.  If {\bf No CV}, then nothing is sent.  Pitch Bend at present has no effect.	\footnote{Note that at present if you have set either knob in the Controller to control voltage, that DAC will be unavailable even outside the Controller.  I'll fix that soon.}
		
	\item{\bf The About Screen}\\
		Presently says:\quad \texttt{GIZMO V5 (C) 2018 BY SEAN LUKE}
	\end{description}

\clearpage
%	\section{About the Clock}

%	\subsection{The Internal Clock}
%	\label{internalclock}

%The Arduino has an internal clock with 32 bits, yielding a maximum of 42,949,672,96 ticks (one tick per microsecond) before it rolls back over to 0.  This sounds like a lot, but it's not: it's a bit over an hour.\footnote{71.5827882666 minutes to be exact.}  This means that once an hour, Gizmo might do something odd, if you're using the internal clock.  Most likely it could drop a single pulse (1/24 of a quarter note), so it might be off by that much with respect to other instruments synced to it.  You've been warned.

\section{Starting, Stopping, and Continuing the Clock }
	\label{startingclock}

		You can start, stop, and continue the clock via NRPN/CC as discussed earlier (see {\bf Control MIDI} in Section \ref{options}).   But there's another ``secret'' way to do it using Gizmo's buttons.  If you are in the Options menu, or if you're at the Root (top-level) menu, long-press the Middle Button: this toggles {\bf starting} or {\bf stopping} the clock.  And if you long-press the Select button, you can toggle {\bf continuing} or {\bf stopping} the clock.  Remember that you can also get to the Options menu from inside the Arpeggiator, Step Sequencer, or Recorder.
		
		Neither long-pressing the Middle Button, nor using NRPN to start/stop/continue, will do anything if the MIDI Clock is set to {\bf Use, Divide,} or {\bf Consume} (in which case the MIDI Clock is out of your control).  See {\bf MIDI Clock} in Section \ref{options}.
				
		Note that starting and stopping the clock doesn't mean that an application starts or stops playing.  However, the Start message {\it will} cause applications to reset themselves to the beginning when appropriate.   This means that if you want to start/stop/pause/continue the Step Sequencer (for example), you can do it as follows: first stop the clock, then start the sequencer playing.  Note that it won't play because there's no clock pulses.  But now you can start, stop, and pause/continue the sequencer by controlling the Start/Stop/Continue clock messages directly.
		
		Manually stopping and starting the clock can produce results which look like bugs, but aren't, so you have think about the effects.  For example, if you're in {\bf Generate Clock} mode and you've manually stopped the clock, and you go into the Step Sequencer and start playing, nothing will happen. More than once this has confused me enough to look through the code thinking I'd written an error.
		

\section{Chaining Multiple Gizmos}

I have three Gizmos which I chain together when playing. One might run a Drum Sequencer, another might run a Sequencer, and another might run an Arpeggiator.  How is this done?

\paragraph{Wiring} The Gizmos are connected in series just like you imagine: the output of each plugs into the input of the next.  In my particular arrangement, I have a controller keyboard (a StudioLogic SL 990) attached to a Novation Zero MKII (which provides CC knobs and buttons).  The Zero goes into Gizmo A, which goes into Gizmo B, which goes into Gizmo C, which goes into a MIDI router, which is attached to my synthesizers.  

\paragraph{Setting up MIDI} If your controller can be set to output on different MIDI channels, then the easy way to do all this is to set the MIDI In of each Gizmo to a different value, and also set the MIDI Out of their downstream synthesizers to different values.  For example:

\begin{itemize}
\item Gizmo A, running the Drum Sequencer, could be set to MIDI In channel 1, with default MIDI out of channel 1 (going to a drum machine) and also sending to channel 2 for certain drum tracks (going to a different drum machine).  
\item Gizmo B, running the Step Sequencer, could be set to MIDI In channel 3, with default MIDI out of channel 3 (going to a synthesizer).
\item Gizmo C, running the Arpeggiator, could be set to MIDI In channel 4, with default MIDI out of channel 4 (going to a second synthesizer).
\item A final synthesizer, which I control from the Controller when all three Gizmos are doing their thing, could be set to MIDI Channel 5.
\end{itemize}

\paragraph{Sending Note Data} By default Gizmo pass through MIDI data on all channels other than its MIDI In.  Thus if each of your Gizmos have a different MIDI In, you can send data to each one just by changing your controller's MIDI channel, assuming they're all currently running applications.

If a Gizmo is in the top menu or is running something which doesn't pass through non-MIDI-In data, you can always turn on Bypass.  Remember that Bypass is on by default when you power your Gizmo up.  I've not found MIDI note latency to be bad at all even in my very strung out setup (SL\(\leftarrow\)Zero\(\leftarrow\)Gizmo\(\leftarrow\)Gizmo\(\leftarrow\)Gizmo\(\leftarrow\)Router\(\leftarrow\)Synths)

\paragraph{Controlling a Gizmo} By default Gizmo pass through MIDI data on channels other than its MIDI In.  If you just want to control a given Gizmo, simply set the upstream and downstream Gizmos to bypass.  


If you just want to control a given Gizmo, simply set the upstream and downstream Gizmos to bypass.  



\section{Development Hints}

So you want to write a Gizmo application?  Here's some hints for you.

\subsection{Notes}

The code is largely written in C, with a few little C++ features, notably
a few functions which pass by reference (arguments with \texttt{\&} in front of them).  The
code is presently written so as to cram as many features as possible into an
{\bf Arduino Uno}, which has only 32K of code space, 2K of working RAM, and 1K of Flash.  
Thus there are a lot of little oddities here and there meant to conserve space and/or
speed.  A few examples:

\begin{itemize}

\item {\bf Division}\quad  Division on the Uno is slow and costly.  Gizmo often uses custom \texttt{div}
  functions which are smaller and faster than a general use of the \texttt{\textbackslash} and \texttt{\%}
  operators.  See the file \texttt{Divison.h}  Also in some cases I've found that (perhaps due to a bug?)
  gcc appears not to be converting division by a power of two into a simple right shift.
  So Gizmo often uses right shifts rather than divisions to force this.  Perhaps this
  is cargo cult programming, perhaps not.

\item {\bf The State Machine}\quad  The state machine is a large case statement located in the go()
  method in \texttt{TopLevel.cpp}. To compile to a jump table, the cases have to be contiguous
  integers; since the Mega and Uno have different sets of cases, this means different
  integer \texttt{\#define}s.  They are NOT presently enums!  Adding a new cases is always fun 
  as you have to move all the \texttt{\#define}s down or up by 1.   Also, in some cases I've 
  inlined the certain functions in the case statement as it makes the code smaller 
  than calling them separately.

\item {\bf Helper Functions}\quad  Gizmo has lots of helper functions meant to reduce code redundancy.
  In several cases these helper functions have intentional side effects which you may
  not realize: for example, the \texttt{doMenuDisplay(...)} function [as a random example] expects
  to handle, and clear, the \texttt{entry} flag.  These can be serious gotchas if you've not
  looked them over carefully.

\item {\bf Strange Code}\quad  Sometimes you'll see weird things in the code, such as code that's been inlined
 in the \texttt{TopLevel.cpp} state machine rather than appear in their own functions; or oddly written
 code which could have been written more straightforwardly; or obvious opportunities for removing
 redundancy through a function call.  The reason behind this weirdness is almost always that the
 Atmel compiler for the Arduino Uno will compile tighter the way it's written.
 
 Indeed at some point soon I'll have to break the two codebases apart: the Uno's restrictions are making a mess of
 spaghetti code out of the codebase.

\end{itemize}

\subsection{Code Files}

The files are in certain categories:

\noindent\begin{tabular}{r@{\hspace{2em}}p{4.7in}}
	\bf Hardware &\\
	\tt MidiShield.h/cpp	& Macros and variables for the SparkFun MIDI Shield \\ 
	\tt LEDDisplay.h/cpp	& Code to do drawing on the Adafruit 16x8 LED \\ 
	\tt DAC.h/cpp		& Code to control the DACs \\ 
\\
\\
	\bf Core Code  &\\
	\tt Timing.h/cpp	& 	Internal and external clocks, notes, pulses, beats \\ 
	\tt TopLevel.h/cpp	& 	The core.  Contains various important functions, the state machine. \\ 
	\tt MidiInterface.h/cpp	& MIDI callbacks, MIDI wrapper functions \\ 
	\tt Utility.h/cpp		& Utility functions used by various applications \\ 
	\tt Storage.h/cpp		& Slot (file) storage in Flash \\ 
	\tt Options.h/cpp		& Storage of global and application-specific options \\ 
	\tt Division.h/cpp		& The arduino doesn't have hardware divide.  This
				file contains various functions for dividing by
				specific common constants (10, 12, 100, etc.) \\ 
	\tt All.h		& 	One header file to bring them all and in the
				darkness bind them. \\ 
	\tt Gizmo.ino		& The standard Arduino entry functions \\ 
\\
\\

	\bf General Applications  &		\it (Note that many parts of applications are embedded
				in the state machine in \texttt{TopLevel.cpp}, and a few bits
				are located in the MIDI callback functions in
				\texttt{TopLevel.cpp}, though I try to avoid embedding
				in the callback functions when possible.)\\
	\tt Arpeggiator.h/cpp & 	The arpeggiator \\ 
	\tt StepSequencer.h/cpp  & 	The step sequencer \\ 
	\tt DrumSequencer.h/cpp  & 	The drum sequencer \\ 
	\tt Control.h/cpp	& 	The MIDI control surface \\ 
	\tt Gauge.h	& 		The MIDI gauge.  There's no \texttt{Gauge.cpp} because it's
				entirely contained within the state machine in
				{\texttt{TopLevel.cpp}} \\ 
	\tt Recorder.h/cpp	& 	The note recorder \\ 
	\tt Split.h/cpp	 & 	The keyboard splitter / layerer \\ 
	\tt Thru.h/cpp & 		The MIDI thru application \\ 
	\tt Measure.h/cpp	& The Measure counter application \\ 
	\tt Synth.h/cpp	& The Synthesizer helper top-level \\ 
\\
\\
	\bf Synth Helpers  & \it (In the \texttt{synth} subdirectory.)\\
	\tt KawaiK4.h/cpp	& 	NRPN\(\rightarrow\)Sysex converter for the Kawai K4 and K4r\\ 
	\tt KorgMicrosampler.h/cpp	& 	NRPN\(\rightarrow\)Sysex converter for the Korg Microsampler\\ 
	\tt OberheimMatrix1000.h/cpp	& 	CC\(\rightarrow\)NRPN converter for the Oberheim Matrix 1000 with the 1.16 or 1.120 ROM upgrades\\ 
	\tt WaldorfBlofeld.h/cpp	& 	NRPN\(\rightarrow\)Sysex converter for the Waldorf Blofeld\\ 
	\tt YamahaTX81Z.h/cpp	& 	NRPN\(\rightarrow\)Sysex converter for the Yamaha TX81Z\\ 
\end{tabular}




\subsection{Application Building Tutorial}

\paragraph{Defining the Application}
Applications in Gizmo are included (or not) in two steps.  First, you have to turn on a \texttt{\#define} which will include all of the application's code.  Second, you need to include the name of the application in the right spot in the application menu.  We'll start with the \texttt{\#define}.

We'll make an application called {\bf Foo}.  Let's assume that your application is meant to run on the Arduino Mega.  In \texttt{All.h} there's an area where it says:

\begin{verbatim}
#if defined(__MEGA__)
...
#endif
\end{verbatim}

This area defines which applications will be turned on when we compile for the Mega.  In this region, add 
your own application:

\begin{verbatim}
#define INCLUDE_FOO
\end{verbatim}

While we're add it, let's add a comment which lets developers know that your application is an option.  You'll notice that in the file \texttt{All.h}, there is a section labeled

\begin{verbatim}
// -- APPLICATIONS --
\end{verbatim}

Under this section, let's add a bit of documentation about our new \texttt{\#define}

\begin{verbatim}
// INCLUDE_FOO                // This will include the Foo application
\end{verbatim}

\paragraph{Modifying the State Machine}
Next we want to add our application as a state in the state machine.  The state machine is for the moment a large enum called \texttt{\_State}, located in the file \texttt{TopLevel.h}.  The first state is state number 255: \texttt{STATE\_NONE}.  After this the {\bf root state}, defined as state number 0: \texttt{STATE\_ROOT}.  The root state is the state that the Gizmo starts up in.  The root state shows the application menu.

After the root state are up to 12 {\bf application states}, the last of which is \texttt{STATE\_OPTIONS}.  These states have no associated number, and so will be numbered consecutively after 0 (the root).  You will add a state here.  For simplicity, add your state as the last application before \texttt{STATE\_OPTIONS}.  In file \texttt{TopLevel.h}, just before the line

\begin{verbatim}
    STATE_OPTIONS,
\end{verbatim}

add the following:

\begin{verbatim}
#ifdef INCLUDE_FOO
    STATE_FOO,
#endif
\end{verbatim}

Now that you have a state, you can add the code for it.  In the file \texttt{TopLevel.cpp}, just before the line

\begin{verbatim}
        case STATE_OPTIONS:
\end{verbatim}

add the following:

\begin{verbatim}
#ifdef INCLUDE_FOO
        case STATE_FOO:
            {
            stateFoo();
            }
        break;
#endif
\end{verbatim}

\paragraph{Modifying the Root Menu}

We also need to modify the root menu to include your application.  It depends on whether you're using a Mega or an Uno.  In the file \texttt{All.h}, change the lines:

\begin{verbatim}
#define MENU_ITEMS()     const char* menuItems[11] = { PSTR("ARPEGGIATOR"), PSTR("STEP SEQUENCER"), PSTR("DRUM SEQUENCER"), PSTR("RECORDER"), PSTR("GAUGE"), PSTR("CONTROLLER"), PSTR("SPLIT"), PSTR("THRU"), PSTR("SYNTH"), PSTR("MEASURE"), options_p };
#define NUM_MENU_ITEMS  (11)
\end{verbatim}

... to something like ...

\begin{verbatim}
#define MENU_ITEMS()     const char* menuItems[12] = { PSTR("ARPEGGIATOR"), PSTR("STEP SEQUENCER"), PSTR("DRUM SEQUENCER"), PSTR("RECORDER"), PSTR("GAUGE"), PSTR("CONTROLLER"), PSTR("SPLIT"), PSTR("THRU"), PSTR("SYNTH"), PSTR("MEASURE"), PSTR("FOO"),  options_p };
#define NUM_MENU_ITEMS  (12)
\end{verbatim}

This adds your application FOO as a menu option just before Options.  If you're using Sysex, this exceeds your total number of apps: you'll need to remove one (or mess around with TopLevel.h/cpp to increase the total apps).

This adds a state which, when called, will call the \texttt{stateFoo()} method.  Additional states would typically 
get tacked onto the end of the \texttt{\#define} list and be called stuff like \texttt{STATE\_FOO\_EDIT} or \texttt{STATE\_FOO\_WHATEVER}.  

The critical item here is that your \texttt{\#define}s in \texttt{TopLevel.h} and \texttt{TopLevel.cpp} must be in the same position,
so the switch statement stays contiguous and can remain a jump table.

Now, create some \texttt{Foo.h} and \texttt{Foo.cpp} files, and add \texttt{Foo.h} to \texttt{All.h} as an \texttt{\#include}.
 
 Your \texttt{Foo.h} file should look like this:
 
 \begin{verbatim}
#ifndef __FOO_H__
#define __FOO_H__ 
#ifdef INCLUDE_FOO

void stateFoo();

#endif
#endif
\end{verbatim}

Your \texttt{Foo.cpp} file should look like this:

\begin{verbatim}
#include "All.h"
#ifdef INCLUDE_FOO

void stateFoo()
    {
    }

#endif
\end{verbatim}


\end{document}
