\documentclass{article}
\usepackage{fullpage}
\usepackage{mathpazo}
\usepackage{microtype}
\usepackage{graphicx}
\usepackage{wrapfig}
\usepackage{amsmath}
\usepackage{amssymb}
\usepackage{array}
\usepackage{color}

\newcommand\myfrac[2]{#1/#2}
\begin{document}

\noindent {\Huge\bf Gizmo}\\
{\large \bf An Arduino MIDI Tool\\
By Sean Luke\quad sean@cs.gmu.edu}\\
\\

\noindent Gizmo is a small MIDI device intended to be inserted between the MIDI OUT of one device and the MIDI IN of one or more other devices.  There are a number of tools like this, such as the esteemed MIDIPal\footnote{Though they share nothing in common, MIDIPal is quite similar to Gizmo in functionality and goals.  But I wasn't even aware of the existence of MIDIPal when I started building Gizmo.  And besides, re-inventing the wheel is fun!  MIDIPal's page is http:/\!/mutable-instruments.net/midipal\ \ This device has been discontinued, but its open source software lives on in other hardware clones, such as the MIDIGal (https:/\!/midisizer.com/midigal/) or MIDIBro (http:/\!/www.audiothingies.com/product/midibro/)} or a variety of tools from MIDISolutions.\footnote{http:/\!/www.midisolutions.com/}

Gizmo is the unholy union of an Arduino Uno or Arduino Mega 2560,\footnote{http:/\!/arduino.cc} an Adafruit 16x8 I2C LED matrix\footnote{https:/\!/www.adafruit.com/categories/326\quad I use the 1.2-inch 16x8 matrix with red round LEDs, which works well.  This particular model can be found at https:/\!/www.adafruit.com/products/2037}, and a SparkFun MIDI Shield\footnote{https:/\!/www.sparkfun.com/products/12898} to perform a variety of helpful MIDI tasks.  Gizmo has several built-in applications, and assuming there's space\footnote{As of this writing, Gizmo on the Arduino Uno uses 32,252 bytes out of 32,256 available.  So if you're like to extend Gizmo on the Uno, you'll need to make sure your extension fits within 4 bytes :-)  Instead: get a Mega.} you can develop additional applications for it.

This is a software project: the hardware is just three off-the-shelf boards and four wires.  There's no enclosure, no custom board.  If you'd like to do that, I'd welcome seeing it.

Gizmo comes built-in with the following applications and other capabilities:

\paragraph{Applications}
\begin{itemize}
\item {\bf Arpeggiator.}  Gizmo's arpeggiator has built-in up, down, up/down, repeated chord, note-assign, and random arpeggios spanning one to three octaves.  Additionally, you can define up to ten additional arpeggiator patterns, each up to 32 notes long, involving up to 15 different chord notes, plus rests.  Arpeggios can be {\it latched} (or not), meaning that they may continue to play even after you have released the keys.  You can specify  the note value relative to the tempo (ranging from eighth triplets to double whole notes), the note length as a percentage of the note value (how legato or staccato a note is), the output MIDI channel, and the degree of swing (syncopation).  Gizmo will show the arpeggio on-screen.
\item {\bf Step Sequencer.}  Gizmo's step sequencer can be organized as 12 16-note tracks, 8 24-note tracks, or 6 32-note tracks.  Each note in a track can have its own unique pitch and velocity, or be a rest, or continue (lengthen, tie) the previous note.  You can also fix the velocity for an entire track, mute tracks, fade their volume, specify their independent output MIDI channels, specify the note value relative to the tempo (again ranging from eighth triplets to double whole notes), the note length as a percentage of the note value (how legato or staccato a note is), and the degree of swing (syncopation).    The step sequencer lets you edit in two modes: either by triggering independent steps like a classic drum sequencer, or by playing a sequence of notes.  The Arduino Uno can store up to two sequences in its slots (shared with the Recorder, discussed next).  The Mega can store up to nine sequences.  Gizmo will show and edit sequences on-screen.
\item {\bf Recorder.}  The Arduino doesn't have a lot of memory, but we provide a note recorder which records and plays back up to 64 notes (pitch and velocity) played over 21 measures. The recorder has 16-voice polyphony.   It's enough to record a very short ditty.  You can also set the recorder to loop the recording while playing, and to provide a click track.  The Arduino Uno can store up to two recordings in its slots (shared with the Step Sequencer).  The Mega can store up to nine recordings.
\item {\bf MIDI Gauge.}  The gauge will display all incoming MIDI information on one or all channels.  Note on, note off, and polyphonic aftertouch are shown with pitch and velocity (or pressure).  Channel aftertouch is shown with the appropriate pitch.  Program changes indicate the number.  Pitch bends indicate the value in full 14-bit.  Control Change, NRPN, and RPN messages display the number, value, and whether the value is being set, incremented, or decremented.  Sysex, song position, song select, tune request, start, continue, stop, and system reset are simply noted.  Rapid-fire MIDI signals such as MIDI clock, active sensing, and time code just turn on individual LEDs.
\item {\bf MIDI Control Surface}.  Gizmo has two potentiometers and two buttons.  Each can be configured to send a unique Program Change, Control Change, NRPN, or RPN commands, or (on the Mega) control voltage values via either of two DACs.  The potentiometers can only send MSB values.

\item {\bf \textit{Mega Only:}\quad Keyboard Splitter}  You can split incoming MIDI notes by pitch and route them to different MIDI out channels.  You can also fade notes between two channels depending on note velocity.

\item {\bf \textit{Mega Only}\quad Midi Thru Filter: Distributing and Layering Notes}  Gizmo can {\it distribute} notes to different MIDI channels as you play: this lets you play multiple mono synths as if they were polyphonic, for example.  Gizmo can also {\it layer} notes: when you play a note, Gizmo will send some \(N>1\) note-on messages instead of one.  This lets you use a polyphonic synthesizer to play a note with more than one voice, making it fatter, but not necessarily going all the way to unison.  And yes, you can do both at the same time.

\end{itemize}

\paragraph{Other Cool Stuff}

As if this weren't enough, Gizmo also comes with the following capabilities.  All of these settings are automatically saved permanently in memory.

\begin{itemize}
\item {\bf Tempo and the MIDI Clock.}  Gizmo can sync to an external MIDI clock, can ignore it and use its own internal clock, and can emit the same as a MIDI clock.  Gizmo can also pass the external MIDI clock through or block it.  When using its own internal clock, Gizmo supports tempos ranging from 1 to 1024 BPM.  You can start, pause/unpause, and stop this clock (sending the appropriate MIDI Start, Stop, or Continue signals).  Gizmo can also divide the MIDI clock, thereby outputting a slower clock than it is inputting.

Gizmo applications can also be set with a {\it note value} (or {\it note speed}) to play their notes relative to this clock, ranging from eighth triplets (1/24 beat) to double whole notes (8 beats).  At any time, Gizmo pulses on-screen LEDs indicating the clock and note speed pulses.  Finally, Gizmo can be configured to vary the degree of {\it swing} or {\it syncopation}.

\item {\bf In and Out MIDI Channels}  Gizmo can be configured with input and output MIDI channels (some applications will use these: others will have multiple channels and will treat these as defaults).  Both channels can be turned off, and the input channel can be OMNI.  Gizmo pulses on-board LEDs indicating incoming and outgoing MIDI data.  

\item {\bf Remote Control via MIDI NRPN.} Gizmo can be controlled remotely via NRPN rather than using its on-board controls.  If you start using the controls, Gizmo's onboard potentiometers are locked out so their noise will not interfere with your remote control.  You can unlock the potentiometers by pressing a button on-board Gizmo, or by sending Gizmo's {\it Release} NRPN message.

\item {\bf Transposition and Volume Control}  You transpose all of Gizmo's MIDI output pitch by anywhere from \(-60...60\) steps.  Additionally you can multiply the MIDI output velocity (volume) by multiples of two ranging from \(1/8\) to \(8\).

\item {\bf Bypass.}  You can quickly put Gizmo in Bypass mode, to pass through all MIDI signals and generate no new ones (with the exception of the Controller, see Section \ref{controller}).  Putting Gizmo in Bypass mode will also send an All Notes Off message, so toggling Bypass is also a useful way to hush your synthesizer.

\item {\bf Screen Brightness}  You can change the screen brightness.  

\item {\bf \textit{Mega Only}\quad Menu Delay}  Many of Gizmo's menus display the first few letters of a menu item, then pause for some interval of time before they start scrolling the full item.  You can change this interval.

\item {\bf \textit{Mega Only}\quad Control Voltage}  You can configure Gizmo to optionally output control voltage (CV) values from 0--5V, through two DACs, in response to notes being played.    One CV output is in response to the note pitch, and the other is in response to the note velocity.

\end{itemize} 

\paragraph{Future Stuff (on the Arduino Mega)}

I have written but not tested some additional code.  It won't appear on the Uno version (not enough space) but may show up on the Mega version.  This stuff includes:

\begin{itemize}

\item {\bf Per-MIDI Device Options}  There will be an area for specific kinds of MIDI devices.  For example, I have written code to convert NRPN values into the unusual sysex messages required by the Kawai K4 Synthesizer so it can be manipulated easily via a standard MIDI controller.

\end{itemize}

Hopefully there'll be more than this.  Message filtering?  A MIDIPal-style CC LFO?  Longer note recorder songs?  And so on.

\section{Building Gizmo}

\begin{enumerate}

\item {\bf Assemble the Hardware}\quad Gizmo consists of an Arduino Uno or Arduino Mega, a SparkFun MIDI Shield\footnote{https:/\!/www.sparkfun.com/products/12898} attached via four wires (I2C) to an Adafruit 16x8 LED Matrix.\footnote{This matrix comes in various colors LED shapes (round, square), and matrix sizes (0.8 inch, 1.2 inch).  I personally use a 1.2 inch round red matrix.  The URL for my model is https:/\!/www.adafruit.com/products/2037}  The SparkFun MIDI Shield does not come with headers to plug into the Arduino, and you'll need to get those too: it's assumed you'll either use plain headers, or use stackable headers.  I use stackable headers as it makes it easy for me to plug my four test wires into it.

Attach the four wires for SDA, SCL, 5V, and GND between the MIDI Shield and the LED Matrix.  

\item {\bf Modify your Arduino Software}\quad You will need a pretty recent version of the Arduino software, as newer versions have better compilers which compile more compact code.  I developed Gizmo on Arduino 1.6.12 for MacOS X.  You will want to modify the source code files for the Arduino libraries in two spots:

\paragraph{Add Nonblocking I2C Writes}  This will dramatically speed up I2C for our purposes.
\begin{enumerate}
\item Locate your \texttt{Wire.cpp} and \texttt{Wire.h} files.  On the Mac they're located in\\\texttt{Arduino.app/Contents/Java/hardware/arduino/avr/libraries/Wire/src/}

\item Add the following line inside the ``public'' method region in \texttt{Wire.h:}

\begin{verbatim}
    uint8_t endTransmissionNonblocking();
\end{verbatim}

\item Add the following method to \texttt{Wire.cpp}:

\begin{verbatim}
    uint8_t TwoWire::endTransmissionNonblocking()
        {
        uint8_t ret = twi_writeTo(txAddress, txBuffer, txBufferLength, 0, 1);
        return ret;
        }
\end{verbatim}
\end{enumerate}

\paragraph{Reduce the I2C Buffer Size from 32 to 20} This gives us some extra static RAM space.

\begin{enumerate}
\item Locate your \texttt{Wire.h} file.  Again, on the Mac it's located in\\
\texttt{Arduino.app/Contents/Java/hardware/arduino/avr/libraries/Wire/src/}

\item Change the \texttt{BUFFER\_LENGTH} constant in the \texttt{Wire.h} file as follows:

\begin{verbatim}
        #define BUFFER_LENGTH 20   // Was 32
\end{verbatim}

\item Identify your \texttt{twi.h} file.  On the Mac it's located in\\
\texttt{Arduino.app/Contents/Java/hardware/arduino/avr/libraries/Wire/src/utility/}

\item Change the \texttt{TWIBUFFER\_LENGTH} constant in the \texttt{twi.h} file as follows:

\begin{verbatim}
        #define TWI_BUFFER_LENGTH 20    // Was 32
\end{verbatim}
\end{enumerate}

\item {\bf Install the FortySevenEffects MIDI Library}\quad But not straight off of its website.\footnote{https:/\!/github.com/FortySevenEffects/arduino\_midi\_library}  The MIDI library is in a state of flux and is having some significant additions being made.  The Gizmo code is extremely sensitive to code size (at least on the Uno).  Thus for now, use the version I include in Gizmo's Github repository.

\item {\bf Install the SoftReset Library}\quad A copy of it is in Gizmo's Github repository.

\item {\bf Switch the MIDI Shield to ``Prog''}

\item {\bf Build and Upload Gizmo's Code to the Arduino}\quad  It should compile cleanly for either an Arduino Uno or an Arduino Mega 2560.  On the Uno it'll complain of Low Memory.

\item {\bf Switch the MIDI Shield to ``Run''}\quad  You can plug in MIDI cables now.

\item {\bf Reset Gizmo}\quad Power up the Arduino.  Then hold down all three pushbuttons on the MIDI Shield.  While doing so, press and release the reset button on the MIDI Shield.  Continue to hold down the three push buttons until the two LEDs on the MIDI Shield light up.  Let go of the pushbuttons.  

This procedure initializes the memory, options, and files in the device.  Now you're ready to go!

\end{enumerate}

\subsection{Starting Gizmo Headless}

Gizmo normally targets the SparkFun MIDI Shield, but you might be able to use its code with some other MIDI shield, as long as the shield just intercepts Serial to do its MIDI and doesn't interfere with I2C.  Such MIDI shields are unlikely to have the button, pots, or LEDs that the SparkFun shield has.  But that's okay, you can control the buttons and pots virtually via NRPN, and just do without the two LEDs.

To do this you'll need to reset Gizmo's options without needing to press any buttons (as you won't have any).  You can do it like this:

\begin{enumerate}
\item In the \texttt{All.h} file, change the line 
\begin{verbatim}
        //#define HEADLESS_RESET
\end{verbatim}
...to...
\begin{verbatim}
        #define HEADLESS_RESET
\end{verbatim}
\item Compile and download the code, and run it once.  Gizmo will show its opening screen, then display \texttt{RSET} and not do anything else.
\item In the \texttt{All.h} file, change the line back, that is, to 
\begin{verbatim}
        //#define HEADLESS_RESET
\end{verbatim}

\end{enumerate}

Gizmo's options have been reset in a slightly different way than normal: the Control MIDI option (see ``Control MIDI'' in Section \ref{options}) has been set to Channel 16, rather than OFF.

Now you need to modify Gizmo's code to run headless, so it doesn't bother lighting up the on-board LEDs or reading the buttons and pots.   You can do this as follows:

\begin{enumerate}
\item In the \texttt{All.h} file, change the line 
\begin{verbatim}
        //#define HEADLESS
\end{verbatim}
...to...
\begin{verbatim}
        #define HEADLESS
\end{verbatim}
\item Compile and download the code, and run it.
\item You should now be able to control Gizmo via NRPN (see ``Control MIDI'' in Section \ref{options}).
\item In the \texttt{All.h} file, you might wish to change the line back, that is, to 
\begin{verbatim}
        //#define HEADLESS
\end{verbatim}

\end{enumerate}

I make no guarantees that this will work: but it might.

\section{User Interface}

\begin{wrapfigure}{r}{3in}
\vspace{-1.5em}\includegraphics[width=3in]{OverallLayout.pdf}
\vspace{-2em}\caption{\small High-Level Gizmo Layout}
\label{HighLevelGizmoLayout}
\end{wrapfigure}

Gizmo has a modest interface: a 16x8 LED display, two additional on-board LEDs (red and green), three buttons, and two potentiometers (dials): but it strives hard to make good use of these.  

Gizmo's top-level interface layout is a menu hierarchy.  You select menu items, which may take you deeper into the hierarchy, or you can go back up towards the top of the hierarchy.  The top level menu chooses between different {\bf applications}.  Once you have entered an application, its {\bf application number} is displayed in the {\bf current application region} (see Figure~\ref{HighLevelGizmoLayout}).

The current application LEDs light up right-to-left in a certain pattern with increasing application numbers, as shown in Figure~\ref{applicationvalues}.  Gizmo's interface can presently support up to eleven top-level applications.  

Also shown in Figure~\ref{HighLevelGizmoLayout} are the {\bf Beat/Bypass} light and the {\bf Note Pulse} light.  The Beat/Bypass turns on and off every {\it beat} (quarter note).  This shows the current {\bf tempo}.\footnote{A beat occurs every 24 {\it pulses} of a MIDI Clock or Gizmo's internal clock.}  The Note Pulse light turns on and off every {\it note pulse}: this is the speed you have chosen for the arpeggiator, step sequencer, etc.  Note pulses are defined in terms of note value: for example, if the note pulse is presently in sixteenth notes, then this light will turn on and off four times faster than the beat light, which is in quarter notes. The note pulse light is also affected by the degree of swing (syncopation) you have defined.

If Gizmo's clock is presently {\bf stopped}, either because you have stopped it internally or because Gizmo is being driven by an external MIDI clock which is currently stopped, then {\bf both the Beat/Bypass and the Note Pulse lights will be off} (though if Bypass mode is on, then Beat/Bypass will still blink rapidly).  See Section \ref{startingclock} for more information about starting and stopping the clock.

The reset of the screen is given over to the application.  In many cases an application will display text and other data in the {\bf Application Display} region, while lighting up certain status LEDs in the {\bf Application Footer}.

You wend your way through menus, choose applications, and manipulate them using Gizmo's three buttons and two potentiometers as follows:

\paragraph{Buttons}  The left button is called the {\bf back button}.  The right button is called the {\bf select button}.  The center button is called (unsurprisingly) the {\bf middle button}.   The buttons respond to being {\bf pressed} and immediately released, and also respond to {\bf long presses}: holding a button down for a long time (presently one-half second) and then releasing it. 

\begin{wrapfigure}{r}{1.6in}
\includegraphics[width=1.6in]{Application.pdf}
\vspace{-2em}\caption{\small Application Values}\vspace{-4em}
\label{applicationvalues}
\end{wrapfigure}

Buttons do different things.  Here are certain common items:

\begin{itemize}
\item {\bf Pressing the Back Button}\quad This is the ``escape'' or ``cancel'' button: it {\bf always goes back up the menu hierarchy.}  Thus you can't ever get lost\,---\,just keep pressing the Back button and you'll eventually find yourself at the root menu.
\item {\bf Long-Pressing the Back Button}\quad This toggles {\bf bypass mode}, where Gizmo tries hard to act as if it's just a MIDI THRU coupler between its input and output MIDI wires: it passes through all the MIDI it receives, and with one exception (the Controller, see Section \ref{controller}) it doesn't output any new data.  In bypass mode, the {\bf Beat/Bypass} light (Figure~\ref{HighLevelGizmoLayout}) will stop beating and instead will flash rapidly.
\item {\bf Pressing the Select Button}\quad If you're being presented with a menu of options, or a number to choose, or a glyph to choose, the Select button will select the currently-displayed option.  In some applications (such as the Step Sequencer, Arpeggiator, and Recorder) the Select Button is assigned other tasks.
\end{itemize}

\paragraph{Potentiometers}

Gizmo's potentionmeters  (or {\bf pots}) do a number of tasks, including scrolling through menu options, moving a cursor horizontally or vertically, or choosing a number.  The left pot does primary tasks, and the right pot assists when appropate.  Note that these are potentiometers, not encoders: if you start turning one, you may experience a big jump initially.  The potentiometers have a limited range (0...1023), and a significant degree of noise, so realistically they can only choose numbers between 0...256 or so.  If asked to select a {\it big} number, the {\bf left pot} will act as a coarse-level selector, and the {\bf right pot} will fine-tune the number.


\paragraph{On-Board LEDs}

There are two LEDs on-board the SparkFun MIDI Shield.  The {\bf red LED} is toggled on and off to reflect incoming MIDI data.  The {\bf green LED} is toggled on and off to reflect outgoing MIDI data.  

\section{Initializing Gizmo}

Gizmo must be initialized before it can be used.  You can also do a factory reset on Gizmo with this procedure:

\begin{enumerate}
\item Hold down all three buttons
\item Press the reset button
\item Continue to hold down all three buttons until both on-board LEDs have lit.
\item Let go of the buttons.
\end{enumerate}

Gizmo will reset all of its flash RAM and reboot.  This means that all options will be set to their default state, and all file storage (including arpeggios) will be emptied.

\section{Starting Gizmo}

\begin{wrapfigure}{r}{2in}
\vspace{-1.5em}\includegraphics[width=2in]{Gizmo1.pdf}
\vspace{-2em}\caption{\small Boot Screen}\vspace{-2em}
\label{BootScreen}
\end{wrapfigure}

If you power up Gizmo, or press its reset button, it will display its {\bf boot screen}, which includes the version number (at right, the version number is ``1'').

Gizmo will then show the {\bf root menu}.  Use the left potentiometer to scroll to any of the following applications, and use the select button to select an application.  Available applications are:
\section {The Arpeggiator}

\begin{wrapfigure}{r}{3in}
\vspace{-1.5em}\includegraphics[width=3in]{arpeggio.pdf}
\vspace{-2em}\caption{\small The Arpeggiator Display}\vspace{-1em}
\label{arpeggiator}
\end{wrapfigure}

The arpeggiator allows you to play different kinds of arpeggios, and also to create and save up to ten of them.  The arpeggiator's main menu lets you choose to play five different kinds of pre-set arpeggios, or to select an arpeggio from user-created arpeggio slots 1--10, or finally to create an arpeggio.

An arpeggio is a pattern for repeatedly playing the notes in a chord.  When you hold down a chord on the keyboard, the arpeggiator uses this pattern to play the notes, typically one at a time, according to the pattern selected.  Patterns do not specify exact notes, but rather note orderings.  For example, an arpeggio might be defined as 1, 2, 2, 3, 4, 2, 3: if 1 is set to be the root (lowest) note of your chord, then if you play a C7 chord (C E G B$\flat$), then the arpeggiator will repeat the sequence C E E G B$\flat$ E G.  This pattern will repeat as long as you hold down the keys in the chord.  

Alternatively if you toggle the {\bf latch}, then the playing will continue after you have released the keys, and will only shift to a new chord when you release all notes and then hold down an entirely new chord.  Latch mode is toggled by pressing the Middle button while playing an arpeggio.  When Latch mode is engaged, an LED will light as shown in Figure~\ref{arpeggiator}.

Also, normally you toggle {\bf bypass mode} by Long-Pressing the back button.  But that's not fast enough to turn arpeggiation on and off in performance.  So the arpeggiator also lets you toggle bypass by pressing the {\bf select} button.  (Of course you can also toggle bypass via NRPN if you like). 

\begin{wrapfigure}{r}{2.7in}
\begin{center}
\vspace{-4em}\includegraphics[width=1in]{updown}
\vspace{-1em}
\caption{\small Up, Down, and Up-Down Symbols}
\vspace{-3em}
\end{center}
\end{wrapfigure}


\paragraph{Playing Preset Arpeggios}

There are six preset arpeggios: {\it up}, {\it down}, {\it up-down}, {\it random}, {\it chord repeat}, and {\it assign}.  They work as follows:

\begin{itemize}
\item {\bf Up}.  Each note in your chord is played in turn, lowest note first.  When all notes are exhausted, the pattern is repeated.
\item {\bf Down}.  Each note in your chord is played in turn, highest note first.  When all notes are exhausted, the pattern is repeated.
\item {\bf Up-Down}.  Each note in your chord is played in turn, lowest note first, up to but not including the highest note.  When all notes are exhausted, all notes in the chord are played in turn, highest note first, down to but not including the lowest note.  When all notes are again exhausted, this pattern repeats.

\item {\bf Random}.  Notes in your chord are played at random.  Gizmo tries not to play the same not twice in a row.
\item {\bf Chord Repeat}.  The entire chord is repeatedly played in its entirety.
\item {\bf Assign}. Each note in your chord is played in turn, in the order in which you had pressed the keys when you formed the chord.  When all notes are exhausted, the pattern is repeated.
\end{itemize}

As you play a preset arpeggio, the notes played by the arpeggio will appear at right.  The symbol representing the preset will appear at left.

While playing a preset arpeggio, you can select the number of octaves by choosing calling up the menu by Long-Pressing the Select button, then choosing {\bf Octaves}.  If you have an octaves value \(\geq 0\), then when the arpeggiator exhausts its notes, it will repeat them again one octave higher, then another octave higher, and so on, up to the number of octaves chosen. You an select any value from 0--6 octaves.\footnote{Note that this means that if your chord is larger than one octave, you may not get what you expected.  For example, if you are playing UP with C5 E5 G5 D6, and with octaves\(=\)1, then the arpeggiator will play C5 E5 G5 D6 C6 E6 G6 D7.  Note that C6 is played after D6 is played.  That's the current behavior right now anyway.}

\begin{wrapfigure}{r}{1.6in}
\vspace{-1em}
\begin{center}
\vspace{-1.5em}\includegraphics[width=1in]{arpeggioexample.pdf}\\
Example Arpeggio\\[2em]
\vspace{-1.5em}\includegraphics[width=1in]{arpeggioup.pdf}\\
Ascending Notes\\in Sparse Mode\\[2em]
\vspace{-1.5em}\includegraphics[width=1in]{arpeggiodoubleup.pdf}\\
Ascending Notes\\in Dense Mode\\
\end{center}
\vspace{-1em}\caption{\small Example Arpeggio Editing Displays}
\label{arpeggioediting}
\end{wrapfigure}


\paragraph{Creating an Arpeggio}

The last menu option in the top-level Arpeggiator menu is {\bf Create}.  If you choose this, you can create an Arpeggio.  To do this you enter notes one by one.  These won't be the {\it actual} notes that are played when you hold down a chord.  Rather, after you save the result, the Arpeggiator will compact these notes into a sequence of consecutive numbers.  For example, if you play C3 E3 G3 E3 G2, then the Arpeggiator will compact these into 2 3 4 3 1.  Then when you play this as an arpeggio, the Arpeggiator will assign each of these numbers, in order, to the notes in your chord. 

The first question you will be asked is \texttt{NOTE}: the Arpeggiator is asking you to specify the note in your arpeggio pattern which will correspond to the root (lowest) note in a played chord.  This allows you to make a pattern which plays notes both above and below the played chord.  For example, if the Root is C3, then the above pattern will instruct the Arpeggiator to play its final note {\it below} the chord root.

After you have entered the root, the Arpeggiator will wait for you to start playing notes.  You can enter a sequence of up to 32 notes, including 15 unique notes.  You can also enter rests by pressing the Middle button.  If you enter more than 15 unique notes, or more than 32 notes in total, the Arpeggiator will simply refuse to accept the next note.

As you enter notes, they are displayed in the left half of the LED, and the latest note pitch is displayed in the right half. Figure~\ref{arpeggioediting} shows different arpeggio patterns.  If you have eight or fewer unique notes in your pattern, then they will be displayed in {\it sparse mode} as shown at right.  If you have over eight notes, they will be displayed in {\it dense mode} in order to show all of the notes.  The LEDs don't show the notes per se, but rather the consecutive numbers that the notes represent.  For example, the ascending notes shown at right could be C, D, E, F, G, A, B, C, or they could be C, C\(\sharp\), D, D\(\sharp\), E, F, F\(\sharp\), B.

If you don't like the latest portion of your arpeggio, you can scroll the cursor back to an earlier section by turning the right knob.  Then if you start playing, the later notes will be discarded and the arpeggio pattern will continue from where the cursor was.  You can also scroll to view your past notes, then scroll all the way back to continue.

An {\bf important note regarding the scrolling.}  The cursor is always located at the position where new notes will be entered.  However as you scroll back and forth, you'll hear the notes of your arpeggio.  When you scroll and hear a note, that note is the note {\it immediately before} the cursor position.  It's the last note in the arpeggio before you start adding new notes. It is {\it not} the note that you'll replace when you start entering notes. 

Anyway, when you are satisfied with your arpeggio, Long-Press the Select button to save.  You'll be asked for a slot to save in, and after that, you'll be back in the main Arpeggiator menu.

\paragraph{Playing a User-Created Arpeggio}

To play a user-created arpeggio, from the main Arpeggiator menu, choose one of the numbers 0--9, each of which represents the arpeggiator slot you had saved your arpeggio in (these options appear after the preset options).  Then hold down a chord!  As you play a user-created arpeggio, the notes played by the arpeggio will appear at right.  The arpeggio pattern will scroll by at left.


The preset arpeggio patterns (such as Up-Down) vary with the number of notes in your chord: if you hold down five notes for example, and octaves=0, then five notes will be played in the arpeggio pattern.  This isn't the case for user-created arpeggios, which have a fixed user-specified number of notes.

What happens if your arpeggio is designed for three notes but you hold down  five notes in your chord?  Only three of your five notes will be played.

What happens if your arpeggio is designed for five notes but you hold down only three notes in your chord?  Then notes from the bottom (or top, as appropriate) of your chord will be transposed up/down one octave to make up the remaining notes.

Also, because they are fixed in size, user-created arpeggios do not respond to the octaves option.

\paragraph{More Arpeggiator Options}

In addition to Octaves, the arpeggiator menu (Long-Press the select button) also includes the Options Menu as a submenu.\footnote{Recall that the Options Menu is for options shared among several applications: that's why it's a separate menu.  That and code space issues.}  See Section \ref{options} for detailed information on each of the options.  The Arpeggiator responds to the following options:

\vspace{1em}
\begin{tabular}{llll}
Tempo& 			In MIDI&		Transpose&	MIDI Clock\\
Note Speed& 		Out MIDI& 	Volume	&\\
Swing& 			Control MIDI&	Length	&\\
\end{tabular}

\vspace{1em}
Of particular interest to you may be Length and Tempo.  And Swing!

\begin{wrapfigure}{r}{3in}
\vspace{-4em}\includegraphics[width=3in]{stepsequencer}
\vspace{-2em}\caption{\small Step Sequencer Display}\vspace{-1em}
\label{stepsequencer}
\end{wrapfigure}

\section {The Step Sequencer}

The Step Sequencer makes it easy to create 16-note, 24-note, or 32-note multitrack loops common on devices such as drum machines.  16-note loops have up to twelve tracks.  24-note loops have up to eight.  32-note loops have up to six tracks.

\begin{wrapfigure}{r}{1in}
\vspace{-1.5em}\includegraphics[width=1in]{none.pdf}
\vspace{-2em}\caption{\small None}\vspace{-2em}
\label{none}
\end{wrapfigure}

When you choose the Step Sequencer from the root menu, you will be asked to load a sequence from a slot.  You can do this, or you can create a new empty sequence by choosing \texttt{-~-~-~-} (meaning {\it none}).  Slots are displayed by number, plus an \texttt{R} (for Recorder) or \texttt{S} (for Step Sequencer) if the slot is filled with a file by one of the two applications.  This letter blinks if the file isn't the type of your current application.

If you have chosen {\it none}, or have chosen an empty slot or one presently filled by the Recorder, you have to initialize the slot first: you'll be asked to choose between 16, 24, or 32 notes per track. After this you are taken to the main Step Sequencer display.  

Figure~\ref{stepsequencer} shows the general step sequencer display structure.  The {\it Tracks} region is where sequencer tracks are displayed and edited.  The {\it Track Number} area tells you the number of the currently-edited track.  Figure~\ref{tracknumber} explains how to interpret this.  The {\it Stopped} LED tells you if the step sequencer is currently stopped (or playing).  The {\it Track Uses a Default Velocity} LED tell you if all notes in the current track are fixed to the same velocity, or permitted to use different velocities.  The {\it Track Uses a Fader} LED is lit when the pre-track fader (volume) is set to something other than maximum volume for the current track.  Finally, the {\it MIDI Out Channel for Track} tells you what channel the current track is playing.  The patterns for the MIDI Out Channel for Track are the same as those in Figure~\ref{midichannelvalues}, except that MIDI Out does not support ALL (OMNI).


\begin{wrapfigure}{r}{1.8in}
\includegraphics[width=1.8in]{track.pdf}
\vspace{-2em}\caption{\small Track Numbers}\vspace{-2em}
\label{tracknumber}
\end{wrapfigure}

Each track has a blinking {\it edit cursor} which determines, normally, where notes are entered if you play them.  The location of this cursor is determined by the potentiometers.  You select your current track by turning the left knob.  In a given track, you choose the current note position by turning the right knob.  The Tracks region also displays a blinking vertical set of LEDs which indicate which notes in the tracks the Step Sequencer is currently playing.  As the Step Sequencer plays notes, this vertical group heads to the right, eventually wrapping around.  This is the {\it play position cursor}.

Each track takes up one or two rows in the Tracks area: 16-note tracks take a single row, and 24-note and 32-note tracks take two rows.    There can be up to twelve tracks, yet the tracks display region is only six LEDs high.  This of course means that  the Tracks region scrolls as you turn the left knob.

You enter notes by moving the edit cursor to the a location, then pressing keys.  You can also press the Middle Button to enter a rest, or Long Press the Middle Button to enter a {\it continuation} (basically the equivalent of a {\it tie} in music).  A continuation just means ``keep playing the previous note'': it essentially lengthens the previous note.  You can enter multiple continuations in series to make a note as long as you like.

Pressing the Select button will toggle whether the Step Sequencer is playing.  When you start playing the Step Sequencer, it waits for the clock to reach a measure boundary before beginning.  Furthermore, even if you start playing the Step Sequencer, if the clock isn't running it won't play at all.

\begin{wrapfigure}{r}{1in}
\vspace{-3em}\includegraphics[width=1in]{playpositionmode}
\vspace{-1em}
\caption{\small Play Position Mode Cursor.  The current track is marked by the two dots.}\vspace{-2em}
\label{playpositionmode}
\end{wrapfigure}

\paragraph{Play Position Mode}

If you turn the right knob fully to the left, you will find that the cursor scrolls right off the left side of the screen.  At this point you are in {\bf play position mode}.  Here, if you press a key, it is entered at the play position cursor.

In play position mode, the function of the Middle Button changes.  Now if you press the Middle Button, it will toggle muting the current track.  You can tell a track is muted because the LEDs at each end of the track are blinking.  Additionally, if you Long Press the Middle Button, it will erase all the notes on the current track (but not change its settings).

You can tell you're in play position mode because the play position cursor has changed to include two dots which indicate the current track, as shown in Figure~\ref{playpositionmode}.

Play position mode isn't all that easy to enter notes, particularly if there is any degree of swing: I'd stick with the normal mode.  You can get out of play position mode just by turning the right knob so that the edit cursor is once again on-screen.

\paragraph{Menu Options}

If you Long Press the Select button, you'll call up the Step Sequencer's menu.  The Step Sequencer has a number of options:

\begin{itemize}
\item {\bf Reset Track}\quad This wipes out and resets the track.  Unlike Long-Pressing the Middle Button while in Play Position mode, this doesn't just clear the track of notes: it also resets all of its settings to their defaults.
\item {\bf Length}\quad This specifies the note length of notes in the track, as a percentage (0 is fully staccato and 100 is fully legato).   You can choose numerical values, or choose \texttt{DFLT}, which tells the Step Sequencer to use the default note length specified in the Options menu (see Section \ref{options}).
\item {\bf Out MIDI (Track)}\quad This specifies the MIDI Out Channel for notes in the track,    You can choose a value 1...16, or choose \texttt{DFLT}, which tells the Step Sequencer to use the default MIDI Out Channel specified in the Options menu (see Section \ref{options}).
\item {\bf Velocity}\quad This lets you specify a velocity that {\it all} notes in the track will use (as a value 0...127).  If you would instead prefer that each note in the track use its own independent velocity as you had entered them, you can choose \texttt{FREE}.
\item {\bf Fader}\quad This lets you specify a volume modifier which is multiplied against the velocities of the notes in the track.  Possible values are 0...127, where 0 completely hushes the notes, and 127 lets them play at their specified velocities. 

% At present NO ECHO MODE is turned off on the Uno -- we don't have enough space.
%\item {\bf No Echo} or {\bf Echo} \quad This toggles {\it no-echo mode}.  Normally when you enter a note, it's automatically played so you get some feedback.  But in some circumstances (such as when you enter notes from a synthesizer's keyboard, but are having the sequencer play sequences out to that same synthesizer) this isn't desirable.  In no-echo mode, you can prevent the note from being automatically played.
\item {\bf Save} \quad This saves the sequence. You then choose the slot to save in.  Slots are again displayed by number, plus an \texttt{R} (for Recorder) or \texttt{S} (for Step Sequencer).
\item {\bf Options} \quad This brings up the Options menu.
\end{itemize}

\paragraph{More Step Sequencer Options}

Beyond the settings above, the step sequencer responds to a number of options in the Options Menu.  See Section \ref{options} for detailed information on each of the options.  The Step Sequencer responds to the following options:

\vspace{1em}
\begin{tabular}{llll}
Tempo& 			In MIDI&		Transpose&	MIDI Clock\\
Note Speed& 		Out MIDI& 	Volume	&	Click\\
Swing& 			Control MIDI&	Length	&\\
\end{tabular}

\vspace{1em}
Of particular use are Tempo, Note Speed, Swing, and especially Click.


\begin{wrapfigure}{r}{3in}
\vspace{-4em}\includegraphics[width=3in]{recorder.pdf}
\vspace{-2em}\caption{\small Recorder Display}\vspace{-1em}
\label{recorder}
\end{wrapfigure}

\section {The Note Recorder}

The Note Recorder lets you record and play back approximately 64 notes\footnote{Why {\it approximately} 64? Because the recorder is recording MIDI NOTE ON and NOTE OFF messages.  If you provide 64 pairs of these messages, it'll take up exactly all the available memory.  But you could plausibly send a few more NOTE ON than NOTE OFF messages, in which case at the end of the recording all the remaining outstanding notes will be terminated.  This would allow you to squeeze in a few (perhaps 16) more notes in rare circumstances.  But I'd not rely on it.  BTW, if you looked carefully you would have noted that the Note Display in Figure~\ref{recorder} has space for 16 more notes too.} spread over 21 measures.\footnote{Why 21?  Because notes are stored with a timestamp of eleven bits, resulting in \(2^11 = 2048\) timesteps.  A single timestep is 1/24 of a quarter note, and a quarter note is 1/4 of a measure, so we have \(2048 / 24 / 4 = 21 1/3\) measures.  We round that down to 21.}  You can save up to two recordings on an Arduino Uno and nine recordings on a Mega.

The Note Recorder screen has two major parts.  The {\it Note Display} section indicates the current note number being played (and hints at the remaining notes available).  The {\it Measure Display} section indicates the current measure being played (and again hints at the remaining measures available).  When recording a song, each of these sections will contain a single lit LED indicating the current note or measure position.  When playing a song, the Note Display and Measure Display will each also contain an additional lit LED indicating the position of the end of the song.

\begin{wrapfigure}{r}{1in}
\includegraphics[width=1in]{none.pdf}
\vspace{-2em}\caption{\small None}\vspace{-1em}
\label{none}
\end{wrapfigure}

When you choose the Note Recorder from the root menu, you will be asked to load a recording from a slot.  You can do this, or you can create a new empty recording by choosing \texttt{-~-~-~-} (meaning {\it none}).  Slots are displayed by number, plus an \texttt{R} (for Recorder) or \texttt{S} (for Step Sequencer) if the slot is filled with a file by one of the two applications.  This letter blinks if the file isn't the type of your current application.

After this you will enter the Note Recorder.

\begin{wrapfigure}{r}{1.5in}
\includegraphics[width=1.5in]{recorderstatus.pdf}
\vspace{-2em}\caption{\small Recorder Status LEDs}\vspace{-4em}
\label{recorderstatus}
\end{wrapfigure}

The Note Recorder is always in one of four states, indicated the patterns at right:

\begin{itemize}
\item Stopped
\item Playing
\item Counting off beats prior to Recording
\item Recording
\end{itemize}

When you enter the Note Recorder, the state is initially Stopped.  To start playing a song, press the middle button.  To stop playing (or stop recording), you also press the middle button.  To start recording a song, Long Press the middle button.  The recorder will count off four beats for you, then start recording.

You can choose {\bf Repeat Mode} from a menu by Long-Pressing the Select (right) Button.  When in repeat mode, the recorder repeats the song it is playing at the measure boundary after it has finished playing the song.  Repeat mode also toggles an LED (see Figure~\ref{recorder}).

You can save your recorded song by pressing the Select Button.  Gizmo will ask you to select a slot to save in, then it will exit the Recorder.  Slots are again displayed by number, plus an \texttt{R} (for Recorder) or \texttt{S} (for Step Sequencer).

\paragraph{More Recorder Options}  
You can also choose the Options Menu from inside the Note Recorder: it's part of the menu brought up by Long-Pressing the Select button.  The Recorder responds to a number of options:

\vspace{1em}
\begin{tabular}{lll}
Tempo& 		In MIDI&		Transpose	\\
Click& 			Out MIDI& 	Volume	\\
MIDI Clock& 			Control MIDI	\\
\end{tabular}

\vspace{1em}
Of particular use to you will be Tempo and Click.  Click gives you an audible click track: and the count-in is also clicked for you.

\paragraph{Recording and Quantization}

The Recorder records nothing but MIDI NOTE ON and NOTE OFF messages.  It presently quantizes these messages, rounding them up or down to the nearest MIDI Clock pulse, that is, 1/24 of a quarter note.\footnote{This is pretty low-resolution, but there you have it.}

\section {The MIDI Gauge}

\begin{wrapfigure}{r}{3in}
\vspace{-6em}\includegraphics[width=3in]{GeneralGauge.pdf}
\vspace{-2em}\caption{\small General Gauge and Frequent MIDI Display}\vspace{-2em}
\label{generalgauge}
\end{wrapfigure}

The MIDI gauge listens to incoming MIDI messages and displays them on-screen.  This only occurs if the MIDI IN channel is something other than NO CHANNEL (See {\bf Options\(\boldsymbol\rightarrow\) In MIDI}, in Section \ref{options}).

There are three kinds of MIDI messages for purposes of display.  {\it General} messages take up the whole screen with a single message.  {\it Note} messages split the screen between pitch and velocity of the given note (on, off, polyphonic aftertouch).  Finally {\it Frequent} messages are simply displayed by lighting a specific LED to indicate their presence. 

Messages that are associated with a MIDI channel will have their channel number displayed at bottom left (see Figure \ref{generalgauge}).  Note that if the MIDI IN channel is something other than OMNI (ALL Channels), then only messages associated with the MIDI IN channel will be displayed.





\begin{wrapfigure}{r}{2in}
\vspace{-2em}\includegraphics[width=2in]{midichannel.pdf}
\vspace{-2em}\caption{\small MIDI Channel Values}\vspace{-5em}
\label{midichannelvalues}
\end{wrapfigure}


\paragraph{General Gauge Display}  These are simple displays which take up the entire screen.

\begin{itemize}
\item {\bf Pitch Bend}\quad The 14-bit value is displayed
\item {\bf Channel Aftertouch}.  \texttt{AT} is displayed, followed by the value.
\item {\bf Program Change}\quad \texttt{PC} is displayed, followed by the value.
\item {\bf Control Change (CC)}\quad If the CC parameter is for 7-bit values, then the value is shown first, then text is scrolled with the value, then \texttt{CC}, then the parameter.  If the CC parameter is for 14-bit values, the MSB of the value is shown first, then text is scrolled with the value, then \texttt{CC}, then the parameter, then in parentheses the 14-bit (MSB+LSB) value.
\item {\bf Registered Parameter Number (RPN)}\quad The value (7-bit MSB only) is shown first, then text is scrolled with the value (MSB), then \texttt{RPN}, then the parameter (14-bit), then in parentheses the 14-bit (MSB+LSB) value.
\item {\bf Non-Registered Parameter Number (NRPN)}\quad The value (7-bit MSB only) is shown first, then text is scrolled with the value (MSB), then \texttt{NRPN}, then the parameter (14-bit), then in parentheses the 14-bit (MSB+LSB) value.
\end{itemize}

\begin{itemize}
\item {\bf Channel Mode}\quad The value is shown first, then text is scrolled with the value, then \texttt{CHANNEL MODE}, then the parameter.  Channel Mode parameters are:\footnote{Why no text to describe these things?  Because the Arduino Uno is out of space.  And they're not that common.}
\begin{center}\begin{tabular}{@{}rll@{}}
{\it Parameter}&{\it Value}&{\it Meaning}\\[0.1em]
\hline\\[-0.9em]
120&Normally 0&All Sound Off\\
121&Normally 0&Reset All Controllers\\
122&0=Off, 1=On&Local Control (On/Off)\\
123&Normally 0&All Notes Off\\
124&Normally 0&Omni Mode Off\\
125&Normally 0&Omni Mode On\\
126&\(C\)&Mono Mode On, with \(C\) channels total\\
127&Normally 0&Poly Mode On\\
\end{tabular}\end{center}
\item {\bf System Exclusive}\quad \texttt{SYSX} is displayed.\footnote{Note that the version of the MIDI library on which Gizmo relies presently doesn't handle large System Exclusive messages well.}
\item {\bf Song Position}\quad \texttt{SPOS} is displayed.
\item {\bf Song Select}\quad \texttt{SSEL} is displayed.
\item {\bf Tune Request}\quad \texttt{TREQ} is displayed.
\item {\bf Start}\quad \texttt{STRT} is displayed.
\item {\bf Continue}\quad \texttt{CONT} is displayed.
\item {\bf Stop}\quad \texttt{STOP} is displayed.
\item {\bf System Reset}\quad \texttt{RSET} is displayed.
\end{itemize}



\begin{wrapfigure}{r}{3in}
\vspace{-5em}\includegraphics[width=3in]{NoteGauge.pdf}
\vspace{-2em}\caption{\small Note Gauge Layout}\vspace{-1em}
\label{notegauge}
\end{wrapfigure}


\paragraph{Note Display}

Note messages are displayed with both the {\bf pitch} (at left) and the {\bf velocity} (at right) of the note value, plus the particular kind of message in question: {\bf Note On, Note Off,} and {\bf Polyphonic Aftertouch}.

\begin{wrapfigure}{r}{1.6in}
\vspace{-1em}\includegraphics[width=1.6in]{notefunction.pdf}
\vspace{-2em}\caption{\small Note Function}\vspace{-2em}
\label{notefunction}
\end{wrapfigure}

How do you distinguish between a Note On, Note Off, or Polyphonic Aftertouch message?  Figure~\ref{notegauge} shows a four-LED region, in orange, where a certain pattern is displayed to indicate this.  The patterns are shown in Figure~\ref{notefunction}.

Note pitches are displayed by showing the note (such as B$\flat$), with the octave number directly below it.  MIDI has eleven different octaves (0--10): Middle C is the bottom note of octave 5.  The octave number is displayed as shown in Figure~\ref{octave}.  As usual, the MIDI In Channel is shown below that.


\paragraph{Frequent MIDI Display}  These are messages which are too fast to display usefully in most cases.  In each case, a specific LED will be toggled (see Figure~\ref{generalgauge}).

\paragraph{}\vspace{-2em}\begin{itemize}
\item {\bf Active Sensing}
\item {\bf MIDI Time Code}
\item {\bf MIDI Clock}
\end{itemize}



\section {The Controller}
\label{controller}

\begin{wrapfigure}{r}{1.8in}
\vspace{-11em}\includegraphics[width=1.8in]{octave.pdf}
\vspace{-2em}\caption{\small Octave Values}\vspace{-1em}
\label{octave}
\end{wrapfigure}

The Controller allows you to assign to each of two buttons and two potentiometers any NRPN, RPN, Control Change (CC), or Program Change (PC) value.  These assignments are stored in Flash memory and survive reboots and power cycling.

The Controller interacts with Bypass mode differently than other applications.  In bypass mode, the Controller passes all the incoming MIDI data to MIDI out as usual.  But it will still emit values when you press the buttons or more the knobs.  This is because it's typical that you'd want to use Controller to (for example) manipulate the VCF on an analog synthesizer while playing it via a remote keyboard.  To do this, you'd plug the keyboard into Gizmo, plug Gizmo into the synthesizer, and turn on bypass mode.

	\begin{description}

	\item{\bf Go}
	
	This enters the controller proper.  Turning the left or right knobs, or pressing the Select or Middle buttons, will cause them to issue MIDI messages as assigned (below).  If you press the Back button, you can exit the controller.
	
	When you turn a knob or press a button, and the knob/button has had its control type set to something other than OFF, then the appropriate MIDI message is sent, and the value of the message is displayed as a number on-screen.  Messages are only sent if the MIDI OUT channel is something other than NO CHANNEL (See {\bf Options\(\boldsymbol\rightarrow\) Out MIDI}, in Section \ref{options}).
	
	Note that for the time being the controller will only send (MSB) values between 0...127.  It cannot send 14-bit NRPN and RPN values.  This is largely a limitation of the resolution of the potentiometers.

	\item{\bf Left Knob} and {\bf Right Knob}
	
	These submenus let you select your control type and parameter number for the left and right potentiometers.
	
	In each, you are first given the option of what kind of {\bf control type} (OFF, CC, NRPN, RPN, PC) you would like to manipulate with the left knob.  If OFF, then turning the knob will not do anything.

	After you have selected a control type, if you have selected CC, NRPN, or RPN, you will be then asked to select a {\bf controller parameter number}.  (Otherwise ,for PC and OFF, you will go back to the top-level Controller Menu).  CC parameter numbers may range 0 ... 119.  NRPN and RPN parameter numbers may range 0 ... 16383. When you have completed this, you will be sent back to the top-level Controller menu.

	\item{\bf Middle Button} and {\bf Select Button}
	
	Thees submenus let you select your control type, parameter number, and on/off values for the middle and select buttons.
	
	In each, you are first given the option of what kind of {\bf control type} (OFF, CC, NRPN, RPN, PC) you would like to manipulate with the middle button.  If OFF, then pressing the button will not do anything.

	After you have selected a control type, if you have selected CC, NRPN, or RPN, you will be then asked to select a {\bf controller parameter number}.  (Otherwise ,for PC and OFF, you will go back to the top-level Controller Menu).  CC parameter numbers may range 0 ... 119.  NRPN and RPN parameter numbers may range 0 ... 16383.
	
	When you have completed this, you will be then asked to enter the value sent when the button is {\bf pressed}.  This value must be 0...127 (the Controller does not send 14-bit values).  Afterwards, you will be similarly asked to enter the value sent when the button is {\bf pressed again} (the buttons act as toggles).  When you have completed this, you will be sent back to the top-level Controller menu.
	\end{description}

\section {The Keyboard Splitter}
\label{splitter}

\begin{wrapfigure}{r}{3in}
\vspace{-2em}\includegraphics[width=3in]{split}
\vspace{-2em}\caption{\small Splitter Layout}
\label{splitter}
\end{wrapfigure}

\textit{[Mega Only]}\qquad The Splitter allows lets you (1) split your keyboard into two zones, each controlling a different MIDI channel; or (2) split your keyboard into {\it three} zones, two controlling different MIDI channels and the third (middle) zone playing both MIDI channels simultaneously\footnote{This is commonly known as {\it layering.}}; or (3) fade or balance your entire keyboard such that if you play a note quietly, one channel will predominate, but if you play loudly, the other channel will predominate.

\begin{wrapfigure}{r}{1.5in}
\includegraphics[width=1.5in]{Fade}
\vspace{-2em}\caption{\small Cumulative Velocity the MIDI Out Channels when fading.  Default is {\color{red}red}, and Alternative is {\color{blue} blue}.}
\vspace{-2em}
\label{splitter}
\end{wrapfigure}

The two MIDI channels in question are the Default MIDI Out Channel (defined in the Options, see Section \ref{options}), and an Alternative MIDI Out Channel which you specify in the Splitter.  If you're splitting your keyboard, you'll also specify up to two notes, {\bf Split Note A} and {\bf Split Note B}  (Split Note B is optional), and whether Control Change, Aftertouch, and the like go to the left split region or the right split region.

\begin{itemize}
\item  If you want to just split your keyboard, specify only Split Note A at the split point.  Notes in the left (lower) region will go our the Alternative MIDI Channel, and notes in the right (upper) region will go out the Default MIDI Channel.

\item If you want your keyboard to play {\bf both} channels, make Split Note A be the lowest note on your keyboard, and Split Note B be the highest note on your keyboard.    Notes played on the keyboard will be sent out both the Alternative MIDI and Default MIDI Channels. 

\item If you want your keyboard to have {\bf three regions}, a left-only region, a right-only region and a middle region where both are played, make Split Note A be the lower note in the middle region and Split Note B be the higher note.   Notes in the left (lower) region will go our the Alternative MIDI Channel, notes in the right (upper) region will go out the Default MIDI Channel, and notes in the middle region will go out both channels.

\item Note that it is possible to also give you keyboard three regions where in the middle region {\bf neither} channel is played.  This is probably not very useful.  But it happens when you define Split Note A to be {\it above} (to the right of) Split Note B.

\item Finally, if you want to fade or balance your keyboard so that notes on the two channels are played with opposite velocities, you simply choose this option (see below).  You don't need to specify either Split Note A or B, but you will need to define the alternative MIDI Channel.
\end{itemize}

\paragraph{Fade Behavior}  The current fade behavior function is:
\[
\begin{split}
\text{\it Default MIDI Velocity} &= Velocity - \text{\it Velocity} \times \left( \text{\it Velocity} + 1 \right) / 128\\
\text{\it Alternative MIDI Velocity} &= \text{\it Velocity} \times \left( \text{\it Velocity} + 1 \right) / 128
\end{split}
\]

This is a fancy way of saying that the total velocity of both channels is the same as the input velocity, and that the ratio of the Alternative velocity versus the Default velocity increases linearly as you get closer to 127.  I don't know if this is the right approach, and certainly different keyboards synthesizers have different and often nonlinear velocity mappings.  But there you have it.  If you'd like to play with it, it's in the function \texttt{handleNoteOn(...)} in \texttt{TopLevel.cpp}.

\paragraph{Defining Notes and Channels}

If you press the Select Button, you'll be asked to enter a note.  This note will be Split Note A.  By default this note is Middle C.  You can back out if you don't want to enter the note.

\begin{wrapfigure}{r}{1in}
\vspace{-1em}\includegraphics[width=1in]{none}
\vspace{-2em}
\end{wrapfigure}

If you press the Middle Button you'll be asked to enter a note.  This note will be Split Note B.  By default this note is off.  You can back out if you don't want to enter the note.  You can also turn Split Note B off again by once again pressing the Middle Button, at which point you'll be presented with \texttt{-~-~-~-}.  You can also back out if you don't want remove Split Note B.

You define the Alternative MIDI Out channel by Long-Pressing the Select Button.  The Alternative MIDI Channel must be a value 1--16 (you can't choose Off).

The Splitter will display your two Split Notes, as shown in Figure \ref{splitter}.

\paragraph{Routing and Fading}

Long-pressing the Middle Button cycles through three options:
\begin{itemize}
\item Split/Layer the Keyboard, and route all CC, PC, Channel Aftertouch, and NRPN/RPN to the {\it right} (upper) split channel, that is, to the Default MIDI Out channel.  Pitch Bend is also routed to the right channel if you're doing a simple split, but if you're layering, it's routed to both channels.  Polyphonic Aftertouch is always routed to the channel handling the note in question, as are all Note On and Note Off messages.  The screen will show the Split A note, and will light the ``left'' LED (see Figure \ref{splitter})..
\item Split/Layer the Keyboard, and route all CC, PC, Channel Aftertouch, and NRPN/RPN to the {\it left} (upper) split channel, that is, to the alternative MIDI Out Channel.  Pitch Bend is also routed to the left channel if you're doing a simple split, but if you're layering, it's routed to both channels.  Polyphonic Aftertouch is always routed to the channel handling the note in question, as are all Note On and Note Off messages.  The screen will show the Split A note at left and the Split B note at right, and will light the ``left'' LED (see Figure \ref{splitter}).
\item Fade/Balance the Keyboard.  All CC, PC, Channel Aftertouch, and NRPN/RPN if routed to the alternative MIDI Out Channel.  Pitch Bend is routed to both channels, as is Polyphonic Aftertouch and all Note On and Note Off messages (though with inverted velocities as discussed earlier).  The screen will display \texttt{FADE}.
\end{itemize}

\section{The Thru Facility}
\label{thru}

\textit{[Mega Only]}\qquad The {\bf Thru Facility} passes MIDI through the box, but if MIDI data comes in via the MIDI In channel, it applies some interesting filters and then routes it out specially.  There are two filters at present.  First, you can {\it distribute} incoming notes, one by one, to separate MIDI channels.  Second, you can have Gizmo add {\it extra notes}, that is, when the note G comes in for example, Gizmo might send four G notes out (essentially at the same time).  You can also do both of these at the same time.

Where would you use such things?  The {\it distribute} facility is useful if you have multiple copies of the same monosynth, you could distribute notes to the various synths and basically treat them as if they were a polysynth\footnote{This is an idea directly stolen from MIDIPal.}  The {\it extra} facility is useful for fattening up a synth but not making it unison (mono).  For example, I have an Oberheim Matrix 1000 with six voices.  If I instruct Gizmo to provide two extra notes, then every time I press a note, Gizmo sends {\it three} copies of that note to the Oberheim, causing three voices to play together.  This basically turns the Matrix into a fatter two-voice polysynth.

I might add more unusual filters later.

	\begin{description}
	
	\item{\bf Go}\\
		When you have set up your extra notes and note distribution (see below), if you choose {\bf Go}, the Thru filter starts and you can start playing.  You'll see the word \texttt{PLAY} on the LED.

	\item{\bf Extra Notes}\\
		This lets you select how many {\it additional} simultaneous note-on (and note-off) messages Gizmo should send down a given channel in response to you playing a note.
		\begin{description}
		\item{\bf \textit{Choose}}: No Extra Notes, or 1...31 \hspace{3in}\smash{\includegraphics[width=1in]{none.pdf}} 
		\end{description}

	\item{\bf Distribute Notes}\\
		This lets you select the number of MIDI channels, beyond the Default MIDI Out channel, over which to distribute notes.  Let's say you selected 2, and the Default MIDI Out channel was \(M\).  This means that three MIDI channels will be used: \(M\), \(M+1\), and \(M+2\).  This wraps around: if, say, \(M = 15\), then it wraps around: the channels used would be 15, 16, and 1).
		
		If you played the five notes A, B, C, D, E, Gizmo would send A down channel \(M\), then send B down channel \(M + 1\), then send C down channel \(M + 2\), then send D down channel \(M\) again (wrapping around), then E down channel \(M+1\) again, and so on.  This would let you essentially use Mono devices listening in on the three MIDI Channels as a 3-voice polyphonic device.
		
		\begin{description}
		\item{\bf \textit{Choose}}: No Extra Channels, or 1...15 \hspace{2.8in}\smash{\includegraphics[width=1in]{none.pdf}} 
		\end{description}
		
	\end{description}

\section {Options}
\label{options}

	Options sets global parameters for the device.  Once set, these parameters are stored in Flash memory and so will survive a power cycle.  Some options can be also accessed from certain other applications as a convenience.  
	
	\begin{description}


	\item{\bf Tempo}\\
	If Gizmo is following its own internal clock rather than relying on an external MIDI clock , this specifies how fast a quarter note is.  If Gizmo is following an external MIDI clock, this value is ignored, and the MIDI clock's specification of a quarter note is used instead.  (See {\bf Options\(\boldsymbol\rightarrow\)MIDI~Clock}).  When not in {\bf Bypass Mode}, the tempo is shown by the pulsing on/off of the {\bf Beat / Bypass Light} (see Figure~\ref{HighLevelGizmoLayout}). 
	
		\begin{description}
		\item{\bf \textit{Choose}}: 1 ... 999 Beats Per Minute.  Note that this is a large number, and so may require you to choose it with the left potentiometer, then fine-tune it with the right potentiometer.\footnote{Gizmo can go lots faster than 999: in theory it could go clear to 31200 or so.  But then your synthesizer would explode and we wouldn't want that.}
		\end{description}
		
	\item{\bf Note Speed}\\
		Various applications (arpeggiators, step sequencers)
		produce notes at a certain rate relative to the tempo.
		For example, though the tempo may specify that a
		quarter note is set to 120 Beats Per Minute, the 
		arpeggiator might be generating eighth notes and
		so is producing notes at twice that speed.  You specify the 
		note speed here.
		
		Note speed is shown by pulsing the {\bf Note Pulse Light} (Figure~\ref{HighLevelGizmoLayout}).

		\begin{description}
			\item{\bf \textit{Choose}}:
			\begin{tabbing}
			Eighth Triplet (Triplet 64th Note)\hspace{3em}\=\(\myfrac{1}{24}\)\hspace{1em}\=Beat\\
			Quarter Triplet (Triplet 32nd Note)\>\(\myfrac{1}{12}\)\>Beat\\
			Thirty-Second Note\>\(\myfrac{1}{8}\)\>Beat\hspace{0.7in}\=\smash{\includegraphics[width=2in]{NoteSpeed1.pdf}}\\
			Half Triplet (Triplet 16th Note)\>\(\myfrac{1}{6}\)\>Beat\\
			\\
			Sixteenth Note\>\(\myfrac{1}{4}\)\>Beat\\
			Triplet\>\(\myfrac{1}{3}\)\>Beat\\
			Eighth Note\>\(\myfrac{1}{2}\)\>Beat\>\smash{\includegraphics[width=2in]{NoteSpeed2.pdf}}\\
			Quarter Note\>\(1\)\>Beat (duh)\\
			\\
			Quarter Note Tied to Triplet\>\(1\ \, \myfrac{1}{3}\)\>Beats\\
			Dotted Quarter Note\>\(1\ \, \myfrac{1}{2}\)\>Beats\\
			Half Note\>\(2\)\>Beats\>\smash{\includegraphics[width=2in]{NoteSpeed3.pdf}}\\
			Half Note Tied to Two Triplets\>\(2\ \, \myfrac{2}{3}\)\>Beats\\
			\\
			Dotted Half Note\>\(3\)\>Beats\\
			Whole Note\>\(4\)\>Beats\\
			Dotted Whole Note\>\(6\)\>Beats\>\smash{\includegraphics[width=2in]{NoteSpeed4.pdf}}\\
			Double Whole Note\>\(8\)\>Beats
			\end{tabbing}
		\end{description}

	\item{\bf Swing}\\
			Swing, or {\bf syncopation}, is the degree to which every other note is delayed.  0\% means no swing at all.  100\% means so much swing that the even note plays at the same time as the next even note.  That's a lot.
		\begin{description}
		\item{\bf \textit{Choose}}: 0\% ... 100\% (larger percentages are more swing, that is, longer delay every other note)
		\end{description}

		Swing is only applied if the Note Speed is set to thirty-second, sixteenth, eighth, quarter, or half-notes.

	\item{\bf Transpose}\\
		Here you can state that any notes generated by Gizmo be {\bf transposed} up or down by as much as 60 notes. If a note is transposed to the point that it exceeds the MIDI range, it is not played.  When appropriate, some applications (such as the MIDI Gauge) ignore this option. 
		\begin{description}
		\item{\bf \textit{Choose}}: \(-60\) ... 60 
		\end{description}

	\item{\bf Volume}\\
		Like Transpose, here you can state that any notes generated by Gizmo have their volume changed by multiplying their MIDI velocity by some value.  If a resulting note velocity exceeds the maximum (127), it is bounded to 127.
			
			\begin{description}
			\item{\bf \textit{Choose}}:
			\begin{tabbing}
			1/8\\
			1/4\\
			1/2\hspace{3.65in}\smash{\includegraphics[width=1.7in]{fractions2.pdf}}\\
			1\qquad (default)\\
			2\\
			4\\
			8\\
			\end{tabbing}
		\end{description}

	\item{\bf In MIDI}\,\footnote{{\it Why aren't these called MIDI In and MIDI Out?}  Because then they'd be indistinguishable on the menu screen before the text started scrolling.}\\
		Many applications expect notes and other controls to come in via a specific input MIDI channel.  You specify it here.
		\begin{description}
		\item{\bf \textit{Choose}}: No Channel, Channels 1...16, or ALL Channels\hspace{0.25in}\smash{\includegraphics[width=1in]{none.pdf}}\hspace{0.2in}\smash{\includegraphics[width=1in]{all.pdf}}\\
		\end{description}
	\item{\bf Out MIDI}\\
		Many applications emit notes etc. via a specific output MIDI channel.  You specify it here.  Other applications can emit notes on several different channels, in which case this value determines the default channel used.
		\begin{description}
		\item{\bf \textit{Choose}}: No Channel, or Channels 1...16\hspace{0.5in}\smash{\includegraphics[width=1in]{none.pdf}}
		\end{description}
	\item{\bf Control MIDI}\\
		The arduino can be controlled via NRPN messages.  You specify the channel here on which it will listen for them
		\begin{description}
		\item{\bf \textit{Choose}}: No Channel, or Channels 1...16\hspace{0.5in}\smash{\includegraphics[width=1in]{none.pdf}}
		\end{description}
		
		The NRPN Messages are:
		
		\noindent\begin{tabular}{@{}rlllp{2.4in}@{}}
		{\it NRPN Parameter}&{\it Min  Value}&{\it Max Value}&{\it MSB/LSB}&{\it Description}\\[0.1em]
		\hline\\[-0.9em]
		0	&0 (released)	&1 (pressed) & MSB& Press/Release the Back Button\\
		1	&0 (released)	&1 (pressed) & MSB & Press/Release the Middle Button\\
		2	&0 (released)	&1 (pressed) & MSB & Press/Release the Select Button\\
		3	&0	& 1023 & MSB + LSB & Turn the Left Potentiometer\\
		4	&0	& 1023 & MSB + LSB & Turn the Right Potentiometer\\
		5	&0 (un-bypass)	&1 (bypass) & MSB& Toggle Bypass\\
		6	& Any Value &Any Value & MSB& Unlock Potentiometers.  When an NRPN message is received, normally the on-board potentiometers are locked so that turning them has no effect.  Pressing an on-board button unlocks them: so does sending this NRPN message.\\
		7	&	Any Value &Any Value & MSB& Start Clock.  If Gizmo's clock is stopped, resets and starts it.  When appropriate, sends a MIDI START message.\\
		8	& Any Value	&Any Value & MSB& Stop Clock.  If Gizmo's clock is playing, stops it.  When appropriate, sends a MIDI STOP message.\\
		9	&	Any Value &Any Value & MSB & Continue Clock.    If Gizmo's clock is stopped, continues it.  When appropriate, sends a MIDI CONTINUE message.\\
		\end{tabular}\\
		\\
		\\
		You can also start/stop/continue the clock via by pressing the Middle button while in the Options menu (see {\bf Starting, Stopping, and Continuing the Clock} at the end of this section).
		
	\item{\bf MIDI Clock}\\
		Gizmo can respond to a MIDI clock, ignore it, or emit its own MIDI clock.  The use of the {\bf Options\(\boldsymbol\rightarrow\)Tempo} setting will depend on the setting chosen here.
		\begin{description}
			\item{\bf \textit{Choose}}:
			\begin{tabbing}
			Ignore\hspace{0.5in}\=Use an internal clock but let any external MIDI clock pass through.\\
			Use\>Use an external MIDI clock and also let it pass through.\\
			Divide\>Use an external MIDI Clock.  Don't pass it through, but instead send out MIDI\\
			\>~~~~~~clock messages whenever a Note Pulse occurs.\\
			Consume\>Use an external MIDI clock but don't let it pass through.\\
			Generate\>Use an internal clock and emit it as a MIDI clock.\\
			\>~~~~~~Don't let any external MIDI clock pass through.\\	
			Block\>Use an internal clock but don't emit as a MIDI clock.\\
			\>~~~~~~Don't let any external MIDI clock pass through.\\
			\end{tabbing}
		\end{description}

The {\bf Divide} option is special: instead of sending MIDI clock every pulse, it sends MIDI clock every note pulse, thus slowing down (dividing) the clock signal.  To set the Note Pulse, see the {\bf Note Speed} option earlier.  The division is as follows:

\begin{center}\begin{tabular}{@{}lr@{}}
{\it Note Speed}&{\it Divisor}\\
				\hline\\[-0.9em]

Eighth Triplet (Triplet 64th Note)	&1\\
Quarter Triplet (Triplet 32nd Note)	&2\\
Thirty-Second Note				&3\\
Half Triplet (Triplet 16th Note)		&4\\
Sixteenth Note					&6\\
Triplet						&8\\
Eighth Note					&12\\
Quarter Note					&24\\
Quarter Note Tied to Triplet		&32\\
Dotted Quarter Note				&36\\
Half Note						&48\\
Half Note Tied to Two Triplets		&64\\
Dotted Half Note				&72\\
Whole Note					&96\\
Dotted Whole Note				&144\\
Double Whole Note				&192\\
\end{tabular}\end{center}

		You can also start/stop/continue the clock via by pressing the Middle button while in the Options menu (see {\bf Starting, Stopping, and Continuing the Clock} at the end of this section), or by sending an appropriate NRPN message (see {\bf Control MIDI} earlier).
		
	\item{\bf Click} or {\bf No Click}\\
		This toggles the note played by Gizmo to provide a click track for applications such as the Step Sequencer or Recorder.  If you select {\bf Click}, you will be asked to play a {\bf note}.  The note pitch and velocity form the click.  If you select {\bf No Click}, this turns off the click track.
		
\hspace{\fill}\smash{\\\includegraphics[width=1in]{note.pdf}}\vspace{-1.5em}

	\item{\bf Screen Brightness}\\
		This changes the brightness of the LED matrix.
		\begin{description}
		\item{\bf \textit{Choose}}: 1...16 (higher values are brighter)
		\end{description}
		
		
	\item{\bf \textit{Mega Only:}\quad Menu Delay}\\
		Gizmo often will display text (such as \texttt{SWING}) by filling the screen with the first few letters (in this case, \texttt{SWI}), then {\it delaying} for a certain interval, then scrolling through the whole word horizontally. You can specify how long that delay is.
		\begin{description}
			\item{\bf \textit{Choose}}:
			\begin{tabbing}
			0\hspace{2em}\=Seconds\hspace{1in}\=(scroll immediately)\\
			1/8\>Seconds\\
			1/4\>Seconds\\
			1/3\>Seconds\>\hspace{1.5in}\=\smash{\includegraphics[width=2in]{fractions.pdf}}\\
			1/2\>Seconds\\
			1\>Second\>(default)\\
			2\>Seconds\\
			3\>Seconds\\
			4\>Seconds\\
			8\>Seconds\\
			\(\infty\)\>Seconds\>(never scroll)\>\smash{\includegraphics[width=0.48in]{infinity}}\\
			\end{tabbing}
		\end{description}


	\item{\bf \textit{Mega Only:}\quad Voltage} (or {\bf No Voltage})\\
		Toggles whether Gizmo will send control voltage information out via the optional DACs in response to notes played or to Controller potentiometers being turned.  Note that control voltage is sent in response to an I2C signal, and this may cause a very slight MIDI delay.
		
	\item{\bf The About Screen}\\
		Presently says:\quad \texttt{GIZMO V1 (C) 2016 BY SEAN LUKE}
	\end{description}

	\section{Starting, Stopping, and Continuing the Clock}
	\label{startingclock}

		You can start, stop, and continue the clock via NRPN as discussed earlier (see {\bf Control MIDI} in Section \ref{options}).   But there's another ``secret'' way to do it using Gizmo's buttons.  If you are in the Options menu, long-press the Middle Button: this toggles {\bf starting} or {\bf stopping} the clock (you can't continue the clock this way).  Remember that you can also get to the Options menu from inside the Arpeggiator, Step Sequencer, or Recorder.
		
		Neither long-pressing the Middle Button, nor using NRPN to start/stop/continue, will do anything if the MIDI Clock is set to {\bf Use, Divide,} or {\bf Consume} (in which case the MIDI Clock is out of your control).  See {\bf MIDI Clock} in Section \ref{options}.
				
		Note that starting and stopping the clock doesn't mean that an application starts or stops playing.  However, the Start message {\it will} cause applications to reset themselves to the beginning when appropriate.   This means that if you want to start/stop/pause/continue the Step Sequencer (for example), you can do it as follows: first stop the clock, then start the sequencer playing.  Note that it won't play because there's no clock pulses.  But now you can start, stop, and pause/continue the sequencer by controlling the Start/Stop/Continue clock messages directly.
		
\section{Buttons and Debouncing}

The SparkFun MIDI Shield uses poor quality buttons.  I can push down on the button and wiggle it and it'll think it's been pressed multiple times.  As a result occasionally pressing a button will send two or more button-presses to Gizmo.

Gizmo has a debounce mechanism built-in: it ignores secondary apparent button-presses if they occurred within \(N\) milliseconds of an earlier button release.  At present \(N\) is set to 100ms.  If you need to press buttons faster than ten times a second (and it's possible) set the \texttt{BUTTON\_PRESSED\_COUNTDOWN\_DEBOUNCE} constant (in \texttt{TopLevel.h}) to something smaller.

\section{Space on the Arduino Uno}
Gimzo uses nearly all available space on the Arduino Uno.  If you want to do any serious development, you should get an Arduino Mega instead, which has lots of available space.  But if you'd like extend Gizmo with a little tiny application on the Uno, you can remove the Note Recorder.  In \texttt{All.h} just uncomment the line that says \texttt{NO\_RECORDER} and you'll get about 1.5K of code space.  It's not a lot but should be enough to write a small app. Note that ``Recorder'' will still appear in the Menu: it just won't do anything.  To get rid of that you need to modify the state machine a bit.  I'd not bother.

\section{Development Status}

\begin{center}
{
\setlength\extrarowheight{0.3em}
\begin{tabular}{@{}llllp{2.8in}@{}}
&&\multicolumn{2}{c}{\it Testing}&\\[0.1em]
\cline{3-4}\\[-0.9em]
{\it Item} & {\it Coded} & {\it Cursory} & {\it Significant} & {\it Known Bugs and Issues}\\[0.1em]
		\hline\\[-1.5em]
Arpeggiator& $\checkmark$ & $\checkmark$ & $\checkmark$  & \\
Step Sequencer& $\checkmark$ & $\checkmark$ & $\checkmark$ & Occasionally the beat doesn't match up with the play cursor; this particularly happens under Swing.  I've not been able to track it down.\\
Recorder& $\checkmark$ & $\checkmark$ & $\checkmark$ & Rarely a note will get stuck after recording: appears that a NOTE OFF message isn't getting matched with the ID of the corresponding NOTE ON message when entered into memory.  
I also think there may be a bug causing chords near the start of the song to be rushed a bit.  Perhaps timing is being delayed for some reason?\\
Gauge& $\checkmark$ & $\checkmark$ &\\ 
Controller& $\checkmark$ & $\checkmark$ & & \\ 
NRPN Control  &  $\checkmark$ & $\checkmark$ &  $\checkmark$ &\\
Transposition / Volume  &  $\checkmark$ & $\checkmark$ &  $\checkmark$ &\\
Brightness / Delay  &  $\checkmark$ & $\checkmark$ &  $\checkmark$ &\\
Tempo / Note Speed &  $\checkmark$ & $\checkmark$ &  $\checkmark$ &\\
MIDI In/Out Channels &  $\checkmark$ & $\checkmark$ &  & No test of OMNI yet\\
Bypass &  $\checkmark$ & $\checkmark$ &  $\checkmark$ &\\
Display Subsystem  &  $\checkmark$ & $\checkmark$ &  $\checkmark$ &\\
Timing  Subsystem &  $\checkmark$ & $\checkmark$ &  $\checkmark$ &\\
Interface Utilities &  $\checkmark$ & $\checkmark$ &  $\checkmark$ & \\
Storage Facility &  $\checkmark$ & $\checkmark$ &  $\checkmark$ &\\
MIDI Clock Control& $\checkmark$ & $\checkmark$ & &\\
Keyboard Splitting& $\checkmark$ & $\checkmark$  & & Works well but the fade option is perhaps a bit unnecessary and weird.\\
Control Voltage& $\checkmark$ & & & Untested\\
Thru Facility & $\checkmark$ & $\checkmark$ & & \\
\end{tabular}
}
\end{center}


\end{document}
